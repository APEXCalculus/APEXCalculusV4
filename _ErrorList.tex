%%
%%  Look for REVISIT to find places where changes may/not have been made
%$  and revisiting is needed.
%%

(1)  Math errors:
(2)  1.2 Example7 para-1 line1: $\delta=\leq\frac\epsilon5$.
% Fixed before 10/16/17; changed to \delta = \frac\epsilon/5
% Check mbx

(3)  2.4 Exercise33: g(0)=1, so the tangent line is $y = -9x+1$ and the normal line is $y = x/9+1$
% Fixed before 10/16/17  Problem statment has point (0,1).
% Check mbx   Reference .tex

(4)  2.6 eq 2.1 three extra )
% Fixed on 10/16/17
% check mbx    Need to delete 3(?) instances of extra ')'

(5)  3.1 p127 example81: the continuity of $f$ should be noted somewhere, somehow
% Fixed on 10/16/17 by adding note to check continuity at break
% check mbx   reference .tex

(6)  3.1 p128 example82 para4 L-1: "The only values to fall in the given interval ... are $0$ and $\pm\pi$,"
% Fixed on 10/16/17   Changed a \ldots to ``etc.'' (see .tex)
%                     Changed wording at end of that paragraph
% check mbx    reference .tex

(7)  3.3 p139 Th30.3: Assumes that $\fp\neq0$ around $c$.  We've changed it to "is positive before and after", and added a #4 with "negative".
% REVISIT  Not sure this is right. If \fp=0 near c, then we have another
% critical number ... think on it.
% no change to mbx yet

(8)  5.3 p219 example122 para-1 L-4: "partition size" isn't defined until p223 (and not with the words "partition size")
%  pg 214 ``divided (or, \textit{partitioned}) into''
%
%  definition of Riemann sums changed significantly, 
%  as is the paragraph preceeding it. 
%
%  check mbx   reference .tex
%
%  REVISIT this to make sure rest of chapter flows.

(9)  5.3 pg. 210 ``Let's use 4 rectangles of equal width of 1'' is awkward.
% 	fixed on 10/19/17  Changed to ``rectangles with an equal''
%		check mbx, reference .tex

(10)  5.3 pg 224  first bullet: remove ``is''   Third bullet: remove ``:''
%  fixed 10/19/17
%   check mbx, reference .tex

(11)  8.3 exercise 4    need $0\leq b_n$ to be correct
%		fixed 10/23/17
%		check mbx, reference .tex  (just added 0\leq before b_n)

(12)  7.4 p373 second displayed equation should have $f'(c_i)^2$
%   fixed 10/23/17   just added '^2'
%   check mbx, no need to reference .tex


(13)  Typos:
(14)  Preface p.viii para3 L-2: "If you wish to deactiv_at_e the interactivity"
% fixed on 10/23/17
% not sure about mbx ... where is the preface? 

(15)  1.1 p3 para4 L1: "allow us _to_ explore"
% already fixed in V3

(16)  1.1 p5 example3 para1 L1: "a table are given _in_ Figure"
% fixed 10/23/17
% check mbx

(17)  1.3 p22 para-2 L3: "the only place _where_ work is necessary"
% fixed 10/23/17  changed the wording of the paragraph along with added
% the word 'where'. 
% check mbx  be sure to reference .tex

(18)  1.3 p25 Example14 Solution para1 L-1: "an-d- indeterminate form."
% fixed 10/23/17   removed 'd' from 'and'
% check mbx

(19)  1.4 p33 para-2 L2: "limits of f, as x approaches 1, -is- _are_ 0"
% fixed 10/23/17  changed the word 'is' to 'are' since it is plural
% check mbx

(20)  2.1 Example39 sol'n p67 para2 L5: "switches from one piece to _the_ other"
% fixed 10/23/17   added the word 'the'
% check mbx

(21)  2.4 Example52 para-1 L3: "Each term-s-"
% fixed 10/23/17   removed 's'
% check mbx

(22)  3.1 p123 para1 L-2: "we might want to know -how- the highest/lowest values"
% fixed 10/23/17   removed 'how'
% check mbx
(23)  3.1 p125 example79 L-2: "_is_ the minimum value"
% fixed 10/23/17   added 'is'
% check mbx

(24)  3.3 p137 margin note: "Theorem 29 _parts_1_and_2_"
% fixed 10/23/17   changed sentence to 'Parts 1 \& 2 of Theorem 
% \ref{thm:incr_decr} also hold if'
% check mbx

(25)  4.3 p179 q2: "extreme values of _a_ function"
% fixed 10/23/17   added 'a', changed 'have' to 'has'
% check mbx

(26)  4.4 p185 para-1 L-1: "Be_ing_ able to do so"
% 10/23/17 added the 'ing' to being
% check mbx

(27)  5.1 p194 L-2: "there are infinite_ly_ _many_ derivatives"
% 10/23/17  changed as suggested
% check mbx

(28)  5.1 example 37 solution: extra ) shouldn't be there
% 10/23/17   fixed solution. The extra ')' was only
% part of the problem; the output was MMA and 
% 'simplified' in a very non-intuitive way. Changed
% to form more suitable for students and the way they answer.
% Answer: $2x^4+\cos x+\frac{2^x}{(\ln 2)^2}+(5-\frac1{\ln 2})x -1 - \frac{1}{(\ln 2)^2}$
% check mbx; use above to replace answer

(29)  5.2 p205 L1: "geometry _to_ compute"
% 10/23/17  added 'to'
% check mbx

(30)  6.1 p259 example144 sol'n para1 L1: "View this _as_ a composition of functions"
% 10/23/17  changed to 'View the integrand as the composition...'
% check mbx

(31)  12.2 limit definition add quotes (see my book)

(32)  12.2 exercise 1 make 'point' plural

(33)  7 p345 para-1 L-1: refers to Section 5.5.4
% Already fixed by 10/25/17
% look at mbx, I guess

(34)  7.1 p346 para1 L2: ditto
% already fixed ditto above
% ditto

(35)  7.5 p378 para-2 L2: "go to zero give_s_ an exact value"
% 10/25/17 added the 's'
% check mbx

(36)  7.5 p379 Example 219 para3 L2: "The -the- mass of the rope still hanging"
% fixed 10/25/17  removed extra 'the'
% check mbx

(37)  7.5 p379 Example 219 para-1 L2: "The rope weigh-t-s ... so the work applying this force"
% fixed 10/25/17 removed extra 't'
% check mbx

(38)  7.5 example 223 para-1 L-3: extra (
% fixed 10/25/17   removed '(' from front of 6240
% check mbx

(39)  7.5 example 225 sol'n para2 L3: "represents a this slice of water with" should be "thin"
% fixed 10/25/17   'this' to 'thin'
% check mbx

(40)  file 08_02_ex_25.tex is missing a subscript
% fixed 10/25/17   make the first sequence '\{a_n\}', not '\{a_\}'
% check mbx

(41)  file 11_02_ex_18.tex has a double comma
% fixed 10/25/17  remove extra ',' from problem statement
% check mbx

(42)  12.3 p693 example404 L3 has an extra (
% fixed 10/25/17  removed '(' at beginning of numerator
% check mbx

(43)  13.6 p800 Thm125 #2 L1 There should be a 1 in the subscript
% fixed 10/25/17  made 'g_1(x)'
% check mbx


(44)  Math Errors:
(45)  2.6 Figure 2.23: figimplicit6BW.pdf is missing the third tangent line which figimplicit6.pdf has
% fixed 10/26/17   for some reason third line was a repeat of second.
%  made third equation '{-1+0.5*x}'. 
% check mbx -  not sure how those images are generated.

(46)  5.2 p208 Exercise12: The other two exercises in this set use a positive number for an area under the x axis.
% fixed 10/26/17    changed the '-4' to '4' in fig_05_02_ex_12.tex
% check mbx - again, not sure how graphics are handled

(47)  5.3 p212 Example119 Solution2: $\sum a_i$ should be $\sum 3a_i-4$
% already fixed by 10/26/17
% check mbx?

(48)  In the calc 3 book eg. 405 #3 f_x is missing a 2x factor from using the chain rule.
% already fixed before John's email on 10/26/17
% mbx prob. ok ... but check

(49)  Typos:
(50)  1.5 p43 Example25 "Table 1.29" should be "Figure 1.29"
% fixed 10/26/17  changed 'Table' to 'Figure'
% check mbx

(51)  2.2 p102 Exercises 15-18: The instructions to 02_02_exset_01.tex has a spare ) at the end.
% fixed 10/26/17   removed ')'
% check mbx

(52)  2.6 p114 Example74 Sol'n Para-1 L-1: refers to the wrong figure
% fixed 10/26/17  reference should be for 'figimplicit10', not 9
% check mbx

(53)  3.2 p132 Theorem27 "be continuous function"
% fixed 10/26/17 added 'a' in between 'be' and 'continuous'
% check mbx

(54)  4.2 p168 Example99 para-2 L-2 "If the circle dime-sized"
% fixed 10/26/17  changed to 'circle is dime...'
% check mbx

(55)  7.2 p354 Figure7.10: "Cutting a slice in they pyramid"
% fixed 10/26/17 changed 'they' to 'the' in the caption of figcrossarea_1a
% check mbx

(56)  7.5: Joule and J should be capitalized (many places)
% Revisit? Couldn't find cases of this. Only 7.6?
% check mbx

(57)  7.5 p385 Example225 para-1 L-3: "Notice how the emptying of the bottom", emptying is misspelled
% Revisit? seems like 'emptying' is the correct spelling ... 
% check mbx

(58)  Math typos:
(59)  1.5 p38 Example 22: The floor function "returns the largest integer smaller than _or equal to_ the input".
% fixed 10/26/17  now: 'than, or equal to, '
% check mbx

(60)  8.1 p399 Example 231 Solution #2 para2 L-2: Should begin "That is: $2=1^2+1$".  The exponent is currently 1, which is mathematically correct, but not in context.
% fixed 10/26/17 changed '1^1' to '1^2'
% check mbx

(61)  8.1 p401 Example 232 Solution #2: We do not have the ability to conclude the sequence diverges.  For the same reason, exercises 30 and 33 can't be done.
% fixed 10/26/17 added an argument for the truth that the sequence
% a_n = \cos n does not converge. Will let this argument suffice as
% justification that exercises 30 and 33 can be answered. 
% mbx will definitely have to be adjusted

(62)  8.1 p405 Example 236 Solution para1: The inequalities should not be strict.
% fixed on 10/27/17   changed the '>' to '\geq' and '<' to '\leq'
% check mbx

(63)  Typos:
(64)  exercises/08_01_ex_25-30.tex are all missing a { before the answer, although this hasn't seemed to cause problems
% fixed on 10/27/17   odd ... added '{' to beginning of each answer portion
% don't think mbx cares

(65)  Math typos:
(66)  2.7 p119 para-2 L1 "In Figure 2.33 we see": $f^{-1}$ should have $(x)$.
% fixed on 10/27/17  added '(x)' to 'f^(-1)'
% check mbx

(67)  3.1 p130 #14: The graph has "1+", the function $f(x)$ does not.
% 10/28/17   the function graphed is '1+...' where the problem states
% 'f(x) = ...', with no '1+'. So '1+' added to front.
% check mbx

(68)  3.3 p137 between Thm29 & KI3: The IVT only applies if f' is continuous.  Instead, we need Darboux's Theorem (I had to look it up), which says that f' satisfies the conclusion of the IVT, even if it's not continuous.  We've gone with:

Let $a$ and $b$ be in $I$ where $\fp(a)>0$ and $\fp(b)<0$. If $\fp$ is continuous, then we can use the Intermediate Value Theorem. Even if $\fp$ isn't continuous, an advanced theorem\footnote{Darboux's Theorem} shows that $\fp$ has the Intermediate Value property.  Either way, there must be some value $c$ between $a$ and $b$ where $\fp(c) = 0$.  This leads us to the following method for finding intervals on which a function is increasing or decreasing.

% 10/28/17  Above is their fix; below is what is in the text. Didn't refer
% to Darboux.
%Let $f$ be differentiable on an interval $I$ and let $a$ and $b$ be in $I$ where $\fp(a)>0$ and $\fp(b)<0$. If $\fp$ is continuous on $[a,b]$, it follows from the Intermediate Value Theorem that there must be some value $c$ between $a$ and $b$ where $\fp(c) = 0$. (It turns out that this is still true even if $\fp$ is not continuous on $[a,b]$.)
%
% Definitely check mbx

(70)  5.4 p239 #54 solution: Should have a negative.
% 10/28/17  added '-' to solution
% check mbx

(71)  6.7 p329 Example 192 Solution 1 L1: I don't see how Theorem 3 relates.  Do you mean Theorem 5, part 3?
% 10/28/17  Made the reference '\ref{thm:special_limits}'
% check mbx

(72)  7.4 p373 para3 L3: the displayed equation should have $[f'(c_i)]^2$.
% 10/28/17   this is Error (12), already fixed
% 

(73)  8.1 p401 Figure 8.2(a): a_n should be a function of n, not x.
% 10/28/17  changed figure legend  of fig_seq4a.tex to 
% be function of n, not x.
% check mbx - not sure how figures are made

(74)  8.3 p434 exercise 40 solution: diverges
% 10/28/17   solution changed to Diverges, use Limit Comp. Test.
% check mbx

(75)  8.7 p467 L1: In the displayed equation, $f^n$ should be $f^{(n)}$
% 10/28/17  Yes, 'f^n' changed to 'f^{(n)}'
% check mbx

(76)  same page, L-5: same thing
% 10/28/17 ditto Error (75)
% check mbx

(77)  8.7 p471 para-2: In the displayed equation, $f^{(8)}$ should be $f^{(8)}(0)$
% 10/28/17   added the '(0)'
% check mbx

(78)  9.4 p529 Example 301 solution para2 L1: P(0,2) should be P(2,0)
% 10/28/17  changed to 'P(2,0)'
% check mbx

(79)  9.5 p539 L-1: The -1 should be -2.  The subsequent quadratic formula correctly uses -2.
% 10/28/17  the equation should have '...-2=0', not '-1=0'.
% check mbx

(80)  9.5 p540: In the decimal approximations associated with $\theta = \sin^{-1} (\frac{-1+\sqrt {33}}{8})$, $0.6399$ should be $0.6349$ (this occurs twice) and $3.7815$ should be $2.5067$.
% 10/28/17   fixed the numbers as suggested
% check mbx

(81)  9.5 p544 Example 309 solution: The 1/2 disappears two steps from the end.
% 10/28/17  put the '1/2' in the next to last step, and the final
% changes from '2\pi' to '\pi'.
% check mbx

(82)  11.3 p647 mnote: The units of acceleration are ft/s^2 and m/s^2.
% 10/28/17 Changed beginning of mnote to 'This text uses' then added
% '$^2$' to both s's of units. Changed the text to fit space better.
% check mbx
 
(83)  12.4 # 14 solution: There's a $\v$ that should just be $v$.
% 10/28/17  removed the '\v' and made it 'v..'
% check mbx

(84)  I've noticed that you sometimes have a habit of starting your examples:
(85)  \example{label}{Title}{
(86)  Problem statement. . .}
(87)  This causes there to be a space before the problem statement.  Which is really only noticeable once you're looking for it.
%
%  Thinking of a solution to this. Can't actually see a difference in Section 1.3.
%  An automated solutions seems best, but am leery of too much of a global
%  search/replace.
%  REVISIT

(88)  9.5 Figure fig:polcalc8 was too big (I changed the way mfigure, etc. worked)
% 11/4/17 changed the trim from '5mm' to '0mm'
% don't think this affects mbx

(89)  1.5 p48 para-2 L1: "one can come up _with_ 0/0".
% 11/2/17  added the 'with'
% check mbx

(90)  1.6 p49 para1 L4: You refer to l'Hospital's rule.  Everywhere else, you spell it l'H\^opital
% 11/2/17  added the '\^'
% check mbx

(91)  2.4 p95 Exercises 42-45: These belong to 2.2 (or should be labeled review)
% 11/2/17   added '\printreview' to line 20 of 02_04_exercises, right before
% the last exset.
% check mbx - not sure how exercises are handled there anyway

(92)  2.7 p120 Figure 2.31, Theorem 23: These look to be the only places where you put the argument to arc trig functions in parentheses.  Usually, you have $\sin^{-1}x$ instead of $\sin^{-1}(x)$.
% 11/2/17  In both the figure and the Thereom the arguments have '(x)'. In the
% figure they were inconsistently applied. Removed them all.
% check mbx

(93)  4.4 p187 Exercise 38: "Exercises 36" should be singular.
% 11/2/17 made singular
% check mbx

(94)  6.1 p267 Theorem 46: "Trigonomentric" is misspelled.
% 11/2/17 Yup. Fixed.
% check mbx

(95)  6.7 p329 solution L1: "This _is_ equivalent to a special limit given"
% 11/2/17 added 'is'
% mbx

(96)  7.2 p358 Figure 7.16: The label (b) is next to the bottom picture, not between the last two.
% 11/2/17  fixed by removing '%' after (b)  & before \\    Also lowered image
% from .63 to 63 and added '\\[5pt]' after (a) and (b) for better spacing
% mbx?

(97)  7.4 p369 para-2 L3: "segment as the-y- hypotenuse"
% 11/2/17  they to the
% mbx

(98)  8.1 p398 Example 230 solution 3: the pattern of signs opens a quotation that never closes
% 11/2/17  added the closing quotes, removed the ending comma. Now reads 
%  ' $-$, $-$, $\ldots$'' due '
% check mbx

(99)  9.2 p506 between Examples 285 & 286: "chose" should be "choose"
% 11/2/17   changed
%mbx

(100)  9.4 p529 Example 301 solution para1 L2: "we often choose" should be "chose" (or "added" should be "add")
% 11/2/17   Kept 'choose' and changed 'added' to 'add'
% mbx

(101)  9.4 p532 para2 L3: "a shape important _to_ the sensitivity of microphones"
% 11/4/17   added 'to'
% mbx

(102)  9.5 p538 para3 L1: "We are interested in the lines tangent _to_ a given graph"
% 11/4/17  added 'to'
% mbx

(103)  9.4 & 9.5: "cardiod" should be "cardioid" (2 times in 9.4, 4 times in 9.5, 3 times in 9.5's exercises)
% 11/4/17   changed 2 cases in 9.4; 2 cases in 9.5 text; exercises 09_05_ex_19,32,35.tex
% mbx

(104)  9.5: "cardiord" should be "cardioid"
% 11/4/17  already fixed?
%  check mbx anyway; not sure when it came in.

(105)  9.5 p541 "Area" para1 L2: "horzontal" should be "horizontal"
% 11/4/17   added 'i'
% mbx

(106)  10.1 p558 para3 L3: "coeffiecients" should be "coefficients"
%11/4/17  done
% mbx

(107)  10.6 p615 para1 L4: The paragraph doesn't end with a period.
% 11/4/17 added period
% mbx

(108)  12.4 p711 para2 L3: "enounter" should be "encounter" (which it is in the next paragraph; don't let that mislead you)
% 11/4/17   fixed
% mbx

(109)  12.7 p730 para3 L1: "In Figures 12.20" should be singular
% 11/4/17 fixed
% mbx

(110)  Back of book, p-6, Integration Rules #5: $n\neq -1$ is in there twice.
% 11/4/17 already fixed
% check mbx

(111)  p190 5.1 Theorem 34 only holds on an interval.  Similarly, the paragraph after should end ``Using Definition 19, we can say that on an interval''.
% Revisit. Made some changes, but not thrilled by them. 
% mbx

(112)  p414 8.2 Theorem 60: part 1 is only true if $r\neq 1$.
% 11/4/17  added ', $r\neq 1$.' at end
% mbx

(113)  p487 8.8 Exercise 15: Equality at x=2 requires Abel's Theorem; equality on (0,1) requires the Cauchy form of the remainder.  (You currently have x^{n+1} move from a denominator to a numerator.
% REVISIT   My proof is incorrect; for 0<x<1, on the displayed eq after 
% 'Thus' a x^{n+1} starts in the denominator and switches to numerator.
% The proof is wrong, and likely very hard to fix. May need to find a
% new problem.

(114)  p487 8.8 Exercise 16: The problem asks for (-1,0).  The solution proves it for (0,1) with an incorrect equality.
% 11/4/17   Fixed; ended up just uncommenting part of the solution
% and commenting out what I had. Apparently long ago I commented out 
% the wrong part
% check mbx; the comments won't show up in those files

(115)  p601 10.4 para3 L4&8: $\sin(-15^\circ)$ should be $(1-\sqrt3)/(2\sqrt2)$ (two times).  The cross product is correct.
% 11/4/17  Changed both instances to '\frac{5(-1+\sqrt3)}{\sqrt2}',
% without '-' in front of the component.
% check mbx

(116)  10.4#26: The solution should be $3\sqrt{30}$.
% 11/4/17  YEs, had an extra '/2' in answer.
% check mbx

(117)  10.4#36: The solution should specify any _unit_ vector.
% 11/4/17  Yes, and added 'Since $\vec u$ and $\vec v$ are parallel, any...'
% to solution.
% chekc mbx

(118)  p634 11.2 Definition 72: Should probably include that $\vec r'$ is continuous.
%11/4/17  added ' where $\vrp(t)$ is continuous on $I$' to def.
% check mbx

(119)  p636 11.2 Example 368 Soln Lines -5-(-9): Six $u$ should be $u'$
% 11/4/17  yes, added six cases of '''. Should look like 'u\,'(' now.
% the '\,' was already there!
% check mbx

(120)  p647 11.3 L-2: The t in $\vec C = v_0<\cos t,\sin t>$ should be $\theta$.
% 11/4/17 changed 't' to '\theta' twice
% mbx

(121)  p653 11.3#27: Both $t$s in $\vec r_2(s)=\langle 6t-6,4t-4\rangle$ should be $s$.
% yes. Fixed
% mbx

(122)  p667 11.5 Example 386 L4: The $t$ in $\vec r(s)=\langle 3t/5-1, 4t/5+2\rangle$ should be $s$.
% yes. fixed
% mbx

(123)  p676 12.1 Example 392 Soln L5: "The above equation" describes an ellipse and its interior.
% changed to 'describes an ellipse and its interior as'
% mbx

(124)  p686 12.2 Example 400 L-1: "2/0" should be "-2/0" (which still doesn't exist).
% yes
% mbx

(125)  p694 12.3 Example 405 Soln 3 L4: The first term is missing its $\sqrt{x^2+1}$: $2xy^3e^{x^2y^3}\sqrt{x^2+1}$.
% already fixed, though 
% mbx should be checked

(126)  p697 12.3 Example 407 Solution parts 1 & 2: In both, you have $f_{yx}=d/dx(f_x)$.
% changed the 'f_x' to 'f_y'
% mbx

(127)  p723 12.6 para5: I believe "the gradient is orthogonal to $\vecr$ itself" should be "the gradient is orthogonal to the level curve" (which gets repeated in the next sentence anyway).
% changed to 'meaning the gradient is orthogonal to the graph of $\vec r$.'
% mbx

(128)  p733 12.7 Example 429 L-2: ``The surface $z=-x^2+y^2$'' should be ``The surface $z=-x^2-y^2+2$''.
% yes
% mbx

(129)  p736 12.7 Example 432 L-1: Ditto.
% yes
% mbx

(130)  p743 12.8 Theorem 115: The conditions should be "$P$ is a critical point and all second order derivatives of $f$ are continuous at $P$" (and differentiability is obviously not enough).
% 11/5/17   changed the wording of the thm significantly. Even the bullets
% changed.
% check mbx.

(131)  p765 13.2 Example 451 1st displayed equ L3&4: The $-17/3$ should be $+17/3$ both times.
% yes. Fixed.
% mbx

(132)  pA19 9.5#9a: you want $\theta$, not $t$
% switched to '\theta'
% mbx

(133)  pA20 10.1#13: The coordinates are allowed to be zero.
% changed 'positive' to 'non-negative'
%mbx

(134)  pA20 10.1#19: The solution should be $x^2+z^2=\frac1{(1+y^2)^2}$.
% 11/5/17 already fixed
% check mbx

(135)  pA21 10.2#7: The solution should be $4i-4j$.
% fixed
% mbx

(136)  pA26 12.2#17a: The solution should be "Along y=mx, the limit is 0."
% changed line to '=0$ for all $m$.'
% mbx

(137)  pA27 12.6#13b: The solution should be $-2/\sqrt{5}$ ($\vec u =\bracket{-1/\sqrt{5},-2/\sqrt{5}}$) (two divisions are missing).
% 11/9/17 fixed    12_05_ex_11.tex
% mbx

(138)  pA28 12.8#11: There are critical points when $x=0$ or $y=0$.  All are absolute minima.
% 11/9/17  12_07_ex_11.tex Yes, each point on x & y axes are minima. Fixed answer.
% mbx

(139)  pA28 12.8#17: The absolute maximum is at $(\sqrt2,\sqrt2,4+4\sqrt2)$.
% 11/9/17  12_07_ex_15.tex   fixed; changed to correct answer
% mbx
(140)  pA31 13.6#13: The last lower limit of integration in the last integral has been negated from the previous line.
%11/9/17   13_06_ex_11.tex   Fixed.
% mbx

(141)  Someone noticed (and has a pet peeve in general) that there are several places where you refer to a circle $x^2+y^2=r^2$, when you really mean the disk enclosed by that circle (usually when defining a two dimensional region of integration):
(142)  p772 13.3 Example 456 L2
(143)  p793 13.5 Example 471
(144)  p794 13.5 Example 472 L7
(145)  p796-797 13.5 Exercises 8, 13, 17-19
(146)  p805 13.6 Example 477 (twice)
(147)  p818 13.6 Exercis 8


(148)  Typo:
(149)  p147 3.4 "means finding the where" should be "means finding where"
(150)  p176 4.3 Example 104 L1: "an power station" should be "a power station"
(151)  p323 6.6#40: "multiply" is missing its first L.
(152)  p599 10.4 para4 L2: "face opposite face" should be "opposite face".
(153)  p610 10.5 L-2: The vector $c$ is missing its arrow.
(154)  p611 10.5 Example 348 The solution begins "The equation of the line line"
(155)  p613-614 10.5 #29-31: Several vectors are missing arrows.  I counted 5 $\ell$ and 1 $c$.
(156)  p621 10.6 #11-14&17: The text has been using vectors over $\ell$ in this context.
(157)  p630 11.2 Definition 69 Line 4: The vector $r$ is missing its arrow.
(158)  p643 11.3 Example 372 Soln Para -1: "the have the" should be "they have the"; "they have they have" should be "they have".
(159)  p661 11.4 Example 383 solution para-2 L-3: accelerating is spelled "acclerating"
(160)  p733 12.7 para-1 L-5: "it is _a_ general geometric concept"
(161)  p740 12.8 Definition 99 #3 L-1: "_an_ -a- absolute maximum"
(162)  p748 12.8 para3 L3 (1st displayed equation): V(wl) should be V(w,l)
(163)  p762 13.2 para-2 L1: "satisfying way of computing area" should be "volume"
(164)  p790 13.4 Exercise 27-30: I_x is defined twice, I_y isn't.
(165)  p812 13.6 para2 L1: "Let $h(x,y,z)$ _be_ a continuous"

(166)  A small list of typos as we prepare for the fall semester.
(167)  p179 4.3#4&5: These problems assume that the numbers are positive.
(168)  p443 8.5 Example 253 Solution Part 3 Line 1: |sin n|/n should be {|sin n|/n^2}.
(169)  p644 11.3 Example 372 Solution Paragraph 5&6: The intervals are 1/5\textsuperscript{th} of a second apart; the ``large change in position'' is from $t=-1$ to $t=-0.8$, not $t=-1$ to $t=-0.9$.
(170)  p670 11.5 Example 389: "explictly" is misspelled.
(171)  p794 13.5 Example 472 Sol'n line 2: $f_y=$ is missing.
(172)  pA30 13.5#19 Sol'n: The last dy should be d\theta

(173)  Message: page 179, exercise 10, last sentence.    
(174)  Currently reads:
(175)  "Does it seem these dimensions where chosen with ..."
(176)  Should read:
(177)  "Does it seem these dimensions were chosen with ..."

(178)  Message: On page 371 of Calculus II Version 3.0, in example 213 after solving the integral it reads 24/39 (for the constant) but it should read 8/27 just as the last statement when calculating the result   My response: it is actually 2/3 * 4/9; look at typesetting.


(179)  10.2, Problem 21: Hey there, I think the answer is incorrectly reported. it should be 1/sqrt(58)*<3,7>, not 1/sqrt(30)<3,7>

(180)  Message: On page 481 near the top of the page in Example 270, I think the second to last step should have (k-n)/(n+1), not (k-n)/n.

(181)  Message: From section 8.1, problem 7, the answers should all be negative (they are listed as positive) (page A.6).

(182)  On page 469 Theorem 76 part 2, it would be helpful to state that the maximum is over z\in I.

(183)  On page 480 in Example 270, the Maclaurin series given is missing the x terms. (e.g. it says 1 + k + k(k-1)/2!+..., but should be 1+k x+k(k-1)/2! x^2+...

(184)  Message: On page 344 exercise 41, the solution compares the function to itself to determine that it diverges, so it seems like either the problem or solution is incorrect. I compared it to x/(x^2+1) (changing the solution).

(185)  Message: On page 172, question 13 in section 4.2. I disagree with the answer for c (and even the idea that c is possible). From d, the man must walk 34.1 feet to raise the weight all the way to the top of the pulley. Therefore in c it is NOT POSSIBLE to consider what happens when the man has walked 40 feet . If he keeps walking past 34.1 feet, the weight goes through the pulley and falls to the ground and our equation is no longer valid. The answer of 1.87 given in the back agrees with what the equation gives us, but when x = 40, I get y = 35.44 (so weight has risen beyond the height of the pulley-another way to see this isn't possible).

(186)  Theorem 59 in 8.1: only need part 1.

(187)  In Section 5.3, Key Idea 8 "The Midpoint Rule summation", the subscript on the summand is {x+1} where it should be {i+1}.

(188)  pg 280 ex 164 the 1/2 gets dropped

(189)  Dot product: formula should highlight role of unit vectors    also for orthog projection

(190)  Chapter 6.1 problem 33  answer is correct, though perhaps not intuitive. By hand: (3/2)*(ln x)^2+C.

(191)  3.3 Increasing/decreaising: Answers in back use unions; shouldn't. Just be a list of sets.

(192)  Can you check 5.2#25 answer? Also in that exercise set maybe for 21 and 25 can you put "find non-zero a and b"

(193)  Message: Page 280. The solution to example 164 should have a 1/2 in front of the ln(1+x^2)

(194)  Message: The tangent line in example 45 has the form y = 3(x-3)+4. If I were writing this, I would have avoided having 3 be the slope and 3 be the x-coordinate of the point of tangency at the same time. If the numbers are different, students will have an easier time discerning what role each number fulfills. I would suggest choosing another point, such as x=1 where f(1) = -2 and f ' (1) = -1, for the point of tangency.

(195)  Subject: Figure 2.13   Message: The caption should say something like, "Zooming in on f and its tangent line at x=3." As written, the tangent line is not mentioned in the caption, but it appears in the figure.

(196)  Subject: Theorem 10 again  Message: In the statement of theorem 10 (the Intermediate Value Theorem), it would be nice to say "there exists at least one value c" instead of "there is a value c". The same goes for the paragraphs that immediately follow Theorem 10 -- there may be multiple roots in the interval [a,b].c    Message: In the statement of the theorem, f(a) < y < f(b) can be replaced by f(a) <= y <= f(b) since you're asserting there exists a value c in [a,b] such that f(c) = y. You could also pair f(a) < y < f(b) with the existence of c in (a,b).

(197)  Figure 1.5   Message: This figure should have a hole in the graph when x=3.

(198)  Message: Example 16 solution, second to last line: the reference should be theorem 2, not theorem 3.

(199)  Message: The first sentence in the solution for example 14 refers to Theorem 3, but it should refer to Theorem 2

(200)  Message: Figure 1.2 should have a hole in the graph at the point (0,1), so that it is consistent with the graphs in Figures 1.8 and 1.9.

(201)  Message: "epsilon-delta" and "$\epsilon$-$\delta$" could be misinterpreted as epsilon minus delta. I would recommend "epsilon and delta" and "$\epsilon$ and $\delta$" as replacements.       
(202)  It would be nice the textbook would mention that the name "epsilon" is for error (both start with the letter "e") and "delta" is for difference or deviation (both start with the letter "d").       
(203)  In example 6, a brief discussion about why we choose delta <= 1.75 would be enlightening. Choosing delta to be the minimum of two numbers will seem arbitrary without any explanation.

(204)  Message: Section 1.1, page 4: The statement, "There are three ways in which a limit may fail to exist," implies that there are only three ways in which a limit may fail to exist, which is not entirely true. For instance, the two-sided limit of sqrt(x) as x approaches 0 fails to exist, but is not covered by the three cases in this list. "Three common ways in which a limit may fail to exist," would be a more accurate statement. Also,   
(205)  "3. The function may oscillate as x approaches c," is not precise enough. For instance, the function sin(x) oscillates as x approaches 0, but it has a limit. Similarly, cos(1/x) / x oscillates with increasing frequency as x approaches 0, yet the limit exists because we can apply the squeeze theorem. The limit will fail to exist for an oscillating function when the values do not approach a single limiting value.

(206)  Subject: question on page 539 example 305     answer should be ``-2'' not ``-1''.

(207)  Message: Chapter 8.5, Example 254, solution to part 1: "We want to find where 1/n^3 < 0.001: 1/n^3 <= 0.001" As you can see, the inequality changed from < to <= mid-sentence.

(208)  Subject: Theorem 64, page 422
(209)  Message: In the theorem about the infinite nature of series, the first sentence "The convergence or divergence remains unchanged by..." implies but does not explicitly mention that the sentence is about series. I would recommend something like, "Whether a series is convergent or divergent remains unchanged by..."

(210)  Message: On page 417, it says "Informally, a telescoping series is one in which the partial sums reduce to just a finite number of terms." This statement is essentially meaningless because every partial sum contains finitely many terms. I would recommend replacing "reduce to just a finite number of terms" by "can be reduced by cancellation to have fewer terms" or something similar.

(211)  Definition 29 on page 404 should say "...bounded if there exist real numbers" not "...bounded if there exists real numbers".

(212)  Also, in sections 8.2 - 8.5, there are plots of a sequence of terms a_n and a sequence of partial sums S_n on the same axes. The only thing distinguishing these two sequences is color, which means that in a grayscale print copy of the book it is not possible to distinguish the sequences from each other. I would recommend using different kinds of plot markers (filled dot and open dot, filled dot and x, or filled dot and filled triangle are commonly used) to distinguish between the two sequences. This will also help readers who are colorblind.

(213)   #33d on page 70. The answer in the back has [-5,5] where as it should be [-sqrt(5),sqrt(5)]. Just relaying the information.

(214)   Figure 8.2.a     Message: In the formula a_n = (3x^2-2x+1) / (x^2-1000) in this figure, the right side is not an expression in the variable n.

(215)  I just noticed, mid lecture, that Theorem 1.9  is not actually true unless one of two conditions are met.  For instance, it fails when g(x) is not continuous at L and f(x) = L is the constant function.        We can fix it by adding one of two constraints:
(216)  1) $f(x) \neq L$ when x is near c, or 
(217)  2) g(L)=K 
(218)  See email for my response/ideas. May be solved with PCC group already

(219)  Subject: error on page 620      I believe there is an error in the solution to example 355. The z-component of the normal vector should be positive 6, not negative six.

(220)  The answer for Section 1.4, 6f should say:
(221)  As f is not defined for x > 2, this limit is not defined.

(222)  Page 37 has a grammatical error.  The sentence should say:
(223)  We have seen, though, that this is not necessarily a good indicator
(224)  of what f(x) actually is.

(225)  Error on page 200
(226)  Message: The sentence 
(227)  "Finally, we find the total signed area under the velocity function from t=0 to t=2 to find the s(2), the height at t=2, which is a displacement, the distance from the current position to the starting position"       is an awkward, unclear, run-on sentence. Something like        
(228)  "Finally, to find the height of the object at time t=2 we calculate the total signed area under the velocity function from t=0 to t=2. This signed area is equal to s(2), the displacement (i.e., signed distance) from the starting position at t=0 to the position at time t=2."
(229)  would be an improvement. Note that in the proposed revision, it is emphasized that displacement is a signed distance.

(230)  Message: Page 319
(231)  Section 6.6 Hyperbolic Functions
(232)  Key Idea 19
(233)  Item #3     The factor in front of ln should be 1/(2a)

(234)  Subject: error on p. 624
(235)  Message: I and my students are liking using APEX Calculus 3.0 for Multivariable Calculus at Alverno College.
(236)  There appears to be an error in Figure 11.3 on p. 624. The figure claims to be showing the graph of a vector valued function defined in Example 357, and "its derivative at one point." In fact, the derivative has not yet been defined, and what is being shown is the vector at a point.

(237)  One typo in Version 3 - page 416, Example 239, 5, the index of the series should probably start at 11 instead of 10.

(238)  Look at answers about intervals of incr/decr and concave up/down. Make answers in terms of intervals, not using unions of sets.

(239)  Thm 63, 8.2, pg 422 not really a test for convergence. Rewrite. Look maybe at mbx conversation to see what was discussed there if I haven't already addressed this above

(240) 12.8 Look at solution to #17. Perhaps more critical points? Depends if the curve is y=\sqrt{4-x^2} or parametrized, perhaps. The given max isn't the correct max.
% 11/9/17 Added on this date; fixed as part of (139). I think this is all that was
% needed; John implied I missed some critical points, but they seem to be listed.
%

(241) Look at 6.8, #14 solution. Should be Diverges, not \pi/2. Function is x/(x^2+4) which won't converge on (-\infty,\infty), so either change answer or change problem.

(242) change p-series and geometric series def's to also make ``p-series'' and ``geoemtirc series'' tests