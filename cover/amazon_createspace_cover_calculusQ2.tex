\documentclass{article}

%%% Width is 0.002252 x page count + 0.125 bleed + 0.125 bleed + 8.5*2
\usepackage[paperwidth=17.88in,paperheight=11.25in]{geometry}
%\usepackage{fontspec}
%\usepackage{pgfplots}
\usepackage{tikz}
\usetikzlibrary{calc}
\pagestyle{empty}

\sffamily
	%%\usepackage{fontspec}
	\usepackage{mathspec}
	\setallmainfonts[Mapping=tex-text]{Calibri}
	\setmainfont[Mapping=tex-text]{Calibri}
	\setsansfont[Mapping=tex-text]{Calibri}
	\setmathsfont(Greek){[cmmi10]}



\newcommand{\apex}{A\kern -1pt \lower -2pt\hbox{P}\kern -3pt \lower .65ex\hbox{E}\kern 0pt X}

\begin{document}


\begin{tikzpicture}[remember picture,overlay]
\coordinate (top left) at (current page.north west);
\coordinate (top right) at (current page.north east);
\coordinate (bot left) at (current page.south west);
\coordinate (bot right) at (current page.south east);
\coordinate [xshift=.125in,yshift=-.125in] (ptop left) at (current page.north west) ;
\coordinate [xshift=.125in,yshift=.125in] (pbot left) at (current page.south west);
\coordinate [xshift=-.125in,yshift=-.125in] (ptop right) at (current page.north east);
\coordinate [xshift=-.125in,yshift=.125in] (pbot right) at (current page.south east);
\coordinate [xshift=8.625in] (lspine top) at (current page.north west);
\coordinate [xshift=-8.625in] (rspine top) at (current page.north east);
\coordinate [xshift=8.625in] (lspine bot) at (current page.south west);
\coordinate [xshift=-8.625in] (rspine bot) at (current page.south east);
\coordinate (frontpagecenter) at ($.5*(lspine top)+.5*(pbot right)$);

\end{tikzpicture}

\begin{tikzpicture}[remember picture,overlay]
%\shade [bottom color=red!50!black,top color=red!70!black!60]
%(top left) rectangle (bot right);
\draw [fill=yellow] (top left) rectangle (bot right);

\draw [fill=yellow!40!white,draw=yellow!40!white] (rspine top) -- (current page.north east) -- (frontpagecenter) -- cycle;

\draw [fill=yellow!60!white,draw=yellow!60!white] (current page.north east) -- (frontpagecenter) -- (current page.south east) --cycle;

\draw [fill=yellow!80!white,draw=yellow!80!white] (frontpagecenter) -- (current page.south east) -- (rspine bot) -- cycle;


%%
%% Draws lines for reference. Turn off for final product.
%%%
%\draw[green] (ptop left) -- (ptop right) -- (pbot right) -- (pbot left) -- (ptop left);
%\draw [green] (lspine top) -- (lspine bot) (rspine top) -- (rspine bot);
%\draw[green] (ptop right) -- (pbot left);

\begin{scope}%[rotate=90]


\node[xshift=-2in,yshift=0in,scale=5,red,transform shape,rotate=90] at ($.5*(ptop right)+.5*(pbot right)$) {\fontspec[Scale=3]{Calibri}\scshape C\fontspec[Scale=2.5]{Calibri}alculus \fontspec[Scale=3]{Calibri}Q2};



\node[xshift=-.75in,yshift=-.10in,scale=4,black,transform shape,rotate=90] at ($.5*(ptop right)+.5*(pbot right)$) {\fontspec[Scale=1]{Calibri} \parbox{170pt}{\hfill\ Version 4.0}};

\node [xshift=-2.55in,yshift=-2.9in,black,scale=3,transform shape,rotate=90] at ($.5*(ptop right)+.5*(pbot right)$) {\fontspec[Scale=1]{Calibri} \apex};

\end{scope}

\draw [black] ($(current page.center)!.75!(current page.south)$) node [xshift=3in] {\fontspec[Scale=2]{Calibri}\parbox{4in}{{ Volume Two of }\\[8pt] \apex\ Calculus for Quarters\\[4pt] Chapters 5 - 7 + \\[4pt] Differential Equations}};

\draw (current page.center) node [yshift=0in,rotate=-90,black] {\fontspec[Scale=2]{Calibri} \scshape \apex \fontspec[Scale=4]{Calibri} \hskip 2in \hskip 2.5in\hskip 2in \fontspec[Scale=2]{Calibri} Version 4.0};

\draw (current page.center) node [yshift=0in,rotate=-90,red] { \scshape  \fontspec[Scale=4]{Calibri}  Calculus Q2 };

\end{tikzpicture}



\end{document}

%The is the first text of a four-part series of Calculus texts intended for schools on the quarter-system. This text covers limits and derivatives. The follow-up to this text is Calculus Q2, which covers integration and its applications along with an introduction to differential equations. Calculus Q3 covers parametric equations, polar coordinates, and vector valued functions and Calculus Q4 covers multivariable functions and vector analysis. Free .pdf versions of these texts can be obtained at apexcalculus.com.

