\documentclass[10pt]{article}
\usepackage[textwidth=7.5in,textheight=9.5in]{geometry}
\pagestyle{empty}
\begin{document}


APEX Calculus

\begin{enumerate}
	\item	Limits
		\begin{enumerate}
		\item		Graphical and Numerical Explorations %Start with graphical example, approximating $\displaystyle \lim_{x\rightarrow 1} \frac{\sin x}{x}$. All seems well. Now $\displaystyle \lim_{x\rightarrow 0} \frac{\sin x}{x}$. All seems well except that not defined. Define \textit{indeterminate form}. Now look numerically.
		
%		Now try a rational function, both numerically and graphically (all approximations).
%		
%		Consider where it seems limits do not exist.
%		
%		Exercises:
%		Find limits based on creating graphs, creating tables. Include examples where things don't work.
%		
%		{\bf Hazard}: we haven't defined limit yet. Did include a ``pseudo-definition'' as a basis for the approximations.
		
		\item		$\delta$-$\epsilon$ Proofs	%Give definition of limit, including pictorial meaning. Give examples of linear, quadratic, simple rational ($1/(x+1)$, perhaps). 
		
%		Exercises: Proofs based on above practice.%
		
		\item		Finding Limits Analytically	Subtitle: Using known limits to find unknown limits. Contains several ``big'' theorems.
		
%		Theorem 1: sum/diff, mult/div, powers, constant multiple
%		
%		Corollary: polynomial, rational
%		
%		Thm: Other important functions (trig, exponential, etc.)
%		
%		Thm: Squeeze Thm. Use to prove $\sin x/x$ limit at 0. 
%		
%		Thm: Limits of functions that agree at all but one point. Practice eliminating factor.
%		
%		Would like to add Difference Quotients, but think chapter is long enough already
%		
%		Exercises: lots of limit practice; ``standard stuff''
		
		\item		One Sided Limits and Continuity
		
%		Define one sided limits. Give examples.
%		
%		Define continuous at a point, continuous on an interval.
%		
%		Thm: Rules of continuity at a point
%		
%		Exercises: Find one sided limits. Is continuous?
		
		\item		Limits Involving Infinity
		
%		Do both limits to infinity, limits resulting in ``infinity''. 
%		
%		Define more clearly indeterminate form.
%		
%		Thm: Rules about limits involving infinity
%		
%		Thm: Something about the limits as $x\rightarrow\infty$ and rational functions?
%		
%		Exercises: Lots of standard stuff. Graph slant asymptote. 

		\end{enumerate}

\item		Derivatives
		\begin{enumerate}
		\item		Introduction Speed during free fall at a particular time; relate this to tangent line problems.
%		
%		Define derivative at a point.
%		
%		Elaborate on this to define the derivative function
%		
%		?? Numerical derivative approximations?
		
		\item		Basic Differentiation Rules	Power ($n>0$), trig, log, exponential.
		
		Average rate of change $\rightarrow$ inst. rate of change
		
		?? Numerical derivative approximations?  (I think above.)
		
			Higher order derivatives
		
		\item		Product and Quotient Rules  Define rest of the trig functions
		
		\item		Chain Rule	$f(x) = a^x$
		
		\item		Implicit Differentiation
		
		\item		Derivatives of Inverse Functions  Especially inverse trig functions
		\end{enumerate}
		
\item	Graphical Behavior of Functions
		\begin{enumerate}
		\item		Extreme Value Thm (including definition of critical numbers, etc.)
		\item	Mean Value Theorem (usually includes Rolle's Thm)
		\item	1st Deriv. Test (intervals of incr/decr)
		\item	2nd Deriv Test (concavity explained, etc.)
		\item	Curve Sketching (recall we've already talked about limits at infinity, asymptotes, etc.)
		\end{enumerate}

\item	Application of Derivatives
		\begin{enumerate}
		\item	Newton's Method (?) This one is good, but not critical to me at the moment
		\item	Related Rates
		\item	Optimization (hopefully with not too trivial examples)
		\item	Differentials (with an eye towards both approximation and a lead-in to integration)
		\end{enumerate}		
		
\item		Fundamental Theorem of Calculus: Integration
		\begin{enumerate}
		\item		Antiderivatives
		\item		Finite/Riemann Sums -- Area Approximations (summation notation, etc.) Left/Right hand sums
		\item		Riemann Sums and Definite Integrals
		\item		Fundamental Theorem of Calculus
		\item		Numerical Integration
		\end{enumerate}

\item		Integration Techniques
		\begin{enumerate}
		\item		Integration by Substitution
			\begin{enumerate}
				\item		Integrals Involving Logarithms
				\item		Integrals Involving Inverse Trigonometric Functions
			\end{enumerate}
		\item		Integration by Parts
		\item		Integrals Involving Trigonometric Functions
		\item		Integrals Involving Partial Fraction Decomposition
		\item		Integrals of Hyperbolic Functions
		\item		L'Hopital's Rule
		\item		Improper Integration
		\end{enumerate}

\item		Applications of the Definite Integral
		\begin{enumerate}
		\item		Area Between Curves (place earlier?)
		\item		Volume: Disk Method
		\item		Volume: Shell Method
		\item		Arc Length, Surface Area
		\item		Work
		\item		Fluid Force/Pressure
		\item		Center of Mass
		\end{enumerate}

\item		Sequences and Series
		\begin{enumerate}
		\item		Sequences
		\item		Series
		\item		Taylor Polynomials
		\end{enumerate}

\end{enumerate}

%
%
%\clearpage
%\textbf{\Large Contents of Calculus,} based on Larson, et al, 4th Edition
%
%\begin{enumerate}
%
%\item		Chapter 2, Limits: 2.1 - 2.5
%\vskip 2in
%\item		Chapter 3, Differentiation: 3.1 - 3.7
%\vskip 2in
%\item		Chapter 4, Applications of Differentiation: 4.1 - 4.8
%\vskip 2in
%\item		Chapter 5, Integration: 5.1, 5.5, 5.7, 5.8
%
%\vskip 2in
%
%\clearpage
%
%\item		Chapter 5, Integration: 5.2 - 5.6, 5.9
%\vskip 2in
%\item		Chapter 6, Applications of Integration: 6.1 - 6.5, 6.7
%\vskip 2in
%\item		Chapter 8, Integration Techniques, Etc.: 8.1 - 8.3, 8.5, 8.7
%\vskip 2in
%\item		Chapter 9, Infinite Series, 9.1 - 9.3, 9.7
%
%\end{enumerate}


\end{document}