\section{Infinite Series}\label{sec:series}

Given the sequence $\{a_n\} = \{1/2^n\} = 1/2,\ 1/4,\ 1/8,\ \ldots$, consider the following sums:

$$\begin{array}{ccccc}
a_1				&=& 1/2					 &=& 1/2\\
a_1+a_2		&=& 1/2+1/4			 &=& 3/4\\
a_1+a_2+a_3 &=& 1/2+1/4+1/8  &=& 7/8\\
a_1+a_2+a_3+a_4 &=& 1/2+1/4+1/8+1/16 & =& 15/16
\end{array}$$
In general, we can show that $$a_1+a_2+a_3+\cdots +a_n = \frac{2^n-1}{2^n} = 1-\frac{1}{2^n}.$$
Let $S_n$ be the sum of the first $n$ terms of the sequence $\{1/2^n\}$. From the above, we see that $S_1=1/2$, $S_2 = 3/4$, etc. Our formula at the end shows that $S_n = 1-1/2^n$. 

Now consider the following limit: $\ds \lim_{n\to\infty}S_n = \lim_{n\to\infty}\big(1-1/2^n\big) = 1$. This limit can be interpreted as saying something amazing: \emph{the sum of \emph{all} the terms of the sequence $\{1/2^n\}$ is 1.} 

\enlargethispage{\baselineskip}

This example illustrates some interesting concepts that we explore in this section. We begin this exploration with some definitions.

\setboxwidth{50pt}
\definition{def:series}{Infinite Series, $n^\text{th}$ Partial Sums, Convergence, Divergence}
{Let $\{a_n\}$ be a sequence.
\begin{enumerate}
\item		The sum $\ds \sum_{n=1}^\infty a_n$ is an \textbf{infinite series} (or, simply \textbf{series}).
\item		Let $\ds S_n = \sum_{i=1}^n a_i$\,; the sequence $\{S_n\}$ is the sequence of \textbf{$n^\text{th}$ partial sums} of $\{a_n\}$.
\item		If the sequence $\{S_n\}$ converges to $L$, we say the series $\ds \sum_{n=1}^\infty a_n$ \textbf{converges} to $L$, and we write $\ds \sum_{n=1}^\infty a_n = L$.
\item		If the sequence $\{S_n\}$ diverges, the series $\ds \sum_{n=1}^\infty a_n$ \textbf{diverges}.
\index{series!definition}\index{series!partial sums}\index{series!convergent}\index{series!divergent}\index{convergence!of series}\index{divergence!of series}
\end{enumerate}
}
\restoreboxwidth

Using our new terminology, we can state that the series $\ds \sum_{n=1}^\infty 1/2^n$ converges, and $\ds \sum_{n=1}^\infty 1/2^n = 1.$\\

We will explore a variety of series in this section. We start with two series that diverge, showing how we might discern divergence.\\

\example{ex_series1}{Showing series diverge}{
\begin{enumerate}
\item		Let $\{a_n\} = \{n^2\}$. Show $\ds \sum_{n=1}^\infty a_n$ diverges.
\item		Let $\{b_n\} = \{(-1)^{n+1}\}$. Show $\ds \sum_{n=1}^\infty b_n$ diverges.
\end{enumerate}
}
{\begin{enumerate}
\item	Consider $S_n$, the $n^\text{th}$ partial sum.
\begin{align*} S_n &= a_1+a_2+a_3+\cdots+a_n \\		
						&= 1^2+2^2+3^2\cdots + n^2.
\intertext{By Theorem \ref{thm:summation}, this is}
						&= \frac{n(n+1)(2n+1)}{6}.
\end{align*}
Since $\ds \lim_{n\to\infty}S_n = \infty$, we conclude that the series $\ds \sum_{n=1}^\infty n^2$ diverges. It is instructive to write $\ds \sum_{n=1}^\infty n^2=\infty$ for this tells us \emph{how} the series diverges: it grows without bound.

A scatter plot of the sequences $\{a_n\}$ and $\{S_n\}$ is given in Figure \ref{fig:series1}(a). The terms of $\{a_n\}$ are growing, so the terms of the partial sums $\{S_n\}$ are growing even faster, illustrating that the series diverges.

%\mfigure{.5}{Scatter plots relating to the series of Example \ref{ex_series1} part 1.}{fig:series1a}{figures/figseries1a}

\item		The sequence $\{b_n\}$ starts with 1, $-1$, 1, $-1$, $\ldots$. Consider some of the partial sums $S_n$ of $\{b_n\}$:
\begin{align*}
S_1 &= 1\\
S_2 &= 0\\
S_3 &= 1\\
S_4 &= 0
\end{align*}
This pattern repeats; we find that $S_n = \left\{\begin{array}{cc} 1  & n\ \text{ is odd}\\
																																		0  & n\  \text{ is even}
																								\end{array}\right..$
As $\{S_n\}$ oscillates, repeating 1, 0, 1, 0, $\ldots$, we conclude that $\ds\lim_{n\to\infty}S_n$ does not exist, hence $\ds\sum_{n=1}^\infty (-1)^{n+1}$ diverges.		

A scatter plot of the sequence $\{b_n\}$ and the partial sums $\{S_n\}$ is given in Figure \ref{fig:series1}(b). When $n$ is odd, $b_n = S_n$ so the marks for $b_n$ are drawn oversized to show they coincide.	
%\mfigure{.8}{Scatter plots relating to the series of Example \ref{ex_series1} part 2.}{fig:series1b}{figures/figseries1b}
\mtable{.6}{Scatter plots relating to Example \ref{ex_series1}.}{fig:series1}{%
\begin{tabular}{c}
\myincludegraphics{figures/figseries1a}\\[10pt]
(a)\\[15pt]
\myincludegraphics{figures/figseries1b}\\[10pt]
(b)
\end{tabular}
}																					
\end{enumerate}
\vskip-1.5\baselineskip
}\\

While it is important to recognize when a series diverges, we are generally more interested in the series that  converge. In this section we will demonstrate a few general techniques for determining convergence; later sections will delve deeper into this topic.

\vskip\baselineskip
\noindent\textbf{\large Geometric Series}\\

One important type of series is a \emph{geometric series}.

\definition{def:geom_series}{Geometric Series}
{A \textbf{geometric series} is a series of the form 
$$\sum_{n=0}^\infty r^n = 1+r+r^2+r^3+\cdots+r^n+\cdots$$
Note that the index starts at $n=0$, not $n=1$.%
\index{series!geometric}\index{geometric series}
}

We started this section with a geometric series, although we dropped the first term of $1$. One reason geometric series are important is that they have nice convergence properties.

\theorem{thm:geom_series}{Geometric Series Test}
{Consider the geometric series $\ds \sum_{n=0}^\infty r^n$.
\begin{enumerate}
\item		The $n^\text{th}$ partial sum is: $\ds S_n = \frac{1-r\,^{n+1}}{1-r}$, $r\neq 1$.
\item		The series converges if, and only if, $|r| < 1$. When $|r|<1$, 
\index{series!geometric}\index{geometric series}\index{convergence!of geometric series}\index{divergence!of geometric series}
$$\sum_{n=0}^\infty r^n = \frac{1}{1-r}.$$
\end{enumerate}
}

According to Theorem \ref{thm:geom_series}, the series 
$$\ds\sum_{n=0}^\infty \frac{1}{2^n} =\sum_{n=0}^\infty \left(\frac 12\right)^2= 1+\frac12+\frac14+\cdots$$ converges as $r=1/2$, and $\ds \sum_{n=0}^\infty \frac{1}{2^n} = \frac{1}{1-1/2} = 2.$ This concurs with our introductory example; while there we got a sum of 1, we skipped the first term of 1.\\
%\clearpage

\enlargethispage{2\baselineskip}
\example{ex_series2}{Exploring geometric series}{
Check the convergence of the following series. If the series converges, find its sum.\\

\noindent 1. $\ds \sum_{n=2}^\infty \left(\frac34\right)^n$\qquad 2. $\ds \sum_{n=0}^\infty \left(\frac{-1}{2}\right)^n$ \qquad 3. $\ds \sum_{n=0}^\infty 3^n$ 
}
{\begin{enumerate}
\item		Since $r=3/4<1$, this series converges. By Theorem \ref{thm:geom_series}, we have that
$$\sum_{n=0}^\infty \left(\frac34\right)^n = \frac{1}{1-3/4} = 4.$$ However, note the subscript of the summation in the given series: we are to start with $n=2$. Therefore we subtract off the first two terms, giving:
$$\sum_{n=2}^\infty \left(\frac34\right)^n = 4 - 1 - \frac34 = \frac94.$$
This is illustrated in Figure \ref{fig:series2a}.

\mfigure{.35}{Scatter plots relating to the series in Example \ref{ex_series2}.}{fig:series2a}{figures/figseries2a}

%\mfigure{.8}{Scatter plots relating to the series of Example \ref{ex_series2} part 1.}{fig:series2a}{figures/figseries2a}

\item	Since $|r| = 1/2 < 1$, this series converges, and by Theorem \ref{thm:geom_series},
$$\sum_{n=0}^\infty \left(\frac{-1}{2}\right)^n = \frac{1}{1-(-1/2)} = \frac23.$$
The partial sums of this series are plotted in Figure \ref{fig:series2}(a). Note how the partial sums are not purely increasing as some of the terms of the sequence $\{(-1/2)^n\}$ are negative.

%\mfigure{.55}{Scatter plots relating to the series of Example \ref{ex_series2} part 2.}{fig:series2b}{figures/figseries2b}

\item		Since $r>1$, the series diverges. (This makes ``common sense''; we expect the sum $$1+3+9+27 + 81+243+\cdots$$ to diverge.) This is illustrated in Figure \ref{fig:series2}(b).

%\mfigure{.3}{Scatter plots relating to the series of Example \ref{ex_series2} part 3.}{fig:series2c}{figures/figseries2c}
\mtable{.67}{Scatter plots relating to the series in Example \ref{ex_series2}.}{fig:series2}{%
\begin{tabular}{c}
%\myincludegraphics{figures/figseries2a}\\[10pt]
%(a)\\[15pt]
\myincludegraphics{figures/figseries2b}\\[10pt]
(a)\\[15pt]
\myincludegraphics{figures/figseries2c}\\[10pt]
(b)
\end{tabular}
}
\end{enumerate}
\vskip-\baselineskip
}\\

\noindent\textbf{\large $p$--Series}\\

Another important type of series is the \emph{p-series}.

\definition{def:pseries}{$p$--Series, General $p$--Series}
{\begin{enumerate}
\item	A \textbf{$p$--series} is a series of the form $$\sum_{n=1}^\infty \frac{1}{n^p}, \qquad \text{where $p>0$.}$$

\item	A \textbf{general $p$--series} is a series of the form 
\index{series!p@$p$-series}\index{p@$p$-series}
$$\sum_{n=1}^\infty \frac{1}{(an+b)^p}, \qquad \text{where $p>0$ and $a$, $b$ are real numbers.}$$
\end{enumerate}
}

Like geometric series, one of the nice things about p--series is that they have easy to determine convergence properties.

\theorem{thm:pseries}{$p$--Series Test}
{A general $p$--series $\ds\sum_{n=1}^\infty \frac{1}{(an+b)^p}$ will converge if, and only if, $p>1$.\index{series!p@$p$-series}\index{p@$p$-series}
\index{convergence!of p@of $p$-series}\index{divergence!of p@of $p$-series}
}
\mnote{.3}{\textbf{Note:} Theorem \ref{thm:pseries} assumes that $an+b\neq 0$ for all $n$. If $an+b=0$ for some $n$, then of course the series does not converge regardless of $p$ as not all of the terms of the sequence are defined.}

\example{ex_series6}{Determining convergence of series}{
Determine the convergence of the following series.\\

\noindent\begin{minipage}[t]{.33\linewidth}
\begin{enumerate}
\item		$\ds\sum_{n=1}^\infty \frac{1}{n}$
\item		$\ds\sum_{n=1}^\infty \frac{1}{n^2}$
\end{enumerate}
\end{minipage}
\begin{minipage}[t]{.33\linewidth}
\begin{enumerate}\addtocounter{enumi}{2}
\item		$\ds\sum_{n=1}^\infty \frac{1}{\sqrt{n}}$
\item		$\ds\sum_{n=1}^\infty \frac{(-1)^n}{n}$
\end{enumerate}
\end{minipage}\begin{minipage}[t]{.33\linewidth}
\begin{enumerate}\addtocounter{enumi}{4}
\item		$\ds\sum_{n=11}^\infty \frac{1}{(\frac12n-5)^3}$
\item		$\ds\sum_{n=1}^\infty \frac{1}{2^n}$
\end{enumerate}
\end{minipage}
}
{\begin{enumerate}
\item		This is a $p$--series with $p=1$. By Theorem \ref{thm:pseries}, this series diverges. 

This series is a famous series, called the \emph{Harmonic Series}, so named because of its relationship to \emph{harmonics} in the study of music and sound. 

\item		This is a $p$--series with $p=2$. By Theorem \ref{thm:pseries}, it converges. Note that the theorem does not give a formula by which we can determine \emph{what} the series converges to; we just know it converges. A famous, unexpected result is that this series converges to $\ds{\pi^2}/{6}$.

\item		This is a $p$--series with $p=1/2$; the theorem states that it diverges.

\item		This is not a $p$--series; the definition does not allow for alternating signs. Therefore we cannot apply Theorem \ref{thm:pseries}. (Another famous result states that this series, the \emph{Alternating Harmonic Series}, converges to $\ln 2$.)

\item		This is a general $p$--series with $p=3$, therefore it converges.

\item		This is not a $p$--series, but a geometric series with $r=1/2$. It converges.
\end{enumerate}
\vskip-1.5\baselineskip
}\\

Later sections will provide tests by which we can determine whether or not a given series converges. This, in general, is much easier than determining \emph{what} a given series converges to. There are many cases, though, where the sum can be determined. \\


\example{ex_series3}{Telescoping series}{
Evaluate the sum $\ds \sum_{n=1}^\infty \left(\frac1n-\frac1{n+1}\right)$.
\index{series!telescoping}\index{telescoping series}}
{It will help to write down some of the first few partial sums of this series.
\begin{align*}
S_1 &=	\frac11-\frac12 & & = 1-\frac12\\
S_2 &=	\left(\frac11-\frac12\right) + \left(\frac12-\frac13\right) & & = 1-\frac13\\
S_3 &=	\left(\frac11-\frac12\right) + \left(\frac12-\frac13\right)+\left(\frac13-\frac14\right) & &= 1-\frac14\\
S_4 &=	\left(\frac11-\frac12\right) + \left(\frac12-\frac13\right)+\left(\frac13-\frac14\right) +\left(\frac14-\frac15\right)& &= 1-\frac15
\end{align*}
Note how most of the terms in each partial sum are canceled out! In general, we see that $\ds S_n = 1-\frac{1}{n+1}$. The sequence $\{S_n\}$ converges,  as $\ds \lim_{n\to\infty}S_n = \lim_{n\to\infty}\left(1-\frac1{n+1}\right) = 1$, and so we conclude that $\ds \sum_{n=1}^\infty \left(\frac1n-\frac1{n+1}\right) = 1$. Partial sums of the series are plotted in Figure \ref{fig:series3}.
\mfigure{.75}{Scatter plots relating to the series of Example \ref{ex_series3}.}{fig:series3}{figures/figseries3}
}\\

The series in Example \ref{ex_series3} is an example of a \sword{telescoping series}. Informally, a telescoping series is one in which most terms cancel with preceding or following terms, reducing the number of terms in each partial sum. The partial sum $S_n$ did not contain $n$ terms, but rather just two: 1 and $1/(n+1)$.\index{series!telescoping}\index{telescoping series}

When possible, seek a way to write an explicit formula for the $n^\text{th}$ partial sum $S_n$. This makes evaluating the limit $\ds\lim_{n\to\infty} S_n$ much more approachable. We do so in the next example.\\

%\noindent\textbf{Note on notation:} Most of the series we encounter will start with $n=1$. For ease of notation, we will often write $\sum a_n$ instead of writing $\ds\sum_{n=1}^\infty a_n$.\\



\example{ex_series4}{Evaluating series}{
Evaluate each of the following infinite series.\\

\noindent 1. $\ds \sum_{n=1}^\infty \frac{2}{n^2+2n}$ \qquad 2. $\ds \sum_{n=1}^\infty \ln\left(\frac{n+1}{n}\right)$
}
{\begin{enumerate}
\item		We can decompose the fraction $2/(n^2+2n)$ as $$\frac2{n^2+2n} = \frac1n-\frac1{n+2}.$$ (See Section \ref{sec:partial_fraction}, Partial Fraction Decomposition, to recall how  this is done, if necessary.)

Expressing the terms of $\{S_n\}$ is now more instructive:
\footnotesize
\begin{align*}
S_1 &= 1-\frac13 &&= 1-\frac13\\
S_2 &= \left(1-\frac13\right) + \left(\frac12-\frac14\right) &&= 1+\frac12-\frac13-\frac14\\
S_3 &= \left(1-\frac13\right) + \left(\frac12-\frac14\right)+\left(\frac13-\frac15\right) &&= 1+\frac12-\frac14-\frac15\\
S_4 &= \left(1-\frac13\right) + \left(\frac12-\frac14\right)+\left(\frac13-\frac15\right)+\left(\frac14-\frac16\right) &&= 1+\frac12-\frac15-\frac16\\
S_5 &= \left(1-\frac13\right) + \left(\frac12-\frac14\right)+\left(\frac13-\frac15\right)+\left(\frac14-\frac16\right)+\left(\frac15-\frac17\right) &&= 1+\frac12-\frac16-\frac17\\
\end{align*}
\normalsize

\drawexampleline
We again have a telescoping series. In each partial sum, most of the terms cancel and we obtain the formula $\ds S_n = 1+\frac12-\frac1{n+1}-\frac1{n+2}.$ Taking limits allows us to determine the convergence of the series:
$$\lim_{n\to\infty}S_n = \lim_{n\to\infty} \left(1+\frac12-\frac1{n+1}-\frac1{n+2}\right) = \frac32,\quad \text{so } \sum_{n=1}^\infty \frac1{n^2+2n} = \frac32.$$
This is illustrated in Figure \ref{fig:series4}(a).
%\mfigure{.3}{Scatter plots relating to the series of Example \ref{ex_series4} part 1.}{fig:series4a}{figures/figseries4a}
\mtable{.5}{Scatter plots relating to the series in Example \ref{ex_series4}.}{fig:series4}{%
\begin{tabular}{c}
\myincludegraphics{figures/figseries4a}\\[10pt]
(a)\\[15pt]
\myincludegraphics{figures/figseries4b}\\[10pt]
(b)
\end{tabular}
}
%\drawexampleline

\item		We begin by writing the first few partial sums of the series:

\begin{align*}
S_1 &= \ln\left(2\right) \\
S_2 &= \ln\left(2\right)+\ln\left(\frac32\right) \\
S_3 &= \ln\left(2\right)+\ln\left(\frac32\right)+\ln\left(\frac43\right) \\
S_4 &= \ln\left(2\right)+\ln\left(\frac32\right)+\ln\left(\frac43\right)+\ln\left(\frac54\right) 
\end{align*}
At first, this does not seem helpful, but recall the logarithmic identity: $\ln x+\ln y = \ln (xy).$ Applying this to $S_4$ gives:
$$S_4 = \ln\left(2\right)+\ln\left(\frac32\right)+\ln\left(\frac43\right)+\ln\left(\frac54\right) = \ln\left(\frac21\cdot\frac32\cdot\frac43\cdot\frac54\right) = \ln\left(5\right).$$

We can conclude that $\{S_n\} = \big\{\ln (n+1)\big\}$. This sequence  does not converge, as $\ds \lim_{n\to\infty}S_n=\infty$. Therefore  $\ds\sum_{n=1}^\infty  \ln\left(\frac{n+1}{n}\right)=\infty$; the series diverges. Note in Figure \ref{fig:series4}(b) how the sequence of partial sums grows slowly; after 100 terms, it is not yet over 5. Graphically we may be fooled into thinking the series converges, but our analysis above shows that it does not.
%\mfigure{.35}{Scatter plots relating to the series of Example \ref{ex_series4} part 2.}{fig:series4b}{figures/figseries4b}
\end{enumerate}
\vskip-1.5\baselineskip
}\\

%\enlargethispage{3\baselineskip}
We are learning about a new mathematical object, the series. As done before, we apply ``old'' mathematics to this new topic.

\theorem{thm:series_prop}{Properties of Infinite Series}
{Let \quad$\ds \sum_{n=1}^\infty a_n = L$,\quad  $\ds\sum_{n=1}^\infty b_n = K$, and let $c$ be a constant.
\begin{enumerate}
\item  Constant Multiple Rule: $\ds\sum_{n=1}^\infty c\cdot a_n = c\cdot\sum_{n=1}^\infty a_n = c\cdot L.$\index{Constant Multiple Rule!of series}
\item		Sum/Difference Rule: $\ds\sum_{n=1}^\infty \big(a_n\pm b_n\big) = \sum_{n=1}^\infty a_n \pm \sum_{n=1}^\infty b_n = L \pm K.$
\index{series!properties}\index{Sum/Difference Rule!of series}
\end{enumerate} 
}

\enlargethispage{2\baselineskip}
Before using this theorem, we provide a few ``famous'' series.

\setboxwidth{20pt}
%\hskip-35pt
%\begin{minipage}{\specialboxlength}
\keyidea{idea:famous_series}{Important Series}
{\begin{enumerate}
\item	\parbox{90pt}{$\ds\sum_{n=0}^\infty \frac1{n!} = e$. } (Note that the index starts with $n=0$.)
\item	$\ds\sum_{n=1}^\infty \frac1{n^2} = \frac{\pi^2}{6}$.
\item	$\ds\sum_{n=1}^\infty \frac{(-1)^{n+1}}{n^2} = \frac{\pi^2}{12}$.
\item	$\ds\sum_{n=0}^\infty \frac{(-1)^{n}}{2n+1} = \frac{\pi}{4}$.
\item	\parbox{90pt}{$\ds\sum_{n=1}^\infty \frac{1}{n} $ \quad diverges.} (This is called the \emph{Harmonic Series}.)\index{Harmonic Series}
\item	\parbox{90pt}{$\ds\sum_{n=1}^\infty \frac{(-1)^{n+1}}{n} = \ln 2$.} (This is called the \emph{Alternating Harmonic Series}.)\index{Alternating Harmonic Series}
\end{enumerate}
}
%\end{minipage}
\restoreboxwidth

\example{ex_series5}{Evaluating series}{
Evaluate the given series.\\

\noindent 1. $\ds\sum_{n=1}^\infty \frac{(-1)^{n+1}\big(n^2-n\big)}{n^3}$\qquad 2. $\ds\sum_{n=1}^\infty \frac{1000}{n!}$\qquad 3. $\ds \frac1{16}+\frac1{25}+\frac1{36}+\frac1{49}+\cdots$
}
{\begin{enumerate}
\item	We start by using algebra to break the series apart:
\begin{align*}
\sum_{n=1}^\infty \frac{(-1)^{n+1}\big(n^2-n\big)}{n^3} &= \sum_{n=1}^\infty\left(\frac{(-1)^{n+1}n^2}{n^3}-\frac{(-1)^{n+1}n}{n^3}\right) \\
						&= \sum_{n=1}^\infty\frac{(-1)^{n+1}}{n}-\sum_{n=1}^\infty\frac{(-1)^{n+1}}{n^2} \\
						&= \ln(2) - \frac{\pi^2}{12}	\approx	-0.1293.
\end{align*}

This is illustrated in Figure \ref{fig:series5}(a).
%\mfigure{.75}{Scatter plots relating to the series of Example \ref{ex_series5} part 1.}{fig:series5a}{figures/figseries5a}

\item		This looks very similar to the series that involves $e$ in Key Idea \ref{idea:famous_series}. Note, however, that the series given in this example starts with $n=1$ and not $n=0$. The first term of the series in the Key Idea is $1/0! = 1$, so we will subtract this from our result below:
\begin{align*}
		\sum_{n=1}^\infty \frac{1000}{n!} &= 1000\cdot\sum_{n=1}^\infty \frac{1}{n!} \\
							&= 1000\cdot (e-1) \approx  1718.28.
\end{align*}
This is illustrated in Figure \ref{fig:series5}(b). The graph shows how this particular series converges very rapidly.
%\mfigure{.45}{Scatter plots relating to the series of Example \ref{ex_series5} part 2.}{fig:series5b}{figures/figseries5b}
\mtable{.55}{Scatter plots relating to the series in Example \ref{ex_series5}.}{fig:series5}{%
\begin{tabular}{c}
\myincludegraphics{figures/figseries5a}\\[10pt]
(a)\\[15pt]
\myincludegraphics{figures/figseries5b}\\[10pt]
(b)
\end{tabular}
}

\item		The denominators in each term are perfect squares; we are adding $\ds \sum_{n=4}^\infty \frac{1}{n^2}$ (note  we start with $n=4$, not $n=1$). This series will converge. Using the formula from Key Idea \ref{idea:famous_series}, we have the following:
\begin{align*}
\sum_{n=1}^\infty \frac1{n^2} &= \sum_{n=1}^3 \frac1{n^2} +\sum_{n=4}^\infty \frac1{n^2} \\
\sum_{n=1}^\infty \frac1{n^2} - \sum_{n=1}^3 \frac1{n^2} &=\sum_{n=4}^\infty \frac1{n^2} \\
\frac{\pi^2}{6} - \left(\frac11+\frac14+\frac19\right) &= \sum_{n=4}^\infty \frac1{n^2} \\
\frac{\pi^2}{6} - \frac{49}{36} &= \sum_{n=4}^\infty \frac1{n^2} \\
0.2838&\approx \sum_{n=4}^\infty \frac1{n^2} 
\end{align*}
\end{enumerate}
\vskip-1.5\baselineskip
}\\

It may take a while before one is comfortable with this statement, whose truth lies at the heart of the study of infinite series: \emph{it is possible that the sum of an infinite list of nonzero numbers is finite.} We have seen this repeatedly in this section, yet it still may ``take some getting used to.''

As one contemplates the behavior of series, a few facts become clear. 
\begin{enumerate}
\item		In order to add an infinite list of nonzero numbers and get a finite result, ``most'' of those numbers must be ``very near'' 0. 
\item		If a series diverges, it means that the sum of an infinite list of numbers is not finite (it may approach $\pm \infty$ or it may oscillate), and:
		\begin{enumerate}
		\item		The series will still diverge if the first term is removed.
		\item		The series will still diverge if the first 10 terms are removed.
		\item		The series will still diverge if the first $1,000,000$ terms are removed.
		\item		The series will still diverge if any finite number of terms from anywhere in the series are removed.
		\end{enumerate}
\end{enumerate}

These concepts are very important and lie at the heart of the next two theorems.

\theorem{thm:series_nth_term}{$n^\text{th}$--Term Test for Divergence}
{Consider the series $\ds\sum_{n=1}^\infty a_n$. If $\ds \lim_{n\to\infty}a_n \neq 0$, then $\ds\sum_{n=1}^\infty a_n$ diverges.
\index{series!nth@$n^\text{th}$--term test}\index{nth@$n^\text{th}$--term test}\index{convergence!nth@$n^\text{th}$--term test}\index{divergence!nth@$n^\text{th}$--term test}
%\begin{enumerate}
%\item		If $\ds\sum_{n=1}^\infty a_n$ converges, then $\ds \lim_{n\to\infty}a_n =0$.
%\item		If $\ds \lim_{n\to\infty}a_n \neq 0$, then $\ds\sum_{n=1}^\infty a_n$ diverges.
%\end{enumerate}
}

\textbf{Important!} This theorem \emph{does not state} that if $\ds \lim_{n\to\infty} a_n = 0$ then $\ds \sum_{n=1}^\infty  a_n $ converges. The standard example of this is the Harmonic Series, as given in Key Idea \ref{idea:famous_series}. The Harmonic Sequence, $\{1/n\}$, converges to 0; the Harmonic Series, $\ds \sum_{n=1}^\infty 1/n$, diverges.

%Note that the two statements in Theorem \ref{thm:series_nth_term} are really the same. In order to converge, the limit of the terms of the sequence must approach 0; if they do not, the series will not converge. 

Looking back, we can apply this theorem to the series in Example \ref{ex_series1}. In that example, the $n^\text{th}$ terms of both sequences do not converge to 0, therefore we can quickly conclude that each series diverges.

One can rewrite Theorem \ref{thm:series_nth_term} to state ``If a series converges, then the underlying sequence converges to 0.'' While it is important to understand the truth of this statement, in practice it is rarely used. It is generally far easier to prove the convergence of a sequence than the convergence of a series. 



\theorem{thm:series_behavior}{Infinite Nature of Series}
{The convergence or divergence of an infinite series remains unchanged by the addition or subtraction of any finite number of terms. That is:
	\begin{enumerate}
	\item		A divergent series will remain divergent with the addition or subtraction of any finite number of terms.
	\item		A convergent series will remain convergent with the addition or subtraction of any finite number of terms. (Of course, the \emph{sum} will likely change.)
	\end{enumerate}
}

Consider once more the Harmonic Series $\ds\sum_{n=1}^\infty  \frac1n$ which diverges; that is, the sequence of partial sums $\{S_n\}$ grows (very, very slowly) without bound. One might think that by removing the ``large'' terms of the sequence that perhaps the series will converge. This is simply not the case. For instance, the sum of the first 10 million terms of the Harmonic Series is about 16.7. Removing the first 10 million terms from the Harmonic Series changes the $n^\text{th}$ partial sums,  effectively subtracting 16.7 from the sum. However, a sequence that is growing without bound will still grow without bound when 16.7 is subtracted from it. 

The equations below illustrate this. The first line shows the infinite sum of the Harmonic Series split into the sum of the first 10 million terms plus the sum of ``everything else.'' The next equation shows us subtracting these first 10 million terms from both sides. The final equation employs a bit of ``psuedo--math'': subtracting 16.7 from ``infinity'' still leaves one with ``infinity.''
\begin{align*}
 \parbox{50pt}{\centering$\ds\sum_{n=1}^\infty \frac1n$} &= \parbox{50pt}{\centering$\ds\sum_{n=1}^{10,000,000}\frac1n$}\quad + \parbox{50pt}{\centering$\ds\sum_{n=10,000,001}^\infty \frac1n$} \rule[-20pt]{0pt}{1pt} \\
 \parbox{50pt}{\centering$\ds\sum_{n=1}^\infty \frac1n$} - \parbox{50pt}{\centering$\ds\sum_{n=1}^{10,000,000}\frac1n$}&= \parbox{50pt}{\centering$\ds\sum_{n=10,000,001}^\infty \frac1n$} \rule[-20pt]{0pt}{1pt}\\
\parbox{50pt}{\centering	$\infty$} - \parbox{50pt}{\centering $16.7$} &=  \parbox{50pt}{\centering$\infty.$}
\end{align*}													

This section introduced us to series and defined a few special types of series whose convergence properties are well known: we know when a $p$-series or a geometric series converges or diverges. Most series that we encounter are not one of these types, but we are still interested in knowing whether or not they converge. The next three sections introduce tests that help us determine whether or not a given series converges. 


\printexercises{exercises/08_02_exercises}