\section{Calculus and Vector--Valued Functions}\label{sec:vvf_calc}

The previous section introduced us to a new mathematical object, the vector--valued function. We now apply calculus concepts to these functions. We start with the limit, then work our way through derivatives to integrals.\\

\noindent\textbf{\large Limits of Vector--Valued Functions}\\

The initial definition of the limit of a vector--valued function is a bit intimidating, as was the definition of the limit in Definition \ref{def:limit}. The theorem following the definition shows that in practice, taking limits of vector--valued functions is no more difficult than taking limits of real--valued functions.

\definition{def:vvf_limit}{Limits of Vector--Valued Functions}
{Let $I$ be an open interval containing $c$, and let $\vec r(t)$ be a vector--valued function defined on $I$, except possibly at $c$. %Let a vector--valued function $\vec r(t)$ be given, defined on an open interval $I$ containing $c$. 
The \textbf{limit of $\vec r(t)$, as $t$ approaches $c$, is $\vec L$}, expressed as 
$$\lim_{t\to c} \vec r(t) = \vec L,$$ means that given any $\epsilon>0$, there exists a $\delta>0$ such that for all $t\neq c$, if $|t-c| <\delta$, we have $\norm{\vec r(t) - \vec L} < \epsilon.$
\index{vector--valued function!limits}\index{limit!of vector--valued functions}
}\\

Note how the measurement of distance between real numbers is the absolute value of their difference; the measure of distance between vectors is the vector norm, or magnitude, of their difference.
\enlargethispage{2\baselineskip}

\theorem{thm:vvf_limit}{Limits of Vector--Valued Functions}
{\begin{enumerate}
	\item Let $\vec r(t) = \la \,f(t),g(t)\,\ra$ be a vector--valued function in $\mathbb{R}^2$ defined on an open interval $I$ containing $c$. Then
	\index{vector--valued function!limits}\index{limit!of vector--valued functions}
	$$\lim_{t\to c} \vec r(t) = \la \lim_{t\to c}f(t)\, , \,\lim_{t\to c} g(t)\ra.$$
	\item Let $\vec r(t) = \la \,f(t),g(t),h(t)\,\ra$ be a vector--valued function in $\mathbb{R}^3$ defined on an open interval $I$ containing $c$. Then 
	$$\lim_{t\to c} \vec r(t) = \la \lim_{t\to c}f(t)\, , \,\lim_{t\to c} g(t)\,, \,\lim_{t\to c} h(t)\ra$$
\end{enumerate}
}
Theorem \ref{thm:vvf_limit} states that we compute limits component--wise.\\

\example{ex_vvflimit1}{Finding limits of vector--valued functions}{
Let $\ds\vec r(t) = \la \frac{\sin t}{t},\, t^2-3t+3,\,\cos t\ra.$ Find $\ds \lim_{t\to 0}\vec r(t)$.
}
{We apply the theorem and compute limits component--wise.
\begin{align*}
\lim_{t\to0} \vec r(t) &= \la \lim_{t\to 0}\frac{\sin t}{t}\, , \, \lim_{t\to 0} t^2-3t+3\, , \, \lim_{t\to 0} \cos t\ra \\
			&= \la 1,3,1\ra.
\end{align*}
\vskip-1.5\baselineskip
}\\

\noindent\textbf{\large Continuity}\\

\definition{def:vvf_continuity}{Continuity of Vector--Valued Functions}
{Let $\vec r(t)$ be a vector--valued function defined on an open interval $I$ containing $c$.
\index{vector--valued function!continuity}\index{continuous function!vector--valued}
\begin{enumerate}
	\item $\vec r(t)$ is \textbf{continuous at $c$} if $\ds \lim_{t\to c} \vec r(t) = \vec r(c)$.
	\item	If $\vec r(t)$ is continuous at all $c$ in $I$, then $\vec r(t)$ is \textbf{continuous on $I$.}
\end{enumerate}
}

We again have a theorem that lets us evaluate continuity component--wise.
\enlargethispage{2\baselineskip}

\theorem{thm:vvf_continuity}{Continuity of Vector--Valued Functions}
{Let $\vec r(t)$ be a vector--valued function defined on an open interval $I$ containing $c$. Then $\vec r(t)$ is continuous at $c$ if, and only if, each of its component functions is continuous at $c$.
\index{vector--valued function!continuity}\index{continuous function!vector--valued}
}\\

\example{ex_vvflimit2}{Evaluating continuity of vector--valued functions}{
Let $\ds\vec r(t) = \la \frac{\sin t}{t},\, t^2-3t+3,\,\cos t\ra.$ Determine whether $\vec r$ is continuous at $t=0$ and $t=1$.}
{While the second and third components of $\vec r(t)$ are defined at $t=0$, the first component, $(\sin t)/t$, is not. Since the first component is not even defined at $t=0$, $\vec r(t)$ is not defined at $t=0$, and hence it is not continuous at $t=0$.

At $t=1$ each of the component functions is continuous. Therefore $\vec r(t)$ is continuous at $t=1$.
}\\

\noindent\textbf{\large Derivatives}\\

Consider a vector--valued function $\vec r$ defined on an open interval $I$ containing $t_0$ and $t_1$. We can compute the displacement of $\vec r$ on $[t_0,t_1]$, as shown in Figure \ref{fig:vvfderiv_intro}(a). Recall that dividing the displacement vector by $t_1-t_0$ gives the average rate of change on $[t_0,t_1]$, as shown in (b). \\

\noindent%\hskip-150pt%
\begin{minipage}{\linewidth+150pt}
\centering
\begin{tabular}{cc}
\myincludegraphics[scale=.9]{figures/figvvfderiv_intro}&\myincludegraphics[scale=.9]{figures/figvvfderiv_introb}\\
(a) & (b)
\end{tabular}
\captionsetup{type=figure}
\caption{Illustrating displacement, leading to an understanding of the derivative of vector--valued functions.}\label{fig:vvfderiv_intro}
\end{minipage}
\enlargethispage{3\baselineskip}
\vskip \baselineskip

The \textbf{derivative} of a vector--valued function is a measure of the \emph{instantaneous} rate of change, measured by taking the limit as the length of $[t_0,t_1]$ goes to 0. Instead of thinking of an interval as $[t_0,t_1]$, we think of it as $[c,c+h]$ for some value of $h$ (hence the interval has length $h$).  The \emph{average} rate of change is 
$$\frac{\vec r(c+h)-\vec r(c)}{h}$$ for any value of $h\neq0$. We take the limit as $h\to0$ to measure the instantaneous rate of change; this is the derivative of $\vec r$.

%We begin with a definition of the derivative that is very similar to Definition \ref{def:the_derivative}.

\definition{def:vvf_derivative}{Derivative of a Vector--Valued Function}
{Let $\vec r(t)$ be continuous on an open interval $I$ containing $c$.
\index{vector--valued function!derivatives}\index{derivative!vector--valued functions}
\begin{enumerate}
	\item The derivative of $\vec r$ at $t=c$ is
	$$ \vrp (c) = \lim_{h\to 0} \frac{\vec r(c+h) - \vec r(c)}{h}.$$
	\item	The derivative of $\vec r$ is
	$$ \vrp (t) = \lim_{h\to 0} \frac{\vec r(t+h) - \vec r(t)}{h}.$$
\end{enumerate}
\vskip-\baselineskip
}\\

\mnote{.35}{Alternate notations for the derivative of $\vec r$ include: $$\vrp(t) = \frac{d}{dt}\big(\,\vec r(t)\,\big) = \frac{d\vec r}{dt}.$$}
%}\\

If a vector--valued function  has a derivative for all $c$ in an open interval $I$, we say that $\vec r(t)$ is \textbf{differentiable} on $I$.

Once again we might view this definition as intimidating, but recall that we can evaluate limits component--wise. The following theorem verifies that this means we can compute derivatives component--wise as well, making the task not too difficult.
\enlargethispage{2\baselineskip}

\theorem{thm:vvf_deriv}{Derivatives of Vector--Valued Functions}
{\begin{enumerate}
	\item Let $\vec r(t) = \la \, f(t), g(t)\,\ra$. Then 
	$$\vrp(t) = \la\, \fp(t), g\primeskip'(t)\, \ra.$$
	\item Let $\vec r(t) = \la \, f(t), g(t), h(t)\,\ra$. Then
	\index{vector--valued function!derivatives}\index{derivative!vector--valued functions}
	$$\vrp(t) = \la\, \fp(t), g\primeskip'(t), h\primeskip'(t)\, \ra.$$
\end{enumerate}
}\\

\example{ex_vvflimit3}{Derivatives of vector--valued functions}{
Let $\vec r(t) = \la t^2,t\ra$. 
\begin{enumerate}
	\item Sketch $\vec r(t)$ and $\vrp(t)$ on the same axes.
	\item	Compute $\vrp(1)$ and sketch this vector with its initial point at the origin and at $\vec r(1)$.
\end{enumerate}
}
{\begin{enumerate}
\item	Theorem \ref{thm:vvf_deriv} allows us to compute derivatives component--wise, so
$$\vrp(t) = \la 2t, 1\ra.$$ $\vec r(t)$ and $\vrp(t)$ are graphed together in Figure \ref{fig:vvflimit3}(a). Note how plotting the two of these together, in this way, is not very illuminating. When dealing with real--valued functions, plotting $f(x)$ with $\fp(x)$ gave us useful information as we were able to compare $f$ and $\fp$ at the same $x$-values. When dealing with vector--valued functions, it is hard to tell which points on the graph of $\vrp$ correspond to which points on the graph of $\vec r$.

\item	We easily compute $\vrp(1) = \la 2,1\ra$, which is drawn in Figure \ref{fig:vvflimit3} with its initial point at the origin, as well as at $\vec r(1) = \la 1,1\ra.$ These are sketched in Figure \ref{fig:vvflimit3}(b).
\mtable{.4}{Graphing the derivative of a vector--valued function in Example \ref{ex_vvflimit3}.}{fig:vvflimit3}{%
\begin{tabular}{c}
\myincludegraphics{figures/figvvflimit3a}\\[5pt]
(a)\\[10pt]
\myincludegraphics{figures/figvvflimit3}\\[5pt]
(b)
\end{tabular}
}

\end{enumerate}
\vskip-1.5\baselineskip
}\\
\clearpage

\example{ex_vvflimit4}{Derivatives of vector--valued functions}{
Let $\vec r(t) = \la \cos t, \sin t, t\ra$. Compute $\vrp(t)$ and $\vrp(\pi/2)$. Sketch $\vrp(\pi/2)$ with its initial point at the origin and at $\vec r(\pi/2)$.
}
{We compute $\vrp$ as $\vrp(t) = \la -\sin t, \cos t, 1\ra$. At $t= \pi/2$, we have $\vrp(\pi/2) = \la -1,0,1\ra$. Figure \ref{fig:vvflimit4} \ifthenelse{\boolean{in_threeD}}{shows a graph of $\vec r(t)$,}{shows two graphs of $\vec r(t)$, from different perspectives,} with $\vrp(\pi/2)$ plotted with its initial point at the origin and at $\vec r(\pi/2)$.
%%%
%%%
% This could be a mess, but necessary?
% The 2D version has two pictures, the 3D has only one. Changing the text to reflect this.
%%%
%%%
\ifthenelse{\boolean{in_threeD}}{%
\mfigurethree{width=125pt,3Dmenu,activate=onclick,deactivate=onclick,
3Droll=-0.5856992334166129,
3Dortho=0.004399999976158142,
3Dc2c=0.6354137063026428 0.6317671537399292 0.4439816176891327,
3Dcoo=-18.837804794311523 -8.740551948547363 60.22870635986328,
3Droo=150.0000027221829,3Dlights=Headlamp,add3Djscript=asylabels.js}{}{.8}{Viewing a vector--valued function and its derivative at one point.}{fig:vvflimit4}{figures/figvvflimit4}
}%% ends "if in threeD"
{ % begins "else, not in threeD
\mtable{.72}{Viewing a vector--valued function, and its derivative at one point, from two different perspectives.}{fig:vvflimit4}{
\begin{tabular}{c}
\myincludegraphics[scale=1.25]{figures/figvvflimit4} \\
(a)\\
\myincludegraphics[scale=1.25]{figures/figvvflimit4a} \\
(b)
\end{tabular}}
}% ends whole if-then-else
%\vskip-1.5\baselineskip
}\\  %ends example

\ifthenelse{\boolean{in_threeD}}{\vskip\baselineskip}{}
In Examples \ref{ex_vvflimit3} and \ref{ex_vvflimit4}, sketching a particular derivative with its initial point at the origin did not seem to reveal anything significant. However, when we sketched the vector with its initial point on the corresponding point on the graph, we did see something significant: the vector appeared to be \textit{tangent} to the graph. We have not yet defined what ``tangent'' means in terms of curves in space; in fact, we use the derivative to define this term.

\definition{def:vector_tangent}{Tangent Vector, Tangent Line}
{Let $\vec r(t)$ be a differentiable vector--valued function on an open interval $I$ containing $c$, where $\vrp(c)\neq \vec 0$.
\index{tangent line}\index{vector--valued function!tangent line}
\begin{enumerate}
	\item A vector $\vec v$ is \textbf{tangent to the graph of $\vec r(t)$ at $t=c$} if $\vec v$ is parallel to $\vrp(c)$.
	\item	The \textbf{tangent line}  to the graph of $\vec r(t)$ at $t=c$ is the line through $\vec r(c)$ with direction parallel to $\vrp(c)$. An equation of the tangent line is 
	$$\vec \ell(t) = \vec r(c) + t\,\vrp(c).$$
\end{enumerate}
}\\

\example{ex_vvfderiv1}{Finding tangent lines to curves in space}{
Let $\vec r(t) = \la t,t^2,t^3\ra$ on $[-1.5,1.5]$. Find the vector equation of the line tangent to the graph of $\vec r$ at $t=-1$. }
{To find the equation of a line, we need a point on the line and the line's direction. The point is given by $\vec r(-1) = \la -1,1,-1\ra$. (To be clear, $\la -1,1,-1\ra$ is a \emph{vector}, not a point, but we use the point ``pointed to'' by this vector.)

The direction comes from $\vrp(-1)$. We compute, component--wise, $\vrp(t) = \la 1,2t, 3t^2\ra$. Thus $\vrp(-1) = \la 1,-2,3\ra$. 
\enlargethispage{2\baselineskip}

%%%
%%%   Long figure statements follow for if in/not in 3D
\ifthenelse{\boolean{in_threeD}}{% in threeD
\mfigurethree{width=125pt,3Dmenu,activate=onclick,deactivate=onclick,
3Droll=0.5961816528018784,
3Dortho=0.00514203542843461,
3Dc2c=0.8352102637290955 0.5152699947357178 0.19214747846126556,
3Dcoo=-5.658682346343994 19.23411750793457 0.2712952494621277,
3Droo=149.99999927774635,
3Dlights=Headlamp,add3Djscript=asylabels.js}{}{.35}{Graphing a curve in space with its tangent line.}{fig:vvfderiv1}{figures/figvvfderiv1}
}%end in threeD
{% begin else, not in three D
\mtable{.3}{Graphing a curve in space with its tangent line.}{fig:vvfderiv1}{%
\begin{tabular}{c}
\myincludegraphics[scale=1.25,trim=5mm 5mm 5mm 5mm,clip]{figures/figvvfderiv1}\\
(a)\\[10pt]
\myincludegraphics{figures/figvvfderiv1b}\\
(b)
\end{tabular}
}% end mtable
}% ends all of the if-then-else statements
The vector equation of the line is $\ell(t) = \la -1,1,-1\ra + t\la 1,-2,3\ra$. \ifthenelse{\boolean{in_threeD}}{This line and $\vec r(t)$ are sketched in Figure \ref{fig:vvfderiv1}.}{This line and $\vec r(t)$ are sketched, from two perspectives, in Figure \ref{fig:vvfderiv1} (a) and (b).}
}\\

\example{ex_vvfderiv3}{Finding tangent lines to curves}{
Find the equations of the lines tangent to $\vec r(t) = \la t^3,t^2\ra$ at $t=-1$ and $t=0$.}
{We find that $\vrp(t) = \la 3t^2,2t\ra$. At $t=-1$, we have
$$\vec r(-1) = \la -1,1\ra\quad \text{and}\quad \vrp(-1) = \la 3,-2\ra,$$
so the equation of the line tangent to the graph of $\vec r(t)$ at $t=-1$ is
$$\ell(t) = \la -1,1\ra + t\la 3,-2\ra.$$ This line is graphed with $\vec r(t)$ in Figure \ref{fig:vvfderiv3}.

\mfigure{.7}{Graphing $\vec r(t)$ and its tangent line in Example \ref{ex_vvfderiv3}.}{fig:vvfderiv3}{figures/figvvfderiv3}
At $t=0$, we have $\vrp(0) = \la 0,0\ra=\vec 0$! This implies that the tangent line ``has no direction.'' We cannot apply Definition \ref{def:vector_tangent}, hence cannot find the equation of the tangent line.
}\\

We were unable to compute the equation of the tangent line to $\vec r(t)= \la t^3,t^2\ra$ at $t=0$ because $\vrp(0) = \vec 0$. The graph in Figure \ref{fig:vvfderiv3} shows that there is a cusp at this point. This leads us to another definition of \textbf{smooth}, previously defined by Definition \ref{def:smooth} in Section \ref{sec:param_eqs}.

\definition{def:vector_smooth}{Smooth Vector--Valued Functions}
{Let $\vec r(t)$ be a differentiable vector--valued function on an open interval $I$ where $\vrp(t)$ is continuous on $I$. $\vec r(t)$ is \textbf{smooth} on $I$ if $\vrp(t)\neq \vec 0$ on $I$.
\index{smooth}\index{vector--valued function!smooth}
}

Having established derivatives of vector--valued functions, we now explore the relationships between the derivative and other vector operations. The following theorem states how the derivative interacts with vector addition and the various vector products.\\

\theorem{thm:vvf_deriv_prop}{Properies of Derivatives of Vector--Valued Functions}
{Let $\vec r$ and $\vec s$ be differentiable vector--valued functions, let $f$ be a differentiable real--valued function, and let $c$ be a real number.
\index{vector--valued function!derivatives}\index{derivative!vector--valued functions}
\index{dot product!and derivatives}\index{cross product!and derivatives}
\begin{enumerate}
	\item $\ds \frac{d}{dt}\Big(\vec r(t) \pm \vec s(t)\Big) = \vrp(t) \pm \vec s\,'(t)$
	\item $\ds \frac{d}{dt}\Big(c\vec r(t)\Big) = c\vrp(t)$
	\item \parbox{200pt}{$\ds \frac{d}{dt}\Big(f(t)\vec r(t)\Big) = \fp(t)\vec r(t) + f(t)\vrp(t)$} \textbf{Product Rule}
	\item \parbox{200pt}{$\ds \frac{d}{dt}\Big(\vec r(t)\cdot \vec s(t) \Big) = \vrp(t)\cdot \vec s(t) + \vec r(t)\cdot \vec s\,'(t)$} \textbf{Product Rule}
	\item \parbox{200pt}{$\ds \frac{d}{dt}\Big(\vec r(t)\times \vec s(t) \Big) = \vrp(t)\times \vec s(t) + \vec r(t)\times \vec s\,'(t)$} \textbf{Product Rule}
	\item \parbox{200pt}{$\ds \frac{d}{dt}\Big(\vec r\big(f(t)\big)\Big) = \vrp\big(f(t)\big)\fp(t)$}  \textbf{Chain Rule}
	\end{enumerate}
}\\

\example{ex_vvfderiv2}{Using derivative properties of vector--valued functions}{
Let $\vec r(t) = \la t, t^2-1\ra$ and let $\vec u(t)$ be the unit vector that points in the direction of $\vec r(t)$.
\begin{enumerate}
	\item Graph $\vec r(t)$ and $\vec u(t)$ on the same axes, on $[-2,2]$.
	\item	Find $\vec u\,'(t)$ and sketch $\vec u\,'(-2)$, $\vec u\,'(-1)$ and $\vec u\,'(0)$. Sketch each with initial point the corresponding point on the graph of $\vec u$.
\end{enumerate}
}
{\begin{enumerate}
	\item To form the unit vector that points in the direction of $\vec r$, we need to divide $\vec r(t)$ by its magnitude. 
	$$\norm{\vec r(t)} = \sqrt{t^2+(t^2-1)^2} \quad \Rightarrow \quad \vec u(t) = \frac{1}{\sqrt{t^2+(t^2-1)^2}}\la t,t^2-1\ra.$$
	
	$\vec r(t)$ and $\vec u(t)$ are graphed in Figure \ref{fig:vvfderiv2a}. Note how the graph of $\vec u(t)$ forms part of a circle; this must be the case, as the length of $\vec u(t)$ is 1 for all $t$.
	\mfigure{.5}{Graphing $\vec r(t)$ and $\vec u(t)$ in Example \ref{ex_vvfderiv2}.}{fig:vvfderiv2a}{figures/figvvfderiv2a}
	%\drawexampleline
	
	\item		To compute $\vec u\,'(t)$, we use Theorem \ref{thm:vvf_deriv_prop}, writing $$\vec u(t) = f(t)\vec r(t),\quad  \text{where}\quad f(t) = \frac{1}{\sqrt{t^2+(t^2-1)^2}}=\big(t^2+(t^2-1)^2\big)^{-1/2}.$$ (We \emph{could} write $$\vec u(t) = \la \frac{t}{\sqrt{t^2+(t^2-1)^2}}, \frac{t^2-1}{\sqrt{t^2+(t^2-1)^2}}\ra$$ and then take the derivative. It is a matter of preference; this latter method requires two applications of the Quotient Rule where our method uses the Product and Chain Rules.)
	
We find $\fp(t)$ using the Chain Rule:
\begin{align*}
\fp(t) &= -\frac12\big(t^2+(t^2-1)^2\big)^{-3/2}\big(2t+2(t^2-1)(2t)\big)\\
			&= -\frac{2t(2t^2-1)}{2\big(\sqrt{t^2+(t^2-1)^2}\,\big)^3}
\end{align*}
We now find $\vec u\,'(t)$ using part 3 of Theorem \ref{thm:vvf_deriv_prop}:
\begin{align*}
\vec u\,'(t) &=  \fp(t)\vec u(t) + f(t)\vec u\,'(t) \\
				&=  -\frac{2t(2t^2-1)}{2\big(\sqrt{t^2+(t^2-1)^2}\,\big)^3}\la t,t^2-1\ra + \frac{1}{\sqrt{t^2+(t^2-1)^2}}\la 1,2t\ra.
\end{align*}
This is admittedly very ``messy;'' such is usually the case when we deal with unit vectors. We can use this formula to compute $\vec u\,'(-2)$, $\vec u\,'(-1)$ and $\vec u\,'(0)$:
\begin{align*}
\vec u\,'(-2) &= \la-\frac{15}{13 \sqrt{13}},-\frac{10}{13
   \sqrt{13}}\ra \approx \la -0.320,-0.213\ra\\
\vec u\,'(-1) &= \la 0,-2\ra\\
\vec u\,'(0) &= \la 1,0\ra
\end{align*}
\mfigure{.75}{Graphing some of the derivatives of $\vec u(t)$ in Example \ref{ex_vvfderiv2}.}{fig:vvfderiv2b}{figures/figvvfderiv2b}
Each of these is sketched in Figure \ref{fig:vvfderiv2b}. Note how the length of the vector gives an indication of how quickly the circle is being traced at that point. When $t=-2$, the circle is being drawn relatively slow; when $t=-1$, the circle is being traced much more quickly.
\end{enumerate}
\vskip-1.5\baselineskip
}\\

It is a basic geometric fact that a line tangent to a circle at a point $P$ is perpendicular to the line passing through the center of the circle and $P$. This is illustrated in Figure \ref{fig:vvfderiv2b}; each tangent vector is perpendicular to the line that passes through its initial point and the center of the circle. Since the center of the circle is the origin, we can state this another way: $\vec u\,'(t)$ is orthogonal to $\vec u(t)$.

Recall that the dot product serves as a test for orthogonality: if $\vec u\cdot \vec v = 0$, then $\vec u$ is orthogonal to $\vec v$. Thus in the above example, $\vec u(t)\cdot \vec u\,'(t)=0$.

This is true of any vector--valued function that has a constant length, that is, that traces out part of a circle. It has important implications later on, so we state it as a theorem (and leave its formal proof as an Exercise.)

\theorem{thm:vects_of_constant_length}{Vector--Valued Functions of Constant Length% are Orthogonal to their Derivative
}
{Let $\vec r(t)$ be a differentiable vector--valued function on an open interval $I$ of constant length. That is, $\norm{\vec r(t)} = c$ for all $t$ in $I$ (equivalently, $\vec r(t)\cdot \vec r(t) = c^2$ for all $t$ in $I$). 
Then $\vec r(t)\cdot\vrp(t) = 0$ for all $t$ in $I$.\index{vector--valued function!of constant length}
}\\

\noindent\textbf{\large Integration}\\

We can define the indefinite integral of a vector--valued function in the same manner we defined indefinite integrals in Definition \ref{def:antider}. However, we cannot define the definite integral of a vector--valued function as we did in Definition \ref{def:def_int}. That definition was based on the signed area between a function $y=f(x)$ and the $x$-axis. An area--based definition will not be useful in the context of vector--valued functions.
Instead, we define the definite integral of a vector--valued function in a manner similar to that of Theorem \ref{thm:riemann_sum}, utilizing Riemann sums. 

\definition{def:vvf_integral}{\parbox[t]{225pt}{Antiderivatives, Indefinite and Definite Integrals of Vector--Valued Functions}}
{Let $\vec r(t)$ be a continuous vector--valued function on $[a,b]$. An \sword{antiderivative} of $\vec r(t)$ is a function $\vec R(t)$ such that $\vec R'(t) = \vec r(t)$.\\

The set of all antiderivatives of $\vec r(t)$ is the \sword{indefinite integral of $\vec r(t)$}, denoted by 
$$\int \vec r(t)\ dt.$$

The definite integral $\int_a^b \vec r(t)\ dt$ of $\vec r(t)$ on $[a,b]$ is 
$$\int_a^b \vec r(t)\ dt =\lim_{||\Delta t||\to 0} \sum_{i=1}^n\vec r(c_i)\Delta t_i,$$ where $\Delta t_i$ is the length of the $i^{\,\text{th}}$ subinterval of a partition of $[a,b]$, $||\Delta t||$ is the length of the largest subinterval in the partition, and $c_i$ is any value in the $i^{\,\text{th}}$ subinterval.%
\index{antiderivative!of vector--valued function}%
\index{definite integral!of vector--valued function}%
\index{indefinite integral!of vector--valued function}%
\index{integration!of vector--valued function}%
}


While the definition of the definite integral may look complicated, in practice computing indefinite and definite integrals of vector--valued functions is accomplished component--wise, as allowed by the following theorem.


%Indefinite and definite integrals of vector--valued functions are also evaluated component--wise.

\theorem{thm:vvf_integration}{\parbox[t]{225pt}{Indefinite and Definite Integrals of Vector--Valued Functions}}
{Let $\vec r(t) = \la f(t),g(t)\ra$ be a vector--valued function in $\mathbb{R}^2$.
\begin{enumerate}
	\item $\ds \int \vec r(t)\ dt = \la \int f(t)\ dt, \int g(t)\ dt\ra$
	\item	$\ds \int_a^b \vec r(t)\ dt = \la \int_a^b f(t)\ dt, \int_a^b g(t)\ dt\ra$
\end{enumerate}
A similar statement holds for vector--valued functions in $\mathbb{R}^3$.
\index{vector--valued function!integration}\index{integration!of vector--valued functions}
}

\example{ex_vvfint1}{Evaluating a definite integral of a vector--valued function}{
Let $\vec r(t) = \la e^{2t},\sin t\ra$. Evaluate $\ds \int_0^1 \vec r(t) \ dt$.
}
{We follow Theorem \ref{thm:vvf_integration}.
\begin{align*}
\int_0^1 \vec r(t) \ dt &= \int_0^1 \la e^{2t},\sin t\ra \ dt \\
				&= \la \int_0^1 e^{2t}\ dt\ , \int_0^1 \sin t \ dt \ra \\
				&= \la \frac12e^{2t}\Big|_0^1\ , -\cos t\Big|_0^1\ra \\
				&= \la \frac12(e^2-1)\ , -\cos(1)+1\ra\\
				&\approx \la 3.19,0.460\ra.
\end{align*}
\vskip-1.5\baselineskip
}\\

\example{ex_vvfint2}{Solving an initial value problem}{
Let $\vrp'(t) = \la 2, \cos t, 12t\ra$. Find $\vec r(t)$ where:
\begin{itemize}
	\item $\vec r(0) = \la-7,-1,2\ra$ and
	\item	$\vrp(0) = \la 5,3,0\ra.$
\end{itemize}
}
{Knowing $\vrp'(t) = \la 2,\cos t, 12t\ra$, we find $\vrp(t)$ by evaluating the indefinite integral.
\begin{align*}
\int \vrp'(t)\ dt &= \la \int 2\ dt\ , \int \cos t\ dt\ , \int 12t\ dt\ra \\
						&= \la 2t+C_1, \sin t+ C_2, 6t^2 + C_3\ra \\
						&= \la 2t,\sin t,6t^2 \ra + \la C_1,C_2,C_3\ra \\
						&= \la 2t,\sin t,6t^2 \ra + \vec C.
\end{align*}
Note how each indefinite integral creates its own constant which we collect as one constant vector $\vec C$. Knowing $\vrp(0) = \la 5,3,0\ra$ allows us to solve for $\vec C$:
\begin{align*}
\vrp(t) & = \la 2t,\sin t,6t^2 \ra + \vec C\\
\vrp(0) &=   \la 0,0,0 \ra + \vec C\\
\la 5,3,0\ra &= \vec C.
\end{align*}
\enlargethispage{2\baselineskip}

So $\vrp(t) = \la 2t,\sin t,6t^2\ra + \la 5,3,0\ra = \la 2t+5, \sin t + 3, 6t^2\ra$. To find $\vec r(t)$, we integrate once more.

\begin{align*}
\int \vrp(t)\ dt &= \la \int 2t+5\ dt, \int \sin t + 3\ dt, \int 6t^2\ dt \ra\\
							&= \la t^2+5t, -\cos t + 3t, 2t^3\ra + \vec C.
\end{align*}
With $\vec r(0) = \la -7,-1,2\ra$, we solve for $\vec C$:
\clearpage
\rule{0pt}{1pt}\vskip-3\baselineskip
\begin{align*}
\vec r(t) &= \la t^2+5t, -\cos t + 3t, 2t^3\ra + \vec C\\
\vec r(0) &= \la 0,-1,0\ra + \vec C\\
\la -7,-1,2\ra &= \la 0,-1,0\ra + \vec C\\
\la -7,0,2\ra &= \vec C.
\end{align*}
So $\vec r(t) = \la t^2+5t, -\cos t + 3t, 2t^3\ra + \la -7,0,2\ra = \la t^2+5t-7,-\cos t+3t,2t^3+2\ra.$
\vskip-\baselineskip
}\\

What does the integration of a vector--valued function \emph{mean}? There are many applications, but none as direct as ``the area under the curve'' that we used in understanding the integral of a real--valued function.

A key understanding for us comes from considering the integral of a derivative: $$\int_a^b \vrp(t)\ dt = \vec r(t)\Big|_a^b = \vec r(b)-\vec r(a).$$ Integrating a \emph{rate of change} function gives \emph{displacement}.\index{displacement}

Noting that vector--valued functions are closely related to parametric equations, we can describe the arc length of the graph of a vector--valued function as an integral. Given parametric equations $x=f(t)$, $y=g(t)$, the arc length on $[a,b]$ of the graph is
$$\text{Arc Length} = \int_a^b\sqrt{\fp(t)^2+g\primeskip'(t)^2}\ dt,$$
as stated in Theorem \ref{thm:arc_length_parametric} in Section \ref{sec:par_calc}. If $\vrt = \la f(t), g(t)\ra$, note that $\sqrt{\fp(t)^2+g\primeskip'(t)^2} = \norm{\vrp(t)}$. Therefore we can express the arc length of the graph of a vector--valued function as an integral of the magnitude of its derivative.

\theorem{thm:vvf_arc_length}{Arc Length of a Vector--Valued Function}
{Let \vrt\ be a vector--valued function where $\vrp(t)$ is continuous on $[a,b]$. The arc length $L$ of the graph of \vrt\ is 
\index{vector--valued function!arc length}\index{arc length}
$$L = \int_a^b \norm{\vrp(t)}\ dt.$$
}

Note that we are actually integrating a scalar--function here, not a vector--valued function.

The next section takes what we have established thus far and applies it to objects in motion. We will let \vrt\ describe the path of an object in the plane or in space and will discover the information provided by $\vrp(t)$ and $\vrp'(t)$.

\printexercises{exercises/11_02_exercises}