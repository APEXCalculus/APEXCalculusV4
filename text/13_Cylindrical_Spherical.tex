\section{Triple Integration with Cylindrical and Spherical Coordinates}\label{sec:cylindrical_spherical}

Just as polar coordinates gave us a new way of describing curves in the plane, in this section we will see how \emph{cylindrical} and \emph{spherical} coordinates give us new ways of desribing surfaces and regions in space.

\vskip\baselineskip
\noindent\textbf{\large Cylindrical Coordinates}\\

In short, cylindrical coordinates can be thought of as a combination of the polar and rectangular coordinate systems. %One can identify a location in the $x$-$y$ plane using polar coordinates (i.e., using $(r,\theta)$), then use a $z$-value to determine the distance
One can identify a point $(x_0,y_0,z_0)$, given in rectangular coordinates, with the point $(r_0,\theta_0,z_0)$, given in cylindrical coordinates, where the $z$-value in both systems is the same, and the point $(x_0,y_0)$ in the $x$-$y$ plane is identified with the polar point $P(r_0,\theta_0)$; see Figure \ref{...}. So that each point in space that does not lie on the $z$-axis is defined uniquely, we will restrict $r\geq 0$ and $0\leq \theta\leq 2\pi$.
\index{cylindrical coordinates}\index{coordinates!cylindrical}

We use the identity $z=z$ along with the identities found in Key Idea \ref{idea:polarconvert} to convert between the  rectangular coordinate $(x,y,z)$ and the  cylindrical coordinate $(r,\theta,z)$, namely
$$\begin{array}{l}
\text{From rectangular to cylindrical:}\quad r=\sqrt{x^2+y^2},\quad \tan\theta = y/x \quad\text{and}\quad z=z;\\
\text{From cylindrical to rectangular:} \quad x=r\cos\theta\quad y=r\sin\theta\quad \text{and}\quad z=z.\end{array}$$
These identities, along with conversions related to spherical coordinates, are given later in Key Idea \ref{...}.
\mnote{.58}{\textbf{Note:} Our rectangular to polar conversion formulas used $r^2=x^2+y^2$, allowing for negative $r$ values. Since we now restrict $r\geq 0$, we can use $r=\sqrt{x^2+y^2}$.}

Setting each of $r$, $\theta$ and $z$ equal to a constant defines a surface in space, as illustrated in the following example.\\

\example{ex_cylindrical1}{Canonical surfaces in cylindrical coordinates}%
{Describe the surfaces $r=1$, $\theta = \pi/3$ and $z=2$, given in cylindrical coordinates.
}
{The equation $r=1$ describes all points in space that are 1 unit away from the $z$-axis. This surface is a ``tube'' or ``cylinder'' of radius 1, centered on the $z$-axis, as graphed in Figure \ref{fig:spacecylinder1} (which describes the cylinder $x^2+y^2=1$ in space). 

The equation $\theta=\pi/3$ describes the plane formed by extending the line $\theta=\pi/3$, as given by polar coordinates in the $x$-$y$ plane, parallel to the $z$-axis.

The equation $z=2$ describes the plane of all points in space that are 2 units above the $x$-$y$ plane. This plane is the same as the plane described by $z=2$ in rectangular coordinates.

All three surfaces are graphed in Figure \ref{fig:cylindrical1}. Note how their intersection uniquely defines the point $P=(1,\pi/3,2)$.
}\\

Cylindrical coordinates are useful when describing certain domains in space, allowing us to evaluate triple integrals over these domains more easily than if we used rectangular coordinates.

Theorem \ref{thm:triple_integration2} shows how to evaluate $\iiint_Dh(x,y,z)\ dV$ using rectangular coordinates. In that evaluation, we use $dV = dz\,dy\,dx$ (or one of the other five orders of integration). Recall how, in this order of integration, the bounds on $y$ are ``curve to curve'' and the bounds on $x$ are ``point to point'': these bounds describe a region $R$ in the $x$-$y$ plane. We could describe $R$ using polar coordinates as done in Section \ref{sec:double_int_polar}. In that section, we saw how we used $dA = r\,dr\,d\theta$ instead of $dA = dy\,dx$. 

Considering the above thoughts, we have $dV = dz\big(r\,dr\,d\theta\big) = r\,dz\,dr\,d\theta$. We set bounds on $z$ as ``surface to surface'' as done in the previous section, and then use ``curve to curve'' and ``point to point'' bounds on $r$ and $\theta$, respectively. Finally, using the identities given above, we replace the integrand $h(x,y,z)$ to $h(r,\theta,z)$.

This process should sound plausible; the following theorem states it is truly a way of evaluating a triple integral.

\theorem{thm:triple_int_cylindrical}{Triple Integration in Cylindrical Coordinates}
{%
Let $w=h(r,\theta,z)$ be a continuous function on a closed, bounded region $D$ in space, bounded in cylindrical coordinates by $\alpha \leq \theta \leq \beta$, $g_1(\theta)\leq r \leq g_2(\theta)$ and $f_1(r,\theta) \leq z \leq f_2(r,\theta)$. Then 
$$\iiint_D h(r,\theta,z)\ dV = \int_\alpha^\beta\int_{g_1(\theta)}^{g_2(\theta)}\int_{f_1(r,\theta)}^{f_2(r,\theta)}h(r,\theta,z) r\,dz\,dr\,d\theta.$$
}

\example{ex_cylindrical2}{Evaluating a triple integral with cylindrical coordinates}
{Find the mass of the solid represented by the region in space bounded by $z=0$, $z=\sqrt{4-x^2-y^2}+3$ and the cylinder $x^2+y^2=4$ (as shown in Figure \ref{fig:cylindrical2}), with density function $\delta(x,y,z) = x^2+y^2+z+1$, using a triple integral in cylindrical coordinates. Distances are measured in centimeters and density is measured in grams/cm$^3$.
}
{We begin by describing this region of space with cylindrical coordinates. The plane $z=0$ is left unchanged; with the identity $r=\sqrt{x^2+y^2}$, we convert the hemisphere of radius 2 to the equation $z=\sqrt{4-r^2}$; the cylinder $x^2+y^2=4$ is converted to $r^2=4$, or, more simply, $r=2$.  We also convert the density function: $\delta(r,\theta,z) = r^2+z+1$. 

To describe this solid with the bounds of a triple integral, we bound $z$ with $0\leq z\leq \sqrt{4-r^2}+3$; we bound $r$ with $0 \leq r \leq 2$; we bound $\theta$ with $0 \leq \theta \leq 2\pi$.

Using Definition \ref{def:mass_3d} and Theorem \ref{thm:triple_int_cylindrical}, we have the mass of the solid is
\begin{align*}
M=\iiint_D\delta(x,y,z)\ dV &= \int_0^{2\pi}\int_0^2\int_0^{\sqrt{4-r^2}+3}\big(r^2+z+1\big)r\,dz\,dr\,d\theta \\
&= \int_0^{2\pi}\int_0^2\big((r^3+4r)\sqrt{4-r^2}+\frac52r^3+\frac{19}2r\big)\,dr\,d\theta \\
&= \frac{1318\pi}{15} \approx 276.04\text{ gm},
\end{align*}
where we leave the details of the remaining double integral to the reader.
}\\

\example{ex_cylindrical3}{Finding the center of mass using cylindrical coordinates}
{Find the center of mass of the solid with constant density whose base can be described by the polar curve $r=\cos(3\theta)$ and whose top is defined by the plane $z=1-x+0.1y$, where distances are measured in feet, as seen in Figure \ref{fig:cylindrical3}. (The volume of this solid was found in Example \ref{ex_doublepol4}.)
}
{We convert the equation of the plane to use cylindrical coordinates: $z= 1-r\cos\theta+0.1r\sin\theta$. Thus the region is space is bounded by $0 \leq z \leq 1-r\cos\theta + 0.1r\sin\theta$, $0 \leq r \leq \cos(3\theta)$, $0 \leq \theta \leq \pi$ (recall that the rose curve $r=\cos(3\theta)$ is traced out once on $[0,\pi]$.

Since density is constant, we set $\delta = 1$ and finding the mass is equivalent to finding the volume of the solid. We set up the triple integral to compute this but do not evaluate it; we leave it to the reader to confirm it evaluates to the same result found in Example \ref{ex_doublepol4}.
$$M = \iiint_D\delta \, dV = \int_0^{\pi}\int_0^{\cos(3\theta)}\int_0^{1-r\cos\theta+0.1r\sin\theta} r\,dz\,dr\,d\theta \approx 0.785.$$

From Definition \ref{def:mass_3d} we set up the triple integrals to compute the moments about the three coordinate planes. The computation of each is left to the reader (using technology is recommended):
\begin{align*}
M_{yz} = \iiint_D x\,dV &= \int_0^{\pi}\int_0^{\cos(3\theta)}\int_0^{1-r\cos\theta+0.1r\sin\theta} (r\cos\theta) r\,dz\,dr\,d\theta\\
&= -0.147.
\end{align*}
\begin{align*}
M_{xz} = \iiint_D y\,dV &= \int_0^{\pi}\int_0^{\cos(3\theta)}\int_0^{1-r\cos\theta+0.1r\sin\theta} (r\sin\theta) r\,dz\,dr\,d\theta\\
&= 0.015.\\
M_{xy} = \iiint_D z\,dV &= \int_0^{\pi}\int_0^{\cos(3\theta)}\int_0^{1-r\cos\theta+0.1r\sin\theta} (z) r\,dz\,dr\,d\theta\\
 &= 0.467.
\end{align*}
The center of mass, in rectangular coordinates,  is located at $(-0.147,0.015,0.467)$, as indicated in the figure.
}\\

\noindent\textbf{\large Spherical Coordinates}
\index{spherical coordinates}\index{coordinates!spherical}

In short, spherical coordinates can be thought of as a ``double application'' of the polar coordinate system. In spherical coordinates, a point $P$ is identified with $(\rho,\theta,\varphi)$, where $\rho$ is the distance from the origin to $P$, $\theta$ is the same angle as would be used to describe $P$ in the cylindrical coordinate system, and $\varphi$ is the angle between the positive $z$-axis and the ray from the origin to $P$; see Figure \ref{..}. So that each point in space that does not lie on the $z$-axis is defined uniquely, we will restrict $\rho \geq 0$, $0 \leq \theta \leq 2\pi$ and $0 \leq \varphi \leq \pi$.
\mnote{.7}{\textbf{Note:} The symbol $\rho$ is the Greek letter ``rho.'' Traditionally it is used in the spherical coordinate system, while $r$ is used in the polar and cylindrical coordinate systems.% In space, $r$ and $\rho$ play different roles. In spherical coordinates, $\rho$ is the dist
} 

The following Key Idea gives conversions to/from our three spatial coordinate systems.

\keyidea{idea:convert_coords}{\parbox[t]{210pt}{Converting Between Rectangular, Cylindrical and Spherical Coordinates}}
{\textbf{Rectangular and Cylindrical}
%\parbox{82pt}{$(x,y,z)\rightarrow (r,\theta,z)$:} 
$$\begin{array}{c}
r^2 = x^2+y^2,\quad \tan \theta = y/x,\quad z=z\\
%\parbox{82pt}{$(r,\theta,z)\rightarrow (x,y,z)$:} 
x=r\cos \theta, \quad y=r\sin\theta,\quad z=z
\end{array}$$

\textbf{Rectangular and Spherical}
$$\begin{array}{c}
\rho = \sqrt{x^2+y^2+z^2},\quad \tan \theta = y/x,\quad \cos \varphi = z/\sqrt{x^2+y^2+z^2}\\
x=\rho\sin\varphi\cos\theta,\quad y=\rho\sin\varphi\sin\theta,\quad z=\rho\cos\varphi
\end{array}$$

\textbf{Cylindrical and Spherical }
$$\begin{array}{c}
\rho=\sqrt{r^2+z^2}, \quad \theta = \theta,\quad \tan \varphi = r/z \\
r=\rho \sin \varphi, \quad \theta = \theta, \quad z=\rho\cos\varphi
\end{array}$$ 
}

Spherical coordinates are useful when describing certain domains in space, allowing us to evaluate triple integrals over these domains more easily than if we used rectangular coordinates or cylindrical coordinates.





%\theorem{thm:volume_between_surfaces}{Volume Between Surfaces}
%{Let $f$ and $g$ be continuous functions on a closed, bounded region $R$, where $f(x,y)\geq g(x,y)$ for all $(x,y)$ in $R$. The volume $V$ between $f$ and $g$ over $R$ is
%\index{volume}
%$$V =\iint_R \big(f(x,y)-g(x,y)\big)\ dA.$$
%}

%\example{ex_trip1}{Finding volume between surfaces}{
%Find the volume of the space region bounded by the planes $z=3x+y-4$, $z=8-3x-2y$, $x=0$ and $y=0$. In Figure \ref{fig:trip1}(a) the planes are drawn; in (b), only the defined region is given.
%\mtable{.45}{Finding the volume between the planes given in Example \ref{ex_trip1}.}{fig:trip1}{%
%\begin{tabular}{c}
%\myincludegraphicsthree{width=150pt,3Dmenu,activate=onclick,deactivate=onclick,
%3Droll=0,
%3Dortho=0.005003686994314194,
%3Dc2c=0.5144106149673462 -0.7084314823150635 0.48322510719299316,
%3Dcoo=69.411376953125 50.58059310913086 38.15925598144531,
%3Droo=149.99999640034895,
%3Dlights=Headlamp,add3Djscript=asylabels.js}{}{figures/figtrip1}\\
%%\myincludegraphics[scale=1.3,trim=1mm 5mm 5mm 0mm,clip]{figures/figtrip1}\\
%(a)\\
%\myincludegraphicsthree{width=150pt,3Dmenu,activate=onclick,deactivate=onclick,
%3Droll=0,
%3Dortho=0.005003686994314194,
%3Dc2c=0.5144106149673462 -0.7084314823150635 0.48322510719299316,
%3Dcoo=69.411376953125 50.58059310913086 38.15925598144531,
%3Droo=149.99999640034895,
%3Dlights=Headlamp,add3Djscript=asylabels.js}{}{figures/figtrip1b}\\
%%\myincludegraphics[scale=1.3,trim=1mm 5mm 5mm 10mm,clip]{figures/figtrip1b}\\
%(b)
%\end{tabular}
%}
%}
%{We need to determine the region $R$ over which we will integrate. To do so, we need to determine where the planes intersect. They have common $z$-values when $3x+y-4=8-3x-2y$. Applying a little algebra, we have:
%\begin{align*}
%3x+y-4 &= 8-3x-2y\\
%6x+3y &=12\\
%2x+y &=4
%\end{align*}
%The planes intersect along the line $2x+y=4$. Therefore the region $R$ is bounded by $x=0$, $y=0$, and $y=4-2x$; we can convert these bounds to integration bounds of $0\leq x\leq 2$, $0\leq y\leq 4-2x$. Thus
%\begin{align*}
%V &= \iint_R \big(8-3x-2y-(3x+y-4)\big)\ dA \\
	%&= \int_0^2\int_0^{4-2x} \big(12-6x-3y\big)\ dy\ dx\\
	%&= 16\text{u}^3.
%\end{align*}
%The volume between the surfaces is $16$ cubic units.
%}\\

\printexercises{exercises/13_06_exercises}