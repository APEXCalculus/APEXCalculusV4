In previous chapters we have explored a relationship between vectors and integration. Our most tangible result: if $\vec v(t)$ is the vector--valued velocity function of a moving object, then integrating $\vec v(t)$ from $t=a$ to $t=b$ gives the displacement of that object over that time interval.

This chapter explores completely different relationships between vectors and integration. These relationships will enable us to compute the work done by a magnetic field in moving an object along a path and find how much air moves through an oddly--shaped screen in space, among other things. 

Our upcoming work with integration will benefit from a review. We are not concerned here with techniques of integration, but rather what an integral ``does'' and how that relates to the notation we use to describe it.\\

\noindent\textbf{\large Integration Review}
\vskip \baselineskip

Recall from Section \ref{sec:iterated_integrals} that when $R$ is a region in the $x$-$y$ plane, $ \iint_R dA$ gives the area of the region $R$. The integral symbols are ``elongated esses'' meaning ``sum'' and $dA$ represents ``a small amount of area.'' Taken together, $\iint_R dA$ means ``sum up, over $R$, small amounts of area.'' This sum then gives the total area of $R$. We use two integral symbols since $R$ is a two--dimensional region.

Now let $z=f(x,y)$ represent a surface. The double integral $\iint_R f(x,y)\ dA$ means ``sum up, over $R$, function values (heights) given by $f$ times small amounts of area.'' Since ``height $\times$ area = volume,'' we are summing small amounts of volume over $R$, giving the total signed volume under the surface $z=f(x,y)$ and above the $x$-$y$ plane.

This notation does not directly inform us \textit{how} to evaluate the double integrals to find an area or a volume. With additional work, we recognize that a small amount of area $dA$ can be measured as the area of a small rectangle, with one side length a small change in $x$ and the other side length a small change in $y$. That is, $dA = dx\,dy$ or $dA = dy\,dx$. We could also compute a small amount of area by thinking in terms of polar coordinates, where $dA = r\,dr\,d\theta$. These understandings lead us to the iterated integrals we used in Chapter \ref{chapter:mult_int}.

Let us back our review up farther. Note that $\int_1^3\ dx = x\big|_1^3 = 3-1 = 2$. We have simply measured the length of the interval $[1,3]$. We could rewrite the above integral using syntax similar to the double integral syntax above:
$$\int_1^3\ dx = \int_Idx,\quad \text{ where $I$ = $[1,3]$}.$$

We interpret ``$\int_I dx$'' as meaning ``sum up, over the interval $I$, small changes in $x$.'' A change in $x$ is a length along the $x$-axis, so we are adding up along $I$ small lengths, giving the total length of $I$. 

We could also write $\int_1^3f(x)\ dx$ as $\int_I f(x)\ dx$, interpreted as ``sum up, over $I$, heights given by $y = f(x)$ times small changes in $x$.'' Since ``height$\times$length = area,'' we are summing up areas and finding the total signed area between $y = f(x)$ and the $x$-axis. 

This method of referring to the process of integration can be very powerful. It is the core of our notion of the Riemann Sum. When faced with a quantity to compute, if one can think of a way to approximate its value through a sum, the one is well on their way to constructing an integral (or, double or triple integral) that computes the desired quantity. We will demonstrate this process throughout this chapter, starting with the next section.

\section{Introduction to Line Integrals}\label{sec:line_int_intro}
We first used integration to find ``area under a curve.'' In this section, we learn to do this (again), but in a different context.

Consider the surface and curve shown in Figure \ref{fig:lineint0}(a). The surface is given by $f(x,y)=1-\cos(x)\sin(y)$. The dashed curve lies in the $x$-$y$ plane and is the familiar $y=x^2$ parabola from $-1\leq x\leq1$; we'll call this curve $C$. The curve drawn with a solid line in the graph is the curve in space that lies on our surface with $x$ and $y$ values that lie on $C$. 

The question we want to answer is this: what is the area that lies below the curve drawn with the solid line? In other words, what is the area of the region above $C$ and under the the surface $f$? This region is shown in Figure \ref{fig:lineint0}(b).

We suspect the answer can be found using an integral, but before trying to figure out what that integral is, let us first try to approximate its value. 

\mtable{.45}{Finding area under a curve in space.}{fig:lineint0}{%
\begin{tabular}{c}
\myincludegraphicsthree{width=150pt,3Dmenu,activate=onclick,deactivate=onclick,
3Droll=0,
3Dortho=0.005003686994314194,
3Dc2c=0.29708048701286316 0.8962657451629639 0.329318642616272,
3Dcoo=28.221464157104492 62.0896110534668 72.89360809326172,
3Droo=149.99999640034895,
3Dlights=Headlamp,add3Djscript=asylabels.js}{width=150pt}{figures/figline_integral_intro1}\\
%\myincludegraphics[scale=1.3,trim=1mm 5mm 5mm 0mm,clip]{figures/figtrip1}\\
(a)\\[10pt]
\myincludegraphicsthree{width=150pt,3Dmenu,activate=onclick,deactivate=onclick,
3Droll=0,
3Dortho=0.005003686994314194,
3Dc2c=0.29708048701286316 0.8962657451629639 0.329318642616272,
3Dcoo=28.221464157104492 62.0896110534668 72.89360809326172,
3Droo=149.99999640034895,
3Dlights=Headlamp,add3Djscript=asylabels.js}{width=150pt}{figures/figline_integral_intro1b}\\
%\myincludegraphics[scale=1.3,trim=1mm 5mm 5mm 10mm,clip]{figures/figtrip1b}\\
(b)\\[10pt]
\myincludegraphicsthree{width=150pt,3Dmenu,activate=onclick,deactivate=onclick,
3Droll=0,
3Dortho=0.005003686994314194,
3Dc2c=0.29708048701286316 0.8962657451629639 0.329318642616272,
3Dcoo=28.221464157104492 62.0896110534668 72.89360809326172,
3Droo=149.99999640034895,
3Dlights=Headlamp,add3Djscript=asylabels.js}{width=150pt}{figures/figline_integral_intro1c}\\
%\myincludegraphics[scale=1.3,trim=1mm 5mm 5mm 10mm,clip]{figures/figtrip1b}\\
(c)
\end{tabular}
}

In Figure \ref{fig:lineint0}(c), four rectangles have been drawn over the curve $C$. The bottom corners of each rectangle lie on $C$, and each rectangle has a height given by the function $f(x,y)$ for some $(x,y)$ pair along $C$ between the rectangle's bottom corners. 

As we know how to find the area of each rectangle, we are able to approximate the area above $C$ and under $f$. Clearly, our approximation will be \textit{an approximation}. The heights of the rectangles do not match exactly with the surface $f$, nor does the base of each rectangle follow perfectly the path of $C$.

In typical calculus fashion, our approximation can be improved by using more rectangles. The sum of the areas of these rectangles gives an approximate value of the true area above $C$ and under $f$. As the area of each rectangle is ``height $\times$ width'', we assert that the
$$\text{area above $C$}\approx \sum (\text{heights}\times\text{widths}).$$

When first learning of the integral, and approximating areas with ``heights $\times$ widths'', the width was a small change in $x$: $dx$. That will not suffice in this context. Rather, each width of a rectangle is actually approximating the arc length of a small portion of $C$. In Section \ref{sec:curvature}, we used $s$ to represent the arc--length parameter of a curve. A small amount of arc length will thus be represented by $ds$. 

The height of each rectangle will be determined in some way by the surface $f$. If we parametrize $C$ by $s$, an $s$-value corresponds to an $(x,y)$ pair that lies on the parabola $C$. Since $f$ is a function of $x$ and $y$, and $x$ and $y$ are functions of $s$, we can say that $f$ is a function of $s$. Given a value $s$, we can compute $f(s)$ and find a height. Thus
\begin{align}
\text{area under $f$ and above $C$}&\approx \sum (\text{heights}\times\text{widths});\notag\\
		\text{area under $f$ and above $C$}							&=\lim_{||\Delta s||\to0}\sum f(c_i)\Delta s_i\notag\\
									&=\int_Cf(s)\ ds.\label{eq:line0}
\end{align}

Here we have introduce a new notation, the integral symbol with a subscript of $C$. It is reminiscent of our usage of $\iint_R$. Using the train of thought found in the Integration Review preceding this section, we interpret ``$\int_C f(s)\ ds$'' as meaning ``sum up, along a curve $C$, function values $f(s)\times$small arc lengths.'' It is understood here that $s$ represents the arc--length parameter.

All this leads us to a definition. The integral found in Equation \ref{eq:line0} is called a \sword{line integral}. We formally define it below, but note that the definition is very abstract. On one hand, one is apt to say ``the defintion makes sense,'' while on the other, one is equally apt to say ``but I don't know what I'm supposed to do with this definition.'' We'll address that after the definition, and actually find an answer to the area problem we posed at the beginning of this section.

\definition{def:line_integral1}{Line Integral Over A Scalar Field}
{Let $C$ be a smooth curve parametrized by $s$, the arc--length parameter, and let $f$ be a continuous function of $s$. A \sword{line integral} is an integral of the form
$$\int_C f(s)\ ds = \lim_{||\Delta s||\to 0}\sum_{i=1}^n f(c_i)\Delta s_i,$$
where $s_1<s_2<\ldots<s_n$ is any partition of the $s$-interval over which $C$ is defined, $c_i$ is any value in the $i\,^\text{th}$ subinterval,  $\Delta s_i$ is the width of the $i\,^\text{th}$ subinterval, and $||\Delta s||$ is the length of the longest subinterval in the partition.\index{line integral!over scalar field}%
}
\mnote{.4}{\textbf{Note:} Definition \ref{def:line_integral1} uses the term \sword{scalar field} which has not yet been defined. Its meaning is discussed in the paragraph preceding Definition \ref{def:line_integral2} when it is compared to a \sword{vector field}.}




When $C$ is a \sword{closed} curve, i.e., a curve that ends at the same point at which it starts,  we use $$\oint_C f(s)\ ds \quad \text{instead of}\quad \int_Cf(s)\ ds.$$

The definition of the line integral does not specify whether $C$ is a curve in the plane or space (or hyperspace), as the definition holds regardless. For now, we'll assume $C$ lies in the $x$-$y$ plane.

This definition of the line integral  doesn't really say anything new. If $C$ is a curve and $s$ is the arc--length parameter of $C$ on $a\leq s\leq b$, then 
$$\int_Cf(s)\ ds = \int_a^bf(s)\ ds.$$
The real difference with this integral from the standard ``$\int_a^bf(x)\ dx$'' we used in the past is that of context. Our previous integrals naturally summed up values over an interval on the $x$-axis, whereas now we are summing up values over a curve. \emph{If} we can parametrize the curve with the arc--length parameter, we can evaluate the line integral just as before. Unfortunately, parametrizing a curve in terms of the arc--length parameter is usually very difficult, so we must develop a method of evaluating line integrals using a different parametrization.

%The trouble here is that we have generally avoided direct use of the arc--length parameter $s$ in the past as it is usually difficult to use. We continue that methodology here. 

Given a curve $C$, find any parametrization of $C$: $x = g(t)$ and $y=h(t)$, for continuous functions $g$ and $h$, where $a\leq t\leq b$. We can represent this parametrization with a vector--valued function, $\vrt = \langle g(t),h(t)\rangle$.

In Section \ref{sec:curvature}, we defined the arc--length parameter in Equation \ref{eq:vvfarc} as 
$$
s(t) = \int_0^t \norm{\vec r\,'(u)}\ du. 
$$
By the Fundamental Theorem of Calculus, $ds = \norm{\vec r\,'(t)}\ dt$. We can substitute the right hand side of this equation for $ds$ in the line integral definition.

We can view $f$ as being a function of $x$ and $y$ since it is a function of $s$. Thus $f(s) =f(x,y) =f\big(g(t),h(t)\big)$. This gives us a concrete way to evaluate a line integral:
$$\int_C f(s)\ ds = \int_a^bf\big(g(t),h(t)\big)\norm{\vec r\,'(t)}\ dt.$$

We restate this as a theorem, along with its three--dimensional analogue, followed by an example where we finally evaluate an integral and find an area.

\theorem{thm:line1}{Evaluating a Line Integral Over A Scalar Field}
{\begin{itemize}
	\item Let $C$ be a curve parametrized by $\vrt =\langle g(t), h(t)\rangle$, $a\leq t\leq b$, where $g$ and $h$ are continuously differentiable, and let $z=f(x,y)$, where $f$ is continuous over $C$. Then\index{line integral!over scalar field}%
	$$\int_Cf(s)\ ds = \int_a^bf\big(g(t),h(t)\big)\norm{\vec r\,'(t)}\ dt.$$
	\item Let $C$ be a curve parametrized by $\vrt =\langle g(t), h(t),k(t)\rangle$, $a\leq t\leq b$, where $g$, $h$ and $k$ are continuously differentiable, and let $w=f(x,y,z)$, where $f$ is continuous over $C$. Then
	$$\int_Cf(s)\ ds = \int_a^bf\big(g(t),h(t),k(t)\big)\norm{\vec r\,'(t)}\ dt.$$
\end{itemize}
}

%\mnote{.8}{\textbf{Notation:} We often write $f\big(\vec r(t)\big)$ to represent $f\big(g(t),h(t)\big)$ or $f\big(g(t),h(t),k(t)\big)$. As the input of $f$ are points, it is an abuse of notation to use a vector, $\vec r(t)$, as $f$\,'s input. As this notation is concise and intuitive, this notational abuse is common practice and used extensively in subsequent sections.} Removed as already explained in 14.3
To be clear, the first point of Theorem \ref{thm:line1} can be used to find the area under a surface $z=f(x,y)$ and above a curve $C$. We will later give an understanding of the line integral when $C$ is a curve in space.

Let's do an example where we actually compute an area.\\

\example{ex_linescalarfield2}{Evaluating a line integral: area under a surface over a curve.}
{Find the area under the surface $f(x,y) =\cos(x)+\sin(y)+2$ over the curve $C$, which is the segment of the line $y=2x+1$ on $-1\leq x\leq 1$, as shown in Figure \ref{fig:linescalarfield2}.
}
{Our first step is to represent $C$ with a vector--valued function. Since $C$ is a simple line, and we have a explicit relationship between $y$ and $x$ (namely, that $y$ is $2x+1$), we can let $x = t$, $y = 2t+1$, and write $\vrt = \langle t, 2t+1\rangle$ for $-1\leq t\leq 1$. 

We find the values of $f$ over $C$ as $f(x,y) = f(t,2t+1) = \cos(t)+\sin(2t+1) + 2$. We also need $\norm{\vec r\,'(t)}$; with $\vrp(t) = \langle 1,2\rangle$, we have $\norm{\vrp(t)} = \sqrt{5}$. Thus $ds = \sqrt{5}\ dt$. 

\mtable{.45}{Finding area under a curve in Example \ref{ex_linescalarfield2}.}{fig:linescalarfield2}{%
\begin{tabular}{c}
\myincludegraphicsthree{width=150pt,3Dmenu,activate=onclick,deactivate=onclick,
3Droll=0,
3Dortho=0.004519370850175619,
3Dc2c=0.7659074664115906 0.5764991044998169 0.28466543555259705,
3Dcoo=36.11199951171875 39.69871139526367 85.62228393554688,
3Droo=149.9999948024772,
3Dlights=Headlamp,add3Djscript=asylabels.js}{width=150pt}{figures/figlinescalarfield2}\\
%\myincludegraphics[scale=1.3,trim=1mm 5mm 5mm 0mm,clip]{figures/figtrip1}\\
(a)\\[10pt]
\myincludegraphicsthree{width=150pt,3Dmenu,activate=onclick,deactivate=onclick,
3Droll=0,
3Dortho=0.004519370850175619,
3Dc2c=0.7659074664115906 0.5764991044998169 0.28466543555259705,
3Dcoo=36.11199951171875 39.69871139526367 85.62228393554688,
3Droo=149.9999948024772,
3Dlights=Headlamp,add3Djscript=asylabels.js}{width=150pt}{figures/figlinescalarfield2b}\\
%\myincludegraphics[scale=1.3,trim=1mm 5mm 5mm 10mm,clip]{figures/figtrip1b}\\
(b)
\end{tabular}
}
\enlargethispage{\baselineskip}

The area we seek is 
\begin{align*}
\int_Cf(s)\ ds &= \int_{-1}^1 \big(\cos(t)+\sin(2t+1) + 2\big)\sqrt{5}\ dt \\
					&= \left.\sqrt{5}\big(\sin(t) - \frac12\cos(2t+1)+2t\big)\right|_{-1}^1\\
					&\approx 14.418\ \text{units}^2.
\end{align*}
\vskip-1.5\baselineskip
}\\

We will practice setting up and evaluating a line integral in another example, then find the area described at the beginning of this section.\\

\example{ex_linescalarfield3}{Evaluating a line integral: area under a surface over a curve.}
{Find the area over the unit circle in the $x$-$y$ plane and under the surface $f(x,y) = x^2-y^2+3$, shown in Figure \ref{fig:linescalarfield3}.
}
{The curve $C$ is the unit circle, which we will describe with the parametrization $\vrt = \langle \cos t, \sin t\rangle$ for $0\leq t\leq 2\pi$. We find $\norm{\vrp(t)} = 1$, so $ds = 1 dt$.  

\mtable{.65}{Finding area under a curve in Example \ref{ex_linescalarfield3}.}{fig:linescalarfield3}{%
\begin{tabular}{c}
\myincludegraphicsthree{width=150pt,3Dmenu,activate=onclick,deactivate=onclick,
3Droll=0,
3Dortho=0.004519370850175619,
3Dc2c=0.7659074664115906 0.5764991044998169 0.28466543555259705,
3Dcoo=36.11199951171875 39.69871139526367 85.62228393554688,
3Droo=149.9999948024772,
3Dlights=Headlamp,add3Djscript=asylabels.js}{width=150pt}{figures/figlinescalarfield3}\\
%\myincludegraphics[scale=1.3,trim=1mm 5mm 5mm 0mm,clip]{figures/figtrip1}\\
(a)\\[10pt]
\myincludegraphicsthree{width=150pt,3Dmenu,activate=onclick,deactivate=onclick,
3Droll=0,
3Dortho=0.004519370850175619,
3Dc2c=0.7659074664115906 0.5764991044998169 0.28466543555259705,
3Dcoo=36.11199951171875 39.69871139526367 85.62228393554688,
3Droo=149.9999948024772,
3Dlights=Headlamp,add3Djscript=asylabels.js}{width=150pt}{figures/figlinescalarfield3b}\\
%\myincludegraphics[scale=1.3,trim=1mm 5mm 5mm 10mm,clip]{figures/figtrip1b}\\
(b)
\end{tabular}
}
We find the values of $f$ over $C$ as $f(x,y) = f(\cos t, \sin t) = \cos^2t-\sin^2t+3$. Thus the area we seek is (note the use of the $\oint f(s) ds$ notation):
\begin{align*}
\oint_C f(s)\ ds &= \int_0^{2\pi}\big(\cos^2t-\sin^2t+3\big)\ dt \\
					&= 6\pi.
\end{align*}
(Note: we may have approximated this answer from the start. The unit circle has a circumference of $2\pi$, and we may have guessed that due to the apparent symmetry of our surface, the average height of the surface is 3.)
}\\

We now consider the example that introduced this section.\\

\example{ex_linescalarfield5}{Evaluating a line integral: area under a surface over a curve.}
{Find the area under $f(x,y) = 1-\cos(x)\sin(y)$ and over the parabola $y = x^2$, from $-1\leq x\leq 1$. }
{We parametrize our curve $C$ as $\vrt = \langle t,t^2\rangle$ for $-1\leq t\leq 1$; we find $\norm{\vrp(t)} = \sqrt{1+4t^2}$, so $ds = \sqrt{1+4t^2}\ dt$. 

Replacing $x$ and $y$ with their respective functions of $t$, we have $f(x,y) = f(t,t^2) = 1-\cos(t)\sin(t^2)$. Thus the area under $f$ and over $C$ is found to be
\begin{align*}
\int_C f(s)\ ds &= \int_{-1}^1 \Big(1-\cos(t)\sin\big(t^2\big)\Big)\sqrt{1+t^2}\ dt.\\
\intertext{This integral is impossible to evaluate using the techniques developed in this text. We resort to a numerical approximation; accurate to two places after the decimal, we find the area is}
 &= 2.17.
\end{align*}
}\\

We give one more example of finding area.\\

\example{ex_linescalarfield4}{Evaluating a line integral: area under a curve in space.}
{Find the area above the $x$-$y$ plane and below the helix parametrized by $\vrt = \langle \cos t,2\sin t,t/\pi\rangle$, for $0\leq t\leq 2\pi$, as shown in Figure \ref{fig:linescalarfield4}.}
{Note how this is problem is different than the previous examples: here, the height is not given by a surface, but by the curve itself. 

We use the given vector-valued function \vrt\ to determine the curve $C$ in the $x$-$y$ plane by simply using the first two components of \vrt: $\vec c(t) = \langle \cos t,2\sin t\rangle$. Thus $ds = \norm{\vec c\,'(t)}\,dt = \sqrt{\sin^2t + 4\cos^2t}\,dt$. 

The height is not found by evaluating a surface over $C$, but rather it is given directly by the third component of \vrt: $t/\pi$. Thus

$$\oint_C f(s)\ ds = \int_0^{2\pi} \frac{t}{\pi}\sqrt{\sin^2t + 4\cos^2t}\,dt \approx 9.69,$$
where the approximation was obtained using numerical methods.
\mfigurethree{width=150pt,3Dmenu,activate=onclick,deactivate=onclick,
3Droll=0,
3Dortho=0.004994507879018784,
3Dc2c=0.6830801963806152 0.5863659381866455 0.43540385365486145,
3Dcoo=1.7807315587997437 0.9528753757476807 54.982017517089844,
3Droo=141.8977232410902,
3Dlights=Headlamp,add3Djscript=asylabels.js}{width=150pt}{.77}{Finding area under a curve in Example \ref{ex_linescalarfield4}.}{fig:linescalarfield4}{figures/figlinescalarfield4}
}\\

Note how in each of the previous examples we are effectively finding ``area under a curve'', just as we did when first learning of integration. We have used the phrase ``area \emph{over} a curve $C$ and under a surface,'' but that is because of the important role $C$ plays in the integral. The figures show how the curve $C$ defines another curve on the surface $z=f(x,y)$, and we are finding the area under that curve.

\vskip \baselineskip
\noindent\textbf{\large Properties of Line Integrals}\\

Many properties of line integrals can be inferred from general integration properties. For instance, if $k$ is a scalar, then $\int_C k\,f(s)ds = k\int_Cf(s)ds$.

One property in particular of line integrals is worth noting. If $C$ is a curve composed of subcurves $C_1$ and $C_2$, where they share only one point in common (see Figure \ref{fig:line_int_prop}(a)), then the line integral over $C$ is the sum of the line integrals over $C_1$ and $C_2$: 
$$\int_Cf(s)\ ds = \int_{C_1}f(s)\ ds+\int_{C_2}f(s)\ ds.$$
\mtable{.4}{Illustrating properties of line integrals.}{fig:line_int_prop}{
\begin{tabular}{c}
\myincludegraphics[scale=.9]{figures/figline_integral_curve_a}\\
(a)\\[10pt]
\myincludegraphics[scale=.9]{figures/figline_integral_curve_b}\\
(b)
\end{tabular}
}

This property allows us to evaluate line integrals over some curves $C$ that are not smooth. Note how in Figure \ref{fig:line_int_prop}(b) the curve is not smooth at $D$, so by our definition of the line integral we cannot evaluate $\int_C f(s)ds$. However, one can evaluate line integrals over $C_1$ and $C_2$ and their sum will be the desired quantity.

A curve $C$ that is composed of two or more smooth curves is said to be \sword{piecewise smooth}. In this chapter, any statement that is made about smooth curves also holds for piecewise smooth curves.\index{smooth curve!piecewise}\index{piecewise smooth curve}

%The second property of note is that the orientation of the parametrization of $C$ does not affect the value of the line integral (\textit{over scalar fields}. Later, we'll see cases where orientation does matter.) In Figure \ref{fig:line_int_prop}, the arrows indicating orientation are extraneous. In Section \ref{sec:line_int_vf}, where we study line integrals \emph{over vector fields}, the orientation of the parametrization does make a difference.

We state these properties as a theorem.

\theorem{thm:line_int_properties_scalar}{Properties of Line Integrals Over Scalar Fields}
{
\begin{enumerate}
	\item	Let $C$ be a smooth curve parametrized by the arc--length parameter $s$, let $f$ and $g$ be continuous functions of $s$, and let $k_1$ and $k_2$ be scalars. Then \index{line integral!properties over a scalar field}%
$$\ds \int_C\big(k_1f(s)+k_2g(s)\big)\ ds = k_1\int_Cf(s)\ ds +k_2\int_Cg(s)\ ds.$$
	\item Let $C$ be piecewise smooth, composed of smooth components $C_1$ and $C_2$. Then
	$$\int_Cf(s)\ ds = \int_{C_1}f(s)\ ds + \int_{C_2}f(s)\ ds.$$
	\end{enumerate}%
	}

\vskip \baselineskip
\noindent\textbf{\large Mass and Center of Mass}\\


We first learned integration as a method to find area under a curve, then later used integration to compute a variety of other quantities, such as arc length, volume, force, etc. In this section, we also introduced line integrals as a method to find area under a curve, and now we explore one more application.

Let a curve $C$ (either in the plane or in space) represent a thin wire with variable density $\delta(s)$. We can approximate the mass of the wire by dividing the wire (i.e., the curve) into small segments of length $\Delta s_i$ and assume the density is constant across these small segments. The mass of each segment is density of the segment $\times$ its length; by summing up the approximate mass of each segment we can approximate the total mass:
$$\text{Total Mass of Wire } = \sum \delta(s_i)\Delta s_i.$$

By taking the limit as the length of the segments approaches 0, we have the definition of the line integral as seen in Definition \ref{def:line_integral1}. When learning of the line integral, we let $f(s)$ represent a height; now we let $f(s) = \delta(s)$ represent a density.

We can extend this understanding of computing mass to also compute the center of mass of a thin wire. (As a reminder, the center of mass can be a useful piece of information as objects rotate about that center.) We give the relevant formulas in the next definition, followed by an example. Note the similarities between this definition and Definition \ref{def:mass_3d}, which gives similar properties of solids in space.

\definition{def:mass_of_thin_wire}{Mass, Center of Mass of Thin Wire}
{Let a thin wire lie along a smooth curve $C$ with continuous density function $\delta(s)$, where $s$ is the arc length parameter. 	\index{mass!center of}\index{mass}%
\begin{enumerate}
	\item The \textbf{mass} of the thin wire is $\ds M = \int_C \delta(s)\ ds$.
	\item	The \textbf{moment about the $y$-$z$ plane} is $\ds M_{yz} = \int_C x\delta(s)\ ds$.
	
	\item	The \textbf{moment about the $x$-$z$ plane} is $\ds M_{xz} = \int_C y\delta(s)\ ds$.
	\item	The \textbf{moment about the $x$-$y$ plane} is $\ds M_{xy} = \int_C z\delta(s)\ ds$.
	\item The \textbf{center of mass} of the wire is $$(\overline{x},\overline{y},\overline{z}) = \left(\frac{M_{yz}}M, \frac{M_{xz}}M,\frac{M_{xy}}M\right).$$
\end{enumerate}
}

\example{ex_linescalarfield6}{Evaluating a line integral: calculating mass.}
{A thin wire follows the path $\vrt = \langle 1+\cos t,1+\sin t, 1+ \sin(2t)\rangle$, $0\leq t\leq 2\pi$. The density of the wire is determined by its position in space: $\delta(x,y,z) = y+z$ gm/cm. The wire is shown in Figure \ref{fig:linescalarfield6}, where a light color indicates low density and a dark color represents high density. Find the mass  and center of mass of the wire.
\mfigurethree{width=150pt,3Dmenu,activate=onclick,deactivate=onclick,
3Droll=0,
3Dortho=0.0045836810022592545,
3Dc2c=0.8566989302635193 -0.4622343182563782 0.22892449796199799,
3Dcoo=85.21116638183594 55.206966400146484 50.90880584716797,
3Droo=141.89772711774134,
3Dlights=Headlamp,add3Djscript=asylabels.js}{width=150pt}{.4}{Finding the mass of a thin wire in Example \ref{ex_linescalarfield6}.}{fig:linescalarfield6}{figures/figlinescalarfield6}}
{We compute the density of the wire as 
$$\delta(x,y,z) = \delta\big(1+\cos t,1+\sin t, 1+\sin(2t)\big) = 2+\sin t+\sin(2t).$$ We compute $ds$ as
$$ds = \norm{\vrp(t)}\ dt = \sqrt{\sin^2t+\cos^2t+4\cos^2(2t)}\ dt = \sqrt{1+4\cos^2(2t)}\ dt.$$
Thus the mass is
$$M = \oint_C \delta(s)\ ds = \int_0^{2\pi} \big(2+\sin t+\sin(2t)\big)\sqrt{1+4\cos^2(2t)}\ dt \approx 21.08\text{gm}. $$
We compute the moments about the coordinate planes:\small
\begin{align*}
M_{yz} &= \oint_C x\delta(s)\ ds = \int_0^{2\pi}(1+\cos t)\big(2+\sin t+\sin(2t)\big)\sqrt{1+4\cos^2(2t)}\ dt \approx 21.08. \\
M_{xz} &= \oint_C y\delta(s)\ ds = \int_0^{2\pi}(1+\sin t)\big(2+\sin t+\sin(2t)\big)\sqrt{1+4\cos^2(2t)}\ dt \approx
26.35\\
M_{xy} &= \oint_C z\delta(s)\ ds = \int_0^{2\pi}\big(1+\sin(2 t)\big)\big(2+\sin t+\sin(2t)\big)\sqrt{1+4\cos^2(2t)}\ dt \approx 25.40
\end{align*}\normalsize
Thus the center of mass of the wire is located at 
$$(\overline{x},\overline{y},\overline{z}) = \left(\frac{M_{yz}}M, \frac{M_{xz}}M,\frac{M_{xy}}M\right) \approx (1,1.25,1.20),$$
as indicated by the dot in Figure \ref{fig:linescalarfield6}. Note how in this example, the curve $C$ is ``centered'' about the point $(1,1,1)$, though the variable density of the wire pulls the center of mass out along the $y$ and $z$ axes.
}\\

We end this section with a callback to the Integration Review that preceded this section. A line integral looks like: $\int_C f(s)\ ds$. As stated before the definition of the line integral, this means ``sum up, along a curve $C$, function values $f(s)$ $\times$ small arc lengths.'' When $f(s)$ represents a height, we have ``height $\times$ length = area.'' When $f(s)$ is a density (and we use $\delta(s)$ by convention), we have ``density (mass per unit length) $\times$ length = mass.''

In the next section, we investigate a new mathematical object, the \emph{vector field}. The remaining sections of this chapter are devoted to understanding integration in the context of vector fields.

%We'll expand our uses for line integrals after the next section. First, we'll investigate a new mathematical object, called a vector field.

\printexercises{exercises/14_01_exercises}
