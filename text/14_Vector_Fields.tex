\section{Vector Fields}\label{sec:vector_fields}

We have studied functions of two or three variables, where the input of such functions is a point (either a point in the plane or in space) and the output is a number.

We could also create functions where the input is a point (again, either in the plane or in space), but the output is a {\em vector}. For instance, we could create the following function: $\vec F(x,y) = \langle x+y, x-y\rangle$, where $\vec F(2,3) = \langle 5,-1\rangle$. We are to think of $\vec F$ assigning the vector $\langle 5,-1\rangle$ to the point $(2,3)$; in some sense, the vector $\langle 5,-1\rangle$ lies at the point $(2,3)$. 

Such functions are extremely useful in any context where magnitude and direction are important. For instance, we could create a function $\vec F$ that represents the electromagnetic force exerted at a point by a electromagnetic field, or the velocity of air as it moves across an airfoil. 

Because these functions are so important, we need to formally define them.

\definition{def:vector_field}{Vector Field}
{\begin{enumerate}
	\item A \textbf{vector field in the plane} is a function $\vec F(x,y)$ whose domain is a subset of $\mathbb{R}^2$ and whose output is a two--dimensional vector:
	$$\vec F(x,y) = \langle M(x,y), N(x,y)\rangle.$$
	
	\item A \textbf{vector field in space} is a function $\vec F(x,y,z)$ whose domain is a subset of $\mathbb{R}^3$ and whose output is a three--dimensional vector:
	$$\vec F(x,y,z) = \langle M(x,y,z), N(x,y,z), P(x,y,z)\rangle.$$
\end{enumerate}
}

This definition may seem odd at first, as a special type of function is called a ``field.'' However, as the function determines a ``field of vectors'', we can say the field is \emph{defined by} the function, and thus the field \emph{is} a function.

\mtable{.4}{Demonstrating methods of graphing vector fields.}{fig:vectorfieldintro}{
\begin{tabular}{c}
\myincludegraphics{figures/figvectorfieldintro_a}\\[-5pt]
(a)\\[10pt]
\myincludegraphics{figures/figvectorfieldintro_b}\\[-5pt]
(b)
\end{tabular}
}

Visualizing vector fields helps cement this connection. When graphing a vector field in the plane, the general idea is to draw the vector $\vec F(x,y)$ at the point $(x,y)$. For instance, using $\vec F(x,y) = \langle x+y,x-y\rangle$ as before, at $(1,1)$ we would draw $\langle 2,0\rangle$. 

In Figure \ref{fig:vectorfieldintro}(a), one can see that the vector $\langle 2,0\rangle$ is drawn \emph{starting from} the point $(1,1)$. A total of 8 vectors are drawn, with the $x$- and $y$-values of $-1,0,1$. In many ways, the resulting graph is a mess; it is hard to tell what this field ``looks like.''

In Figure \ref{fig:vectorfieldintro}(b), the same field is redrawn with each vector $\vec F(x,y)$ drawn \emph{centered on} the point $(x,y)$. This makes for a better looking image, though the long vectors can cause confusion: when one vector intersects another, the image looks cluttered.

A common way to address this problem is limit the length of each arrow, and represent long vectors with thick arrows, as done in Figure \ref{fig:vectorfieldintrob}(a). Usually we do not use a graph of a vector field to determine exactly the magnitude of a particular vector. Rather, we are more concerned with the relative magnitudes of vectors: which are bigger than others? Thus limiting the length of the vectors is not problematic.

\mtable{.687}{Demonstrating methods of graphing vector fields.}{fig:vectorfieldintrob}{
\begin{tabular}{c}
\myincludegraphics{figures/figvectorfieldintro_c}\\[-5pt]
(a)\\[10pt]
\myincludegraphics{figures/figvectorfieldintro_d}\\[-5pt]
(b)
\end{tabular}
}

Drawing arrows with variable thickness is best done with technology; search the documentation of your favorite graphing program for terms like ``vector fields'' or ``slope fields'' to learn how. Technology obviously allows us to plot many vectors in a vector field nicely; in Figure \ref{fig:vectorfieldintrob}(b), we see the same vector field drawn with many vectors, and finally get a clear picture of how this vector field behaves. (If this vector field represented the velocity of air moving across a flat surface, we could see that the air tends to move either to the upper--right or lower--left, and moves very slowly near the origin.)

We can similarly plot vector fields in space, as shown in Figure \ref{fig:vectorfieldintroc}, though it is not often done. The plots get very busy very quickly, as there are lots of arrows drawn in a small amount of space. In Figure \ref{fig:vectorfieldintroc} the field $\vec F = \langle -y,x,z\rangle$ is graphed. If one could view the graph from above, one could see the arrows point in a cirlce about the $z$-axis. One should also note how the arrows far from the origin are larger than those close to the origin. 

It is good practice to try to visualize certain vector fields in one's head. For instance, consider a point mass at the origin and the vector field that represents the gravitational force exerted by the mass at any point in the room. The field would consist of arrows pointing toward the origin, increasing in size as they near the origin (as the gravitational pull is strongest near the point mass).

\mfigurethree{width=150pt,3Dmenu,activate=onclick,deactivate=onclick,
3Droll=0,
3Dortho=0.004676746670156717,
3Dc2c=0.7426115870475769 -0.4871146082878113 0.45961663126945496,
3Dcoo=40.4605598449707 -31.11708641052246 21.908815383911133,
3Droo=141.89772583942957,
3Dlights=Headlamp,add3Djscript=asylabels.js}{}{.29}{Graphing a vector field in space.}{fig:vectorfieldintroc}{figures/figvectorfieldintro_f}

\vskip \baselineskip
\noindent\textbf{\large Vector Field Notation and Del Operator}\\

Definition \ref{def:vector_field} defines a vector field $\vec F$ using the notation $$\vec F(x,y) = \langle M(x,y), N(x,y)\rangle\quad \text{and}\quad \vec F(x,y,z) = \langle M(x,y,z), N(x,y,z),P(x,y,z)\rangle.$$ That is, the components of $\vec F$ are each functions of $x$ and $y$ (and also $z$ in space). As done in other contexts, we will drop the ``of $x$, $y$ and $z$'' portions of the notation and refer to vector fields in the plane and in space as 
$$\vec F = \langle M, N\rangle \quad \text{and} \quad \vec F  = \langle M,N,P\rangle,$$ respectively, as this shorthand is quite convenient.

Another item of notation will become useful: the ``del operator.'' Recall in Section \ref{sec:directional_derivative} how we used the symbol $\nabla$ (pronounced ``del'') to represent the gradient of a function of two variables. That is, if $z = f(x,y)$, then ``del $f$\,'' $= \nabla f = \langle f_x, f_y\rangle$. 

We now define $\nabla$ to be the ``del operator.'' It is a vector whose components are partial derivative operations. 

In the plane, $\ds\nabla = \left\langle \frac{\partial}{\partial x}, \frac{\partial}{\partial y}\right\rangle$; in space, $\ds\nabla = \left\langle \frac{\partial}{\partial x}, \frac{\partial}{\partial y},\frac{\partial}{\partial z}\right\rangle$.

Let $\vec F = \langle x+\sin y,y^2+z,x^2\rangle$. We can use vector operations and find the dot product of $\nabla$ and $\vec F$:
\begin{align*}
\nabla \cdot \vec F &= \left\langle \frac{\partial}{\partial x}, \frac{\partial}{\partial y},\frac{\partial}{\partial z}\right\rangle\cdot  \langle x+\sin y,y^2+z,x^2\rangle \\
  &= \frac{\partial}{\partial x}(x+\sin y)+ \frac{\partial}{\partial y}(y^2+z) + \frac{\partial}{\partial z}(x^2) \\
		&=1+2y.
\end{align*}

We can also compute their cross products:
\begin{align*}
\nabla\times \vec F &= \left\langle \frac{\partial}{\partial y}\big(x^2\big)-\frac{\partial}{\partial z}\big(y^2+z\big),\frac{\partial}{\partial z}\big(x+\sin y\big)-\frac{\partial}{\partial x}\big(x^2\big),\frac{\partial}{\partial x}\big(y^2+z\big)-\frac{\partial}{\partial y}\big(x+\sin y\big)\right\rangle \\
			&=\langle -1,-2x,-\cos y\rangle.
\end{align*}

We do not yet know why we would want to compute the above. However, as we next learn about properties of vector fields, we will see how these dot and cross products with the del operator are quite useful.

\vskip \baselineskip
\noindent\textbf{\large Divergence and Curl}\\

Two properties of vector fields will prove themselves to be very important: divergence and curl. Each is a special ``derivative'' of a vector field; that is, each measures an instantaneous rate of change of a vector field.

If the vector field represents the velocity of a fluid or gas, then the \sword{divergence} of the field is a measure of the ``compressibility'' of the fluid. If the divergence is negative at a point, it means that the fluid is compressing: more fluid is going into the point than is going out. If the divergence is positive, is means the fluid is expanding: more fluid is going out at that point than going in. A divergence of zero means the same amount of fluid is going in as is going out. If the divergence is zero at all points, we say the field is \sword{incompressible}. 

It turns out that the proper measure of divergence is simply $\nabla \cdot \vec F$, as stated in the following definition.

\definition{def:divergence}{Divergence of a Vector Field}
{The \sword{divergence} of a vector field $\vec F$ is
$$\divv \vec F = \nabla \cdot \vec F.$$
\begin{itemize}
	\item In the plane, with $\vec F = \langle M,N\rangle$, $\divv \vec F = M_x+N_y$.
	\item	In space, with $\vec F = \langle M,N,P\rangle$, $\divv \vec F = M_x+N_y+P_z$.
\end{itemize}
}

%Divergence is a measure of the \emph{compressibility} of the ``substance'' the field represents. Imagine drawing a small box anywhere on the graph of the vector field in Figure \ref{fig:vectorfieldintrob}(b). There would be some vectors pointing into the box, and some arrows pointing out of the box. In short, divergence measures ``the amount going out minus the amount going in.'' If the vector field represented the flow of air being pumped into an enclosed room, then there is likely more air going into a region than going out. Via a limit process, we can shrink the box to a point, and divergence measures the amount going out of the point vs. the amount going into the point. If the divergence of a field is zero everywhere, we say that the field represents an \sword{incompressible fluid} -- that is, the same amount goes into each point as goes out. (It turns out that the divergence of the field in Figure \ref{fig:vectorfieldintrob} is zero everywhere.)

Curl is a measure of the spinning action of the field. Let $\vec F$ represent the flow of water over a flat surface. If a small round cork were held in place at a point in the water, would the water cause the cork to spin? No spin corresponds to zero curl; counterclockwise spin corresponds to positive curl and clockwise spin corresponds to negative curl. 

In space, things are a bit more complicated. Again let $\vec F$ represent the flow of water, and imagine suspending a tennis ball in one location in this flow. The water may cause the ball to spin along an axis. If so, the curl of the vector field is a \emph{vector} (not a \emph{scalar}, as before), parallel to the axis of rotation, following a right hand rule: when the thumb of one's right hand points in the direction of the curl, the ball will spin in the direction of the curling fingers of the hand.

In space, it turns out the proper measure of curl is $\nabla \times \vec F$, as stated in the following definition. To find the curl of a planar vector field $\vec F = \langle M,N\rangle$, embed it into space as $\vec F = \langle M, N, 0\rangle$ and apply the cross product definition. Since $M$ and $N$ are functions of just $x$ and $y$ (and not $z$), all partial derivatives with respect to $z$ become 0 and the result is simply $\langle 0,0,N_x-M_y\rangle$. The third component is the measure of curl. 

\definition{def:curl}{Curl of a Vector Field}
{\begin{itemize}
	\item Let $\vec F = \langle M,N\rangle$ be a vector field in the plane. The \sword{curl} of $\vec F$ is $\curl \vec F = N_x - M_y$.
	\item Let $\vec F = \langle M,N,P\rangle$ be a vector field in space. The \sword{curl} of $\vec F$ is $\curl \vec F = \nabla \times \vec F = \langle P_y-N_z,M_z-P_x,N_x - M_y\rangle$.
\end{itemize}
}

We adopt the convention of referring to curl as $\nabla \times \vec F$, regardless of whether $\vec F$ is a vector field in two or three dimensions. 

We now practice computing these quantities.\\

\example{ex_vectorfield1}{Computing divergence and curl of planar vector fields}
{For each of the planar vector fields given below, view its graph and try to visually determine if its divergence and curl are 0. Then compute the divergence and curl.

%\noindent\begin{minipage}[t]{.48\linewidth}
%\begin{enumerate}
	%\item $\vec F = \langle y,0\rangle$ (see Figure \ref{fig:vectorfield1a}(a))
	%\item $\vec F = \langle -y,x\rangle$ (see Figure \ref{fig:vectorfield1a}(b))
%\end{enumerate}
%\end{minipage}\begin{minipage}[t]{.6\linewidth}
%\begin{enumerate}\addtocounter{enumi}{2}
	%\item $\vec F = \langle x,y\rangle$ (see Figure \ref{fig:vectorfield1b}(a))
	%\item $\vec F = \langle \cos y, \sin x\rangle$ 
		%(see Figure \ref{fig:vectorfield1b}(b))
%\end{enumerate}
%\end{minipage}
\begin{enumerate}
	\item $\vec F = \langle y,0\rangle$ (see Figure \ref{fig:vectorfield1a}(a))
	\item $\vec F = \langle -y,x\rangle$ (see Figure \ref{fig:vectorfield1a}(b))
	\item $\vec F = \langle x,y\rangle$ (see Figure \ref{fig:vectorfield1b}(a))
	\item $\vec F = \langle \cos y, \sin x\rangle$ (see Figure \ref{fig:vectorfield1b}(b))
\end{enumerate}
}
{\begin{enumerate}
	\item The arrow sizes are constant along any horizontal line, so if one were to draw a small box anywhere on the graph, it would seem that the same amount of fluid would enter the box as exit. Therefore it seems the divergence is zero; it is, as 
	$$\divv\vec F = \nabla \cdot \vec F = M_x + N_y = \frac{\partial}{\partial x}(y) + \frac{\partial}{\partial y}(0) = 0.$$
	%\mfigure{.35}{The vector field in part (a) of Example \ref{ex_vectorfield1}.}{fig:vectorfield1a}{figures/figvectorfield1_a}
	\mtable{.5}{The vector fields in parts (a) and (b) in Example \ref{ex_vectorfield1}.}{fig:vectorfield1a}{
	\begin{tabular}{c}
	\myincludegraphics{figures/figvectorfield1_a}\\[-5pt]
	(a)\\[10pt]
	\myincludegraphics{figures/figvectorfield1_b}\\[-5pt]
	(b)
	\end{tabular}
	}
	At any point on the $x$-axis, arrows above it move to the right and arrows below it move to the left, indicating that a cork placed on the axis would spin clockwise. A cork placed anywhere above the $x$-axis would have water above it moving to the right faster than the water below it, also creating a clockwise spin. A clockwise spin also appears to be created at points below the $x$-axis. Thus it seems the curl should be negative (and not zero). Indeed, it is:
	$$\curl \vec F = \nabla\times\vec F = N_x-M_y = \frac{\partial}{\partial x}(0) - \frac{\partial}{\partial y}(y) = -1.$$
	
	\item It appears that all vectors that lie on a circle of radius $r$, centered at the  origin, have the same length (and indeed this is true). That implies that the divergence should be zero: draw any box on the graph, and any fluid coming in will lie along a circle that takes the same amount of fluid out. Indeed, the divergence is zero, as
	$$\divv\vec F = \nabla \cdot \vec F = M_x + N_y = \frac{\partial}{\partial x}(-y) + \frac{\partial}{\partial y}(x) = 0.$$
	
		Clearly this field moves objects in a circle, but would it induce a cork to spin? It appears that yes, it would: place a cork anywhere in the flow, and the point of the cork closest to the origin would feel less flow than the point on the cork farthest from the origin, which would induce a counterclockwise flow. Indeed, the curl is positive:
	$$\curl \vec F = \nabla\times\vec F = N_x-M_y = \frac{\partial}{\partial x}(x) - \frac{\partial}{\partial y}(-y) = 1-(-1) = 2.$$
	Since the curl is constant, we conclude the induced spin is the same no matter where one is in this field.
	
	\item At the origin, there are many arrows pointing out but no arrows pointing in. We conclude that at the origin, the divergence must be positive (and not zero). If one were to draw a box anywhere in the field, the edges farther from the origin would have larger arrows passing through them than the edges close to the origin, indicating that more is going from a point than going in. This indicates a positive (and not zero) divergence. This is correct:
	$$\divv\vec F = \nabla \cdot \vec F = M_x + N_y = \frac{\partial}{\partial x}(x) + \frac{\partial}{\partial y}(y) = 1+1=2.$$
	
	One may find this curl to be harder to determine visually than previous examples. One might note that any arrow that induces a clockwise spin on a cork will have an equally sized arrow inducing a counterclockwise spin on the other side, indicating no spin and no curl. This is correct, as
	$$\curl \vec F = \nabla\times\vec F = N_x-M_y = \frac{\partial}{\partial x}(y) - \frac{\partial}{\partial y}(x) = 0.$$
	
	%\drawexampleline
	\mtable{.6}{The vector fields in parts (c) and (d) in Example \ref{ex_vectorfield1}.}{fig:vectorfield1b}{
	\begin{tabular}{c}
	\myincludegraphics{figures/figvectorfield1_c}\\[-5pt]
	(a)\\[10pt]
	\myincludegraphics{figures/figvectorfield1_d}\\[-5pt]
	(b)
	\end{tabular}
	}
	\item	One might find this divergence hard to determine visually as large arrows appear in close proximity to small arrows, each pointing in different directions. Instead of trying to rationalize a guess, we compute the divergence:
	$$\divv\vec F = \nabla \cdot \vec F = M_x + N_y = \frac{\partial}{\partial x}(\cos y) + \frac{\partial}{\partial y}(\sin x) = 0.$$ 
	Perhaps surprisingly, the divergence is 0.
	
	Will all the loops of different directions in the field, one is apt to reason the curl is variable. Indeed, it is:
	$$\curl \vec F = \nabla\times\vec F = N_x-M_y = \frac{\partial}{\partial x}(\sin x) - \frac{\partial}{\partial y}(\cos y) = \cos x + \sin y.$$
	Depending on the values of $x$ and $y$, the curl may be positive, negative, or zero.
\end{enumerate}
\vskip-1.5\baselineskip
}\clearpage

\example{ex_vectorfield2}{Computing divergence and curl of vector fields in space}
{Compute the divergence and curl of each of the following vector fields.
\begin{enumerate}
	\item $\vec F = \langle x^2+y+z, -x-z, x+y\rangle$
	\item	$\vec F = \langle e^{xy}, \sin(x+z),x^2+y\rangle$
\end{enumerate}
}
{We compute the divergence and curl of each field following the definitions.
\begin{enumerate}
	\item $\divv \vec F = \nabla \cdot \vec F = M_x+N_y+P_z = 2x+0+0= 2x.$
	
	$\begin{aligned}\curl\vec F = \nabla \times \vec F &= \langle P_y-N_z,M_z-P_x,N_x - M_y\rangle\\ &= \langle 1 - (-1), 1-1,-1-(1)\rangle = \langle 2,0,-2\rangle.
	\end{aligned}$
	
	For this particular field, no matter the location in space, a spin is induced with axis parallel to $\langle 2,0,-2\rangle.$
	\item $\divv \vec F = \nabla \cdot \vec F = M_x+N_y+P_z = ye^{xy}+0+0= ye^{xy}.$
		
	$\begin{aligned}\curl\vec F = \nabla \times \vec F &= \langle P_y-N_z,M_z-P_x,N_x - M_y\rangle\\ &= \langle 1-\cos(x+z), -2x, \cos(x+z) - xe^{xy}\rangle. \end{aligned}$
\end{enumerate}
\vskip-1\baselineskip
}
\\

\example{ex_vectorfield3}{Creating a field representing gravitational force}
{The force of gravity between two objects is inversely proportional to the square of the distance between the objects. Locate a point mass at the origin. Create a vector field $\vec F$ that represents the gravitational pull of the point mass at any point $(x,y,z)$. Find the divergence and curl of this field. 
}
{The point mass pulls toward the origin, so at $(x,y,z)$, the force will pull in the direction of $\langle -x, -y, -z\rangle$. To get the proper magnitude, it will be useful to find the unit vector in this direction. Dividing by its magnitude, we have $$\vec u = \left\langle \frac{-x}{\sqrt{x^2+y^2+z^2}}, \frac{-y}{\sqrt{x^2+y^2+z^2}},\frac{-z}{\sqrt{x^2+y^2+z^2}}\right\rangle.$$
The magnitude of the force is inversely proportional to the square of the distance between the two points. Letting $k$ be the constant of proportionality, we have the magnitude as $\ds\frac{k}{x^2+y^2+z^2}$. Multiplying this magnitude by the unit vector above, we have the desired vector field:
$$\vec F = \left\langle \frac{-kx}{(x^2+y^2+z^2)^{3/2}}, \frac{-ky}{(x^2+y^2+z^2)^{3/2}},\frac{-kz}{(x^2+y^2+z^2)^{3/2}}\right\rangle.$$
We leave it to the reader to confirm that $\divv \vec F = 0$ and $\curl \vec F = \vec 0$.

The analogous planar vector field is given in Figure \ref{fig:vectorfield3}. Note how all arrows point to the origin, and the magnitude gets very small when ``far'' from the origin.
}\\

\mfigure{.78}{A vector field representing a planar gravitational force.}{fig:vectorfield3}{figures/figvectorfield3}
A function $z=f(x,y)$ naturally induces a vector field, $\vec F = \nabla f = \langle f_x,f_y\rangle$. Given what we learned of the gradient in Section \ref{sec:directional_derivative}, we know that the vectors of $\vec F$ point in the direction of greatest increase of $f$. Because of this, $f$ is said to be the \sword{potential function} of $\vec F$. Vector fields that are the gradient of potential functions will play an important role in the next section.\\

\example{ex_vectorfield4}{A vector field that is the gradient of a potential function}
{Let $f(x,y) = 3-x^2-2y^2$ and let $\vec F = \nabla f$. Graph $\vec F$, and find the divergence and curl of $\vec F$. 
}
{Given $f$, we find $\vec F = \nabla f = \langle -2x,-4y\rangle$. A graph of $\vec F$ is given in Figure \ref{fig:vectorfield4}(a). In part (b) of the figure, the vector field is given along with a graph of the surface itself; one can see how each vector is pointing in the direction of ``steepest uphill'', which, in this case, is not simply just ``toward the origin.''

\mtable{.4}{A graph of a function $z=f(x,y)$ and the vector field $\vec F = \nabla f$ in Example \ref{ex_vectorfield4}.}{fig:vectorfield4}
{\begin{tabular}{c}
\myincludegraphics{figures/figvectorfield4}\\[-5pt]
(a)\\[10pt]
\myincludegraphicsthree{width=150pt,3Dmenu,activate=onclick,deactivate=onclick,
3Droll=0,
3Dortho=0.004781691357493401,
3Dc2c=0.7686017751693726 -0.47064894437789917 0.4332907199859619,
3Dcoo=0.000002273424570375937 -0.0000013264116205391474 56.08348083496094,
3Droo=117.22547128132403,
3Dlights=Headlamp,add3Djscript=asylabels.js}{}{figures/figvectorfield4b}\\
(b)
\end{tabular}
}
We leave it to the reader to confirm that $\divv \vec F = -6$ and $\curl \vec F = 0$.
}\\

There are some important concepts visited in this section that will be revisited in subsequent sections and again at the very end of this chapter. One is: given a vector field $\vec F$, both $\divv\vec F$ and $\curl\vec F$ are measures of rates of change of $\vec F$. The divergence measures how much the field spreads (diverges) at a point, and the curl measures how much the field twists (curls) at a point. Another important concept is this: given $z=f(x,y)$, the gradient $\nabla f$ is also a measure of a rate of change of $f$. We will see how the integrals of these rates of change produce meaningful results.

This section introduces the concept of a vector field. The next section ``applies calculus'' to vector fields. A common application is this: let $\vec F$ be a vector field representing a force (hence it is called a ``force field,'' though this name has a decidedly comic-book feel) and let a particle move along a curve $C$ under the influence of this force. What work is performed by the field on this particle? The solution lies in correctly applying the concepts of line integrals in the context of vector fields.


\printexercises{exercises/14_02_exercises}
%Consider the same field as before, letting it represent the flow of water across a flat surface. If one were to drop a small cork into the water, it would certainly move according to the pattern of the field. 
