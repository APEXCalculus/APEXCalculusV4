\thispagestyle{empty}
\Huge
\noindent {\bf \textsc{Preface}}\\
\large
\emph{A Note on Using this Text}
\vskip 2\baselineskip
\normalsize

Thank you for reading this short preface. Allow us to share a few key points about the text so that you may better understand what you will find beyond this page.

This text is Part III of a three--text series on Calculus. The first part covers material taught in many ``Calc 1'' courses: limits, derivatives, and the basics of integration, found in Chapters 1 through 6.1. The second text covers material often taught in ``Calc 2:'' integration and its applications, along with an introduction to sequences, series and Taylor Polynomials, found in Chapters 5 through 8. The third text covers topics common in ``Calc 3'' or ``multivariable calc:'' parametric equations, polar coordinates, vector--valued functions, and functions of more than one variable, found in Chapters 9 through 13. All three are available separately for free at \texttt{\href{http://www.vmi.edu/APEX}{www.vmi.edu/APEX}}. These three texts are intended to work together and make one cohesive text, \textit{APEX Calculus}, which can also be downloaded from the website. 

Printing the entire text as one volume makes for a large, heavy, cumbersome book. One can certainly only print the pages they currently need, but some prefer to have a nice, bound copy of the text. Therefore this text has been split into these three manageable parts, each of which can be purchased for under \$15 at \href{http://amazon.com}{Amazon.com}. 

A result of this splitting is that sometimes a concept is said to be explored in an ``earlier section,'' though that section does not actually appear in this particular text. Also, the index makes reference to topics, and page numbers, that do not appear in this text. This is done intentionally to show the reader what topics are available for study.  Downloading the .pdf of \textit{APEX Calculus} will ensure that you have all the content.\\ 
%material is referenced that is not contained in the present text. The context should make it clear whether the ``missing'' material is in the \textit{Calculus I} or \textit{Calculus III} portion. Downloading the appropriate .pdf, or the whole \textit{APEX Calculus} .pdf, will give access to these topics.
  %For instance, in this text, ``Theorem 20'' may be mentioned, although Theorem 20 is only presented in Part I. To minimize confusion, theorems, definitions and key ideas are referenced by their title or subject matter, not their number.

%The current publisher of this text does not allow one text to be split across multiple volumes, with continuity of chapters and page numberings. This is the one drawback of the current publishing model that has many advantages, highlighted below. Because of this, there are a few peculiarities 

\noindent\textbf{\large For Students: How to Read this Text}\\

Mathematics textbooks have a reputation for being hard to read. High--level mathematical writing often seeks to say much with few words, and this style often seeps into texts of lower--level topics. This book was written with the goal of being easier to read than many other calculus textbooks, without becoming too verbose. 

Each chapter and section starts with an introduction of the coming material, hopefully setting the stage for ``why you should care,'' and ends with a look ahead to see how the just--learned material helps address future problems. 

\textit{Please read the text;} it is written to explain the concepts of Calculus. There are numerous examples to demonstrate the meaning of definitions, the truth of theorems, and the application of mathematical techniques. When you encounter a sentence you don't understand, read it again. If it still doesn't make sense, read on anyway, as sometimes confusing sentences are explained by later sentences.

\textit{You don't have to read every equation.} The examples generally show ``all'' the steps needed to solve a problem. Sometimes reading through each step is helpful; sometimes it is confusing. When the steps are illustrating a new technique, one probably should follow each step closely to learn the new technique. When the steps are showing the mathematics needed to find a number to be used later, one can usually skip ahead and see how that number is being used, instead of getting bogged down in reading how the number was found.

\textit{Most proofs have been omitted.} In mathematics, \textit{proving} something is always true is extremely important, and entails much more than testing to see if it works twice. However, students often are confused by the details of a proof, or become concerned that they should have been able to construct this proof on their own. To alleviate this potential problem, we do not include the proofs to most theorems in the text. The interested reader is highly encouraged to find proofs online or from their instructor. In most cases, one is very capable of understanding what a theorem \textit{means} and \textit{how to apply it} without knowing fully \textit{why} it is true.
\\

\thispagestyle{empty}
\noindent\textbf{\large Interactive, 3D Graphics}\\

New to Version 3.0 is the addition of interactive, 3D graphics in the .pdf version. Nearly all graphs of objects in space can be rotated, shifted, and zoomed in/out so the reader can better understand the object illustrated. 

As of this writing, the only pdf viewers that support these 3D graphics are Adobe Reader \& Acrobat (and only the versions for PC/Mac/Unix/Linux computers, not tablets or smartphones). To activate the interactive mode, click on the image. Once activated, one can click/drag to rotate the object and use the scroll wheel on a mouse to zoom in/out. (A great way to investigate an image is to first zoom in on the page of the pdf viewer so the graphic itself takes up much of the screen, then zoom inside the graphic itself.) A CTRL-click/drag pans the object left/right or up/down. By right-clicking on the graph one can access a menu of other options, such as changing the lighting scheme or perspective. One can also revert the graph back to its default view. If you wish to deactive the interactivity, one can right-click and choose the ``Disable Content'' option. \\

\noindent\textbf{\large Thanks}\\

There are many people who deserve recognition for the important role they have played in the development of this text. First, I thank Michelle for her support and encouragement, even as this ``project from work'' occupied my time and attention at home. Many thanks to Troy Siemers, whose most important contributions extend far beyond the sections he wrote or the 227 figures he coded in Asymptote for 3D interaction.  He provided incredible support, advice and encouragement for which I am very grateful. My thanks to Brian Heinold and Dimplekumar Chalishajar for their contributions and to Jennifer Bowen for reading through so much material and providing great feedback early on. Thanks to Troy, Lee Dewald, Dan Joseph, Meagan Herald, Bill Lowe, John David, Vonda Walsh, Geoff Cox, Jessica Libertini and other faculty of VMI who have given me numerous suggestions and corrections based on their experience with teaching from the text. (Special thanks to Troy, Lee \& Dan for their patience in teaching Calc III while I was still writing the Calc III material.) Thanks to Randy Cone for encouraging his tutors of VMI's Open Math Lab to read through the text and check the solutions, and thanks to the tutors for spending their time doing so. A very special thanks to Kristi Brown and Paul Janiczek who took this opportunity far above \& beyond what I expected, meticulously checking every solution and carefully reading every example. Their comments have been extraordinarily helpful. I am also thankful for the support provided by Wane Schneiter, who as my Dean provided me with extra time to work on this project. I am blessed to have so many people give of their time to make this book better.\\

\clearpage
\noindent\textbf{\large \apex\  -- Affordable Print and Electronic teXts}\\

\apex\ is a consortium of authors  who collaborate to produce high--quality, low--cost textbooks. The current textbook--writing paradigm is facing a potential revolution as desktop publishing and electronic formats increase in popularity. However, writing a good textbook is no easy task, as the time requirements alone are substantial. It takes countless hours of work to produce text, write examples and exercises, edit and publish. Through collaboration, however, the cost to any individual can be lessened, allowing us to create texts that we freely distribute electronically and sell in printed form for an incredibly low cost. Having said that, nothing is entirely free; someone always bears some cost. This text ``cost'' the authors of this book their time, and that was not enough. \textit{APEX Calculus} would not exist had not the Virginia Military Institute, through a generous Jackson--Hope grant, given the lead author significant time away from teaching so he could focus on this text.

Each text is available as a free .pdf, protected by a Creative Commons Attribution - Noncommercial 4.0 copyright. That  means you can give the .pdf to anyone you like, print it in any form you like, and even edit the original content and redistribute it. If you do the latter, you must  clearly reference this work and you cannot sell your edited work for money.

We encourage others to adapt this work to fit their own needs. One might add sections that are ``missing'' or remove sections that your students won't need. The source files can be found at \texttt{\href{https://github.com/APEXCalculus}{github.com/APEXCalculus}}.

You can learn more at \texttt{\href{http://www.vmi.edu/APEX}{www.vmi.edu/APEX}}.
\thispagestyle{empty}

