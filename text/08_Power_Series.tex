\section{Power Series}\label{sec:power_series}

So far, our study of series has examined the question of ``Is the sum of these infinite terms finite?,'' i.e., ``Does the series converge?'' We now approach series from a different perspective: as a function. Given a value of $x$, we evaluate $f(x)$ by finding the sum of a particular series that depends on $x$ (assuming the series converges). We start this new approach to series with a definition.

\definition{def:power_series}{Power Series}
{Let $\{a_n\}$ be a sequence, let $x$ be a variable, and let $c$ be a real number.
\index{power series}\index{series!power}
	\begin{enumerate}
		\item The \sword{power series in $x$} is the series
		$$\sum_{n=0}^\infty a_nx^n = a_0+a_1x+a_2x^2+a_3x^3+\ldots$$
		
		\item The \sword{power series in $x$ centered at $c$} is the series
		$$\sum_{n=0}^\infty a_n(x-c)^n = a_0+a_1(x-c)+a_2(x-c)^2+a_3(x-c)^3+\ldots$$
	\end{enumerate}
}

\example{ex_ps1}{Examples of power series}{
Write out the first five terms of the following power series:\\

\noindent $\ds 1.\ \sum_{n=0}^\infty x^n \qquad\qquad 2.\ \sum_{n=1}^\infty (-1)^{n+1}\frac{(x+1)^n}n\qquad\qquad 3.\ \sum_{n=0}^\infty (-1)^{n+1} \frac{(x-\pi)^{2n}}{(2n)!}.$
}
{\begin{enumerate}
	\item One of the conventions we adopt is that $x^0=1$ regardless of the value of $x$. Therefore
	$$\sum_{n=0}^\infty x^n = 1+x+x^2+x^3+x^4+\ldots$$
	This is a geometric series in $x$.
	
	\item	This series is centered at $c=-1$. Note how this series starts with $n=1$. We could rewrite this series starting at $n=0$ with the understanding that $a_0=0$, and hence the first term is $0$.
	$$\sum_{n=1}^\infty (-1)^{n+1}\frac{(x+1)^n}n = (x+1) - \frac{(x+1)^2}{2} + \frac{(x+1)^3}{3} - \frac{(x+1)^4}{4}+\frac{(x+1)^5}{5}\ldots$$
	
	\item		This series is centered at $c=\pi$. Recall that $0!=1$.
	$$\sum_{n=0}^\infty (-1)^{n+1} \frac{(x-\pi)^{2n}}{(2n)!} = -1+\frac{(x-\pi)^2}{2} - \frac{(x-\pi)^4}{24}+ \frac{(x-\pi)^6}{6!}-\frac{(x-\pi)^8}{8!}\ldots $$
\end{enumerate}
\vskip-1.5\baselineskip
}\\

We introduced power series as a type of function, where a value of $x$ is given and the sum of a series is returned. Of course, not every series converges. For instance, in part 1 of Example \ref{ex_ps1}, we recognized the series $\ds \sum_{n=0}^\infty x^n$ as a geometric series in $x$. Theorem \ref{thm:geom_series} states that this series converges only when $|x|<1$. 

This raises the question: ``For what values of $x$ will a given power series converge?,'' which  leads us to a theorem and definition.

\theorem{thm:radius_converge}{Convergence of Power Series}
{Let a power series $\ds \sum_{n=0}^\infty a_n(x-c)^n$ be given. Then one of the following is true:
\index{convergence!of power series}\index{power series!convergence}\index{series!power}
\begin{enumerate}
	\item The series converges only at $x=c$.
	\item	There is an $R>0$ such that the series converges for all $x$ in \\	
	$(c-R,c+R)$ and diverges for all $x<c-R$ and $x>c+R$.
	\item	The series converges for all $x$.
\end{enumerate}
}

%A corollary to Theorem \ref{thm:radius_converge} is this: given a series $\ds \sum_{n=0}^\infty a_n(x-c)^n$ for some real number $c$, one of the following is true:
	%\begin{enumerate}
		%\item The series converges only at $x=c$.
		%\item	There is an $R>0$ such that the series converges for all $x$ in $(c-R, c+R)$ and diverges for all $x<c-R$ and $x>c+R$.
		%\item	The series converges for all $x$.
	%\end{enumerate}
	
	The value of $R$ is important when understanding a power series, hence it is given a name in the following definition. Also, note that part 2 of Theorem \ref{thm:radius_converge} makes a statement about the interval $(c-R,c+R)$, but the not the endpoints of that interval. A series may/may not converge at these endpoints.
	
	\definition{def:radius_converge}{Radius and Interval of Convergence}
	{  \begin{enumerate}
		\item The number $R$ given in Theorem \ref{thm:radius_converge} is the \sword{radius of convergence} of a given series. When a series converges for only $x=c$, we say the radius of convergence is 0, i.e.,  $R=0$. When a series converges for all $x$, we say the series has an infinite radius of convergence, i.e., $R=\infty$.
			\item	The \sword{interval of convergence} is the set of all values of $x$ for which the series converges.
			\index{convergence!radius of}\index{convergence!interval of}\index{radius of convergence}\index{interval of convergence}\index{series!radius of convergence}\index{series!interval of convergence}
	\end{enumerate}
	}

To find the values of $x$ for which a given series converges, we will use the convergence tests we studied previously (especially the Ratio Test). However, the tests all required that the terms of a series be positive. The following theorem gives us a work--around to this problem.

\theorem{thm:abs_power}{\parbox[t]{210pt}{The Radius of Convergence of a Series and Absolute Convergence}}
{%{\begin{itemize}
	%\item The series $\ds \sum_{n=0}^\infty a_nx^n$ and $\ds \sum_{n=0}^\infty \big|a_nx^n\big|$ have the same radius of convergence $R$.
	%\item 
	The series $\ds \sum_{n=0}^\infty a_n(x-c)^n$ and $\ds \sum_{n=0}^\infty \big|a_n(x-c)^n\big|$ have the same radius of convergence $R$.
%\end{itemize}
}
	
Theorem \ref{thm:abs_power} allows us to find the radius of convergence $R$ of a series by applying the Ratio Test (or any applicable test) to the absolute value of the terms of the series. We practice this in the following example.\\

\example{ex_ps2}{Determining the radius and interval of convergence.}{
Find the radius and interval of convergence for each of the following series:\\

\noindent$\ds 1.\ \sum_{n=0}^\infty \frac{x^n}{n!} \qquad 2.\ \sum_{n=1}^\infty (-1)^{n+1}\frac{x^n}{n}\qquad 3.\ \sum_{n=0}^\infty 2^n(x-3)^n\qquad 4.\ \sum_{n=0}^\infty n!x^n$
}
{\begin{enumerate}
	\item We apply the Ratio Test to the series $\ds \sum_{n=0}^\infty \left|\frac{x^n}{n!}\right|$:
		\begin{align*}
		\lim_{n\to\infty} \frac{\big|x^{n+1}/(n+1)!\big|}{\big|x^n/n!\big|} &= \lim_{n\to\infty} \left|\frac{x^{n+1}}{x^n}\cdot\frac{n!}{(n+1)!}\right|\\
			&= \lim_{n\to\infty} \left|\frac x{n+1}\right|\\
			&= 0 \text{ for all } x.
		\end{align*}
		The Ratio Test shows us that regardless of the choice of $x$, the series converges. Therefore the radius of convergence is $R=\infty$, and the interval of convergence is $(-\infty,\infty)$.
		
	\item		We apply the Ratio Test to the series $\ds \sum_{n=1}^\infty \left|(-1)^{n+1}\frac{x^n}{n}\right| = \sum_{n=1}^\infty \left|\frac{x^n}{n}\right|$:
	\begin{align*}
	\lim_{n\to\infty} \frac{\big|x^{n+1}/(n+1)\big|}{\big|x^n/n\big|} &= \lim_{n\to\infty} \left|\frac{x^{n+1}}{x^n}\cdot \frac{n}{n+1}\right| \\
			&= \lim_{n\to\infty} |x|\frac{n}{n+1}\\
			&= |x|.
	\end{align*}
	The Ratio Test states a series converges if the limit of $|a_{n+1}/a_n| = L<1$. We found the limit above to be $|x|$; therefore, the power series converges when $|x| <1$, or when $x$ is in $(-1,1)$. Thus the radius of convergence is $R=1$.
	\drawexampleline
	
	To determine the interval of convergence, we need to check the endpoints of $(-1,1)$. When $x=-1$, we have the opposite of the Harmonic Series:
	\begin{align*}
	\sum_{n=1}^\infty (-1)^{n+1}\frac{(-1)^n}{n} &= \sum_{n=1}^\infty \frac{-1}{n}\\
				&= -\infty.
	\end{align*}
	The series diverges when $x=-1$.
	
	When $x=1$, we have the series $\ds \sum_{n=1}^\infty (-1)^{n+1}\frac{(1)^n}{n}$, which is the Alternating Harmonic Series, which converges. Therefore the interval of convergence is $(-1,1]$.
	
	\item		We apply the Ratio Test to the series $\ds\sum_{n=0}^\infty \big|2^n(x-3)^n\big|$:
	\begin{align*}
	\lim_{n\to\infty} \frac{\big| 2^{n+1}(x-3)^{n+1}\big|}{\big|2^n(x-3)^n\big|} &= \lim_{n\to\infty} \left|\frac{2^{n+1}}{2^n}\cdot\frac{(x-3)^{n+1}}{(x-3)^n}\right|\\
			&=\lim_{n\to\infty} \big|2(x-3)\big|.
	\end{align*}
	
According to the Ratio Test, the series converges when $\big|2(x-3)\big|<1 \implies \big|x-3\big| < 1/2$. The series is centered at 3, and $x$ must be within $1/2$ of 3 in order for the series to converge. Therefore the radius of convergence is $R=1/2$, and we know that the series converges absolutely for all $x$ in $(3-1/2,3+1/2) = (2.5, 3.5)$.

We check for convergence at the endpoints to find the interval of convergence. When $x=2.5$, we have:
\begin{align*}
\sum_{n=0}^\infty 2^n(2.5-3)^n &= \sum_{n=0}^\infty 2^n(-1/2)^n \\
			&=\sum_{n=0}^\infty (-1)^n,
\end{align*}
which diverges. A similar process shows that the series also diverges at $x=3.5$. Therefore the interval of convergence is $(2.5, 3.5)$.

\item		We apply the Ratio Test to $\ds \sum_{n=0}^\infty \big|n!x^n\big|$:
\begin{align*}
\lim_{n\to\infty} \frac{\big| (n+1)!x^{n+1}\big|}{\big|n!x^n\big|} &= \lim_{n\to\infty} \big|(n+1)x\big|\\
		&= \infty\ \text{ for all $x$, except $x=0$.}
\end{align*}

The Ratio Test shows that the series diverges for all $x$ except $x=0$. Therefore the radius of convergence is $R=0$.
\end{enumerate}
\vskip-1.5\baselineskip
}\\

We can use a power series to define a function:
$$f(x) = \sum_{n=0}^\infty a_nx^n$$
where the domain of $f$ is a subset of the interval of convergence of the power series. One can apply calculus techniques to such functions; in particular, we can find derivatives and antiderivatives. 

\theorem{thm:calc_power_series}{\parbox[t]{210pt}{Derivatives and Indefinite Integrals of Power Series Functions}}
{Let $\ds f(x) = \sum_{n=0}^\infty a_n(x-c)^n$ be a function defined by a power series, with radius of convergence $R$.
	\begin{enumerate}
		\item $f(x)$ is continuous and differentiable on $(c-R,c+R)$.
		\item	$\ds \fp(x) = \sum_{n=1}^\infty a_n\cdot n\cdot (x-c)^{n-1}$, with radius of convergence $R$.
		\item	$\ds \int f(x)\ dx = C+\sum_{n=0}^\infty a_n\frac{(x-c)^{n+1}}{n+1}$, with radius of convergence $R$.
		\index{series!power!derivatives and integrals}\index{integration!of power series}\index{derivative!power series}\index{power series!derivatives and integrals}
	\end{enumerate}
}

A few notes about Theorem \ref{thm:calc_power_series}:
		\begin{enumerate}
			\item The theorem states that differentiation and integration do not change the radius of convergence. It does not state anything about the \emph{interval} of convergence. They are not always the same.
			\item		Notice how the summation for $\fp(x)$ starts with $n=1$. This is because the constant term $a_0$ of $f(x)$ goes to 0.
			\item	Differentiation and integration are simply calculated term--by--term using the Power Rules.
		\end{enumerate}

\example{ex_ps3}{Derivatives and indefinite integrals of power series}{
Let $\ds f(x) = \sum_{n=0}^\infty x^n$. Find $\fp(x)$ and $\ds F(x) =\int f(x)\ dx$, along with their respective intervals of convergence.
}
{We find the derivative and indefinite integral of $f(x)$, following Theorem \ref{thm:calc_power_series}.\\

\begin{enumerate}

\item $\ds \fp(x) = \sum_{n=1}^\infty nx^{n-1} = 1+2x+3x^2+4x^3+\cdots.$

In Example \ref{ex_ps1}, we recognized that $\ds \sum_{n=0}^\infty x^n$ is a geometric series in $x$. We know that such a geometric series converges when $|x|<1$; that is, the interval of convergence is $(-1,1)$.

To determine the interval of convergence of $\fp(x)$, we consider the endpoints of $(-1,1)$:
$$\fp(-1) =  1-2+3-4+\cdots,\quad \text{which diverges.}$$
$$\fp(1) = 1+2+3+4+\cdots,\quad \text{which diverges.}$$

Therefore, the interval of convergence of $\fp(x)$ is $(-1,1)$. 

\item $\ds F(x) = \int f(x)\ dx = C+\sum_{n=0}^\infty \frac{x^{n+1}}{n+1} = C+ x+\frac{x^2}{2}+\frac{x^3}3+\cdots$

To find the interval of convergence of $F(x)$, we again consider the endpoints of $(-1,1)$:
$$F(-1) = C-1+1/2-1/3+1/4+\cdots$$
The value of $C$ is irrelevant; notice that the rest of the series is an Alternating Series that whose terms converge to 0. By the Alternating Series Test, this series converges. (In fact, we can recognize that the terms of the series after $C$ are the opposite of the Alternating Harmonic Series. We can thus say that $F(-1) = C-\ln 2$.)
$$F(1) = C+1+1/2+1/3+1/4+\cdots$$
Notice that this summation is $C\ +$ the Harmonic Series, which diverges. Since $F$ converges for $x=-1$ and diverges for $x=1$, the interval of convergence of $F(x)$ is $[-1,1)$.
\end{enumerate}
\vskip-1.5\baselineskip
}\\

The previous example showed how to take the derivative and indefinite integral of a power series without motivation for why we care about such operations. We may care for the sheer mathematical enjoyment ``that we can'', which is motivation enough for many. However, we would be remiss to not recognize that we can learn a great deal from taking derivatives and indefinite integrals.\\ 

Recall that $\ds f(x) = \sum_{n=0}^\infty x^n$ in Example \ref{ex_ps3} is a geometric series. According to Theorem \ref{thm:geom_series}, this series converges to $1/(1-x)$ when $|x|<1$. Thus we can say
$$	f(x) = \sum_{n=0}^\infty x^n = \frac 1{1-x},\quad \text{ on }\quad (-1,1).$$

Integrating the power series, (as done in Example \ref{ex_ps3},) we find
\begin{equation} F(x)  = C_1+\sum_{n=0}^\infty \frac{x^{n+1}}{n+1},\label{eq:ps3a}\end{equation}
while integrating the function $f(x) = 1/(1-x)$ gives
\begin{equation} F(x)  = -\ln|1-x| + C_2.\label{eq:ps3b}\end{equation}

Equating Equations \eqref{eq:ps3a} and \eqref{eq:ps3b}, we have 
$$F(x) = C_1+\sum_{n=0}^\infty \frac{x^{n+1}}{n+1} = -\ln|1-x| + C_2.$$
Letting $x=0$, we have $F(0) = C_1 = C_2$. This implies that we can drop the constants and conclude
$$\sum_{n=0}^\infty \frac{x^{n+1}}{n+1} = -\ln|1-x|.$$
We established in Example \ref{ex_ps3} that the series on the left converges at $x=-1$; substituting $x=-1$ on both sides of the above equality gives
$$-1+\frac12-\frac13+\frac14-\frac15+\cdots = -\ln 2.$$
On the left we have the opposite of the Alternating Harmonic Series; on the right, we have $-\ln 2$. We conclude that 
$$1-\frac12+\frac13-\frac14+\cdots = \ln 2.$$
\sword{Important:} We stated in Key Idea \ref{idea:famous_series} (in Section \ref{sec:series}) that the Alternating Harmonic Series converges to $\ln 2$, and referred to this fact again in Example \ref{ex_alt1} of Section \ref{sec:alt_series}. However, we never gave an argument for why this was the case. The work above finally shows how we conclude that the Alternating Harmonic Series converges to $\ln 2$. \index{Alternating Harmonic Series}

We use this type of analysis in the next example.\\

\example{ex_ps4}{Analyzing power series functions}{
Let $\ds f(x) = \sum_{n=0}^\infty \frac{x^n}{n!}$. Find $\ds \fp(x)$ and $\ds \int f(x)\ dx$, and use these to analyze the behavior of $f(x)$.
}
{We start by making two notes: first, in Example \ref{ex_ps2}, we found the interval of convergence of this power series is $(-\infty,\infty)$. Second, we will find it useful later to have a  few terms of the series written out:
\begin{equation}\sum_{n=0}^\infty \frac{x^n}{n!} = 1 + x + \frac{x^2}2+\frac{x^3}{6} + \frac{x^4}{24} +\cdots\label{eq:ps4}\end{equation}

We now find the derivative:
\begin{align*}
\fp(x) &= \sum_{n=1}^\infty n\frac{x^{n-1}}{n!} \\
&=\sum_{n=1}^\infty \frac{x^{n-1}}{(n-1)!} = 1+x+\frac{x^2}{2!}+\cdots. 
\intertext{Since the series starts at $n=1$ and each term refers to $(n-1)$, we can re-index the series starting with $n=0$:}
		&= \sum_{n=0}^\infty \frac{x^{n}}{n!}\\
		&= f(x).
\end{align*}
We found the derivative of $f(x)$ is $f(x)$. The only functions for which this is true are of the form $y=ce^x$ for some constant $c$. As $f(0) = 1$ (see Equation \eqref{eq:ps4}), $c$ must be 1. Therefore we conclude that 
$$f(x) = \sum_{n=0}^\infty \frac{x^n}{n!} = e^x$$% \quad\text{for all $x$}.$$
for all $x$.

We can also find $\ds \int f(x)\ dx$:
\begin{align*}
\int f(x)\ dx &= C+\sum_{n=0}^\infty \frac{x^{n+1}}{n!(n+1)} \\
				&= C+ \sum_{n=0}^\infty \frac{x^{n+1}}{(n+1)!}
\end{align*}
We write out a few terms of this last series:
$$C+ \sum_{n=0}^\infty \frac{x^{n+1}}{(n+1)!} = C+ x+ \frac{x^2}2+\frac{x^3}{6}+\frac{x^4}{24}+\cdots$$
The integral of $f(x)$ differs from $f(x)$ only by a constant, again indicating that $f(x) = e^x$.
}\\


Example  \ref{ex_ps4} and the work following Example \ref{ex_ps3} established relationships between a power series function and ``regular'' functions that we have dealt with in the past. In general, given a power series function, it is difficult (if not impossible) to express the function in terms of elementary functions. We chose examples where things worked out nicely.

In this section's last example, we show how to solve a simple differential equation with a power series.\\

\example{ex_ps5}{Solving a differential equation with a power series.}
{Give the first 4 terms of the power series solution to $y\primeskip' = 2y$, where $y(0) = 1$.
}
{The differential equation $y\primeskip' = 2y$ describes a function $y=f(x)$ where the derivative of $y$ is twice $y$ and $y(0)=1$. This is a rather simple differential equation; with a bit of thought one should realize that if $y=Ce^{2x}$, then $y\primeskip' = 2Ce^{2x}$, and hence $y\primeskip' = 2y$. By letting $C=1$ we satisfy the initial condition of $y(0)=1$.

Let's ignore the fact that we already know the solution and find a power series function that satisfies the equation. The solution we seek will have the form
$$f(x)  = \sum_{n=0}^\infty a_nx^n = a_0+a_1x+a_2x^2+a_3x^3+\cdots$$
for unknown coefficients $a_n$. We can find $\fp(x)$ using Theorem \ref{thm:calc_power_series}:
$$\fp(x) = \sum_{n=1}^\infty a_n\cdot n\cdot x^{n-1} = a_1+2a_2x+3a_3x^2+4a_4x^3\cdots.$$
Since $\fp(x) = 2f(x)$, we have
\begin{align*}
a_1+2a_2x+3a_3x^2+4a_4x^3\cdots &= 2\big(a_0+a_1x+a_2x^2+a_3x^3+\cdots\big)\\
			&=2a_0+2a_1x+2a_2x^2+2a_3x^3+\cdots
\end{align*}
The coefficients of like powers of $x$ must be equal, so we find that
$$a_1 = 2a_0,\quad 2a_2 = 2a_1,\quad 3a_3 = 2a_2,\quad 4a_4 = 2a_3,\quad \text{etc.}$$
The initial condition $y(0) = f(0) = 1$ indicates that $a_0 = 1$; with this, we can find the values of the other coefficients:
\begin{align*}
a_0 = 1 \text{ and } a_1=2a_0 &\Rightarrow a_1 = 2;\\
a_1 = 2 \text{ and } 2a_2 = 2a_1 &\Rightarrow a_2=4/2 =2;\\
a_2=2 \text{ and } 3a_3 = 2a_2 &\Rightarrow a_3=8/(2\cdot3)=4/3;\\
a_3=4/3 \text{ and } 4a_4 = 2a_3 &\Rightarrow a_4 =16/(2\cdot3\cdot4)= 2/3. 
\end{align*}
Thus the first 5 terms of the power series solution to the differential equation $y\primeskip'=2y$ is 
$$f(x) = 1+ 2x+2x^2 + \frac43x^3+\frac23x^4+\cdots$$
In Section \ref{sec:taylor_series}, as we study Taylor Series, we will learn how to recognize this series as describing $y=e^{2x}$. 
}\\

Our last example illustrates that it can be difficult to recognize an elementary function by its power series expansion. It is far easier to start with a known function, expressed in terms of elementary functions, and represent it as a power series function. One may wonder why we would bother doing so, as the latter function probably seems more complicated. In the next two sections, we show both \emph{how} to do this and  \emph{why} such a process can be beneficial. 

\printexercises{exercises/08_06_exercises}