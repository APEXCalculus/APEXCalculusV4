{$C$ is the curve that follows the triangle with vertices at $(0,0,2)$, $(4,0,0)$ and $(0,3,0)$, traversing the the vertices in that order and returning to $(0,0,2)$, and $\surfaceS$ is the portion of the plane $z=2-x/2-2y/3$ enclosed by $C$; $\vec F = \langle y,-z,y\rangle$. 

{\hfill\myincludegraphicsthree{width=90pt,3Dmenu,activate=onclick,deactivate=onclick,
3Droll=0,
3Dortho=0.004999519791454077,
3Dc2c=0.7375360131263733 -0.6143473982810974 0.2803889811038971,
3Dcoo=32.48771286010742 72.419189453125 49.87675094604492,
3Droo=200.00000360071368,
3Dlights=Headlamp,add3Djscript=asylabels.js}{scale=.5}{figures/fig14_07_ex_11}\hfill}
}
{Circulation on $C$: The flow along the line from $(0,0,2)$ to $(4,0,0)$ is 0; from $(4,0,0)$ to $(0,3,0)$ it is $-6$, and from $(0,3,0)$ to $(0,0,2)$ it is 6. The total circulation is $0+(-6)+6=0$.

$\iint_\surfaceS\big(\curl \vec F\big)\cdot\vec n\ dS = \iint_\surfaceS 0 \ dS = 0$.
}
