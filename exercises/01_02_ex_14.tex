{$\displaystyle \lim_{x\to 1} \frac1x = 1$}
{Let $\epsilon >0$ be given. We wish to find $\delta >0$ such that when $|x-1|<\delta$, $|f(x)-1|<\epsilon$. 

Consider $|f(x)-1|<\epsilon$, keeping in  mind we want to make a statement about $|x-1|$:
\begin{gather*}
|f(x) -1 | < \epsilon \\
|1/x-1 |<\epsilon \\
| (1-x)/x | < \epsilon \\
| x-1 |/|x| < \epsilon \\
| x-1 | < \epsilon\cdot|x| \\
\end{gather*}
Since $x$ is near 1, we can safely assume that, for instance, $1/2<x<3/2$. Thus
$\epsilon/2 < \epsilon\cdot x $.

Let $\delta =\frac{\epsilon}{2}$. Then:
\begin{gather*}
|x-1|<\delta \\
|x-1| < \frac{\epsilon}{2}\\
|x-1| < \epsilon\cdot x\\
|x-1|/x < \epsilon\\
\end{gather*}
Assuming $x$ is near 1, $x$ is positive and we can bring it into the absolute value signs on the left.
\begin{gather*}
|(x-1)/x| < \epsilon\\
|1-1/x| < \epsilon\\
|(1/x) -1| < \epsilon,
\end{gather*}
which is what we wanted to prove.
}


