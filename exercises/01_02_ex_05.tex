{$\displaystyle \lim_{x\to 3} \big(x^2-3\big) = 6$}
{Let $\epsilon >0$ be given. We wish to find $\delta >0$ such that when $|x-3|<\delta$, $|f(x)-6|<\epsilon$. 

Consider $|f(x)-6|<\epsilon$, keeping in  mind we want to make a statement about $|x-3|$:
\begin{gather*}
|f(x) -6 | < \epsilon \\
|x^2-3 -6 |<\epsilon \\
| x^2-9 | < \epsilon \\
| x-3 |\cdot|x+3| < \epsilon \\
| x-3 | < \epsilon/|x+3| \\
\end{gather*}
Since $x$ is near 3, we can safely assume that, for instance, $2<x<4$. Thus
\begin{gather*}
2+3<x+3<4+3 \\
5 < x+3 < 7 \\
\frac{1}{7} < \frac{1}{x+3} < \frac{1}{5} \\
\frac{\epsilon}{7} < \frac{\epsilon}{x+3} < \frac{\epsilon}{5} \\
\end{gather*}
Let $\delta =\frac{\epsilon}{7}$. Then:
\begin{gather*}
|x-3|<\delta \\
|x-3| < \frac{\epsilon}7\\
|x-3| < \frac{\epsilon}{x+3}\\
|x-3|\cdot|x+3| < \frac{\epsilon}{x+3}\cdot|x+3|\\
\end{gather*}
Assuming $x$ is near 3, $x+3$ is positive and we can drop the absolute value signs on the right.
\begin{gather*}
|x-3|\cdot|x+3| < \frac{\epsilon}{x+3}\cdot(x+3)\\
|x^2-9| < \epsilon\\
|(x^2-3) - 6| < \epsilon,
\end{gather*}
which is what we wanted to prove.
}


