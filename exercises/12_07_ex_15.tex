{$\ds f(x,y) = x^2+2x+y^2+2y$, constrained to the region bounded by the circle $x^2+y^2=4$.
}
{The region has no ``corners'' or ``vertices,'' just a smooth edge.\\
To find critical points along the circle $x^2+y^2=4$, we solve for $y^2$: $y^2=4-x^2$. We can go further and state $y=\pm\sqrt{4-x^2}$. 

We can rewrite $f$ as $f(x)=x^2+2x + (4-x^2) + 2\sqrt{4-x^2} = 2x+4+2\sqrt{4-x^2}$. (We will return and use $-\sqrt{4-x^2}$ later.) Solving $f\,'(x)=0$, we get $x=\sqrt{2} \Rightarrow y=\sqrt{2}$. $f\,'(x)$ is also undefined at $x=\pm 2$, where $y=0$. 

Using $y=-\sqrt{4-x^2}$, we rewrite $f(x,y)$ as $f(x) = 2x+4-2\sqrt{4-x^2}$. Solving $\fp(x) =0$, we get $x=-\sqrt{2},\ y=-\sqrt{2}$. Again, $f\,'(x)$ is undefined at $x=\pm 2$.

The function $z=f(x,y)$ itself has a critical point at $(-1,-1)$. \\
Checking the value of $f$ at $(-1,-1)$, $(\sqrt{2},\sqrt{2})$, $(-\sqrt{2},-\sqrt{2})$, $(2,0)$ and $(-2,0)$, we find the absolute maximum is at $(\sqrt2,\sqrt2,4+4\sqrt2)$ and the absolute minimum is at $(-1,-1,-2)$.
}