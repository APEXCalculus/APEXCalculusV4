{David faces Goliath with only a stone in a 3ft sling,  which he whirls above his head at 4 revolutions per second. They stand 20ft apart.
\begin{enumerate}
	\item [(a)]	At what $t$-values must David release the stone in his sling in order to hit Goliath?
	\item	[(b)] What is the speed at which the stone is traveling when released?
	\item	[(c)] Assume David releases the stone from a height of 6ft and Goliath's forehead is 9ft above the ground. What angle of elevation must David apply to the stone to hit Goliath's head?
\end{enumerate}
}
{The stone, while whirling, can be modeled by $\vrt = \la 3\cos(8\pi t),3\sin (8\pi t)\ra$. 
\begin{enumerate}
	\item For $t$-values $t=\sin^{-1}(3/20)/(8\pi) + n/4 \approx 0.006 + n/4$, where $n$ is an integer.
	\item		$\norm{\vrp(t)} = 24\pi\approx 51.4$ft/s
	\item		At $t=0.006$, the stone is approximately $19.77$ft from Goliath. Using the formula for projectile motion, we want the angle of elevation that lets a projectile starting at $\la 0,6\ra$ with a initial velocity of $51.4$ft/s arrive at $\la 19.77,9\ra$. The desired angle is $0.27$ radians, or $15.69^\circ$.
\end{enumerate}
}