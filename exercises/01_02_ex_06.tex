{$\displaystyle \lim_{x\to 2} \big(x^3-1\big) = 7$}
{Let $\epsilon >0$ be given. We wish to find $\delta >0$ such that when $|x-2|<\delta$, $|f(x)-7|<\epsilon$. 

Consider $|f(x)-7|<\epsilon$, keeping in  mind we want to make a statement about $|x-2|$:
\begin{gather*}
|f(x) -7 | < \epsilon \\
|x^3-1 -7 |<\epsilon \\
| x^3-8 | < \epsilon \\
| x-2 |\cdot|x^2+2x+4| < \epsilon \\
| x-3 | < \epsilon/|x^2+2x+4| \\
\end{gather*}
Since $x$ is near 2, we can safely assume that, for instance, $1<x<3$. Thus
\begin{gather*}
1^2+2\cdot1+4<x^2+2x+4<3^2+2\cdot3+4 \\
7 < x^2+2x+4 < 19 \\
\frac{1}{19} < \frac{1}{x^2+2x+4} < \frac{1}{7} \\
\frac{\epsilon}{19} < \frac{\epsilon}{x^2+2x+4} < \frac{\epsilon}{7} \\
\end{gather*}
Let $\delta =\frac{\epsilon}{19}$. Then:
\begin{gather*}
|x-2|<\delta \\
|x-2| < \frac{\epsilon}{19}\\
|x-2| < \frac{\epsilon}{x^2+2x+4}\\
|x-2|\cdot|x^2+2x+4| < \frac{\epsilon}{x^2+2x+4}\cdot|x^2+2x+4|\\
\end{gather*}
Assuming $x$ is near 2, $x^2+2x+4$ is positive and we can drop the absolute value signs on the right.
\begin{gather*}
|x-2|\cdot|x^2+2x+4| < \frac{\epsilon}{x^2+2x+4}\cdot(x^2+2x+4)\\
|x^3-8| < \epsilon\\
|(x^3-1) - 7| < \epsilon,
\end{gather*}
which is what we wanted to prove.
}



