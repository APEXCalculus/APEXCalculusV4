{It is ``common sense'' that it is far better to measure a long distance with a long measuring tape rather than a short one. A measured distance $D$ can be viewed as the product of the length $\ell$ of a measuring tape times the number $n$ of times it was used. For instance, using a 3' tape 10 times gives a length of 30'. To measure the same distance with a 12' tape, we would use the tape 2.5 times. (I.e., $30=12\times 2.5$.) Thus $D = n\ell$.\\

Suppose each time a measurement is taken with the tape, the recorded distance is within 1/16'' of the actual distance. (I.e., $d\ell = 1/16'' \approx 0.005$ft). Using differentials, show why common sense proves correct in that it is better to use a long tape to measure long distances.
}
{With $D = n\ell$, the total differential is $dD = \ell\, dn+ n\,d\ell.$ If one measures with a short tape, $n$ must be large and hence $n\,d\ell$ is going to be greater than when a large tape is used (wherein $n$ will be small).
}