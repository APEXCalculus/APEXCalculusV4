{\textbf{Projectile Motion:} The $x$-value of an object moving under the principles of projectile motion is $x(\theta,v_0,t)= (v_0\cos\theta)t$. A particular projectile is fired with an initial velocity of $v_0=250$ft/s and an angle of elevation of $\theta = 60^\circ$. It travels a distance of $375$ft in 3 seconds.\\

Is the projectile more sensitive to errors in initial speed or angle of elevation?
}
{Distance of the projectile is a function of two variables (leaving $t=3$): $D(v_0,\theta) = 3\v_0\cos\theta$. The total differential of $D$ is $dD = 3\cos\theta dv_0-3v_0\sin\theta d\theta$. The coefficient of $d\theta$ has a much greater magnitude than the coefficient of $dv_0$, so a small change in the angle of elevation has a much greater effect on distance traveled than a small change in initial velocity.
}